\documentclass[thesis]{subfiles}

\begin{document}

\chapter{\pyop3}

the top level API:
loop expressions, made of statements
also kernels, access descriptors


% not sure where this goes, bit too specific to go here, but it probably is a
% language/DSL thing not an indexing thing since it affects how many loops we have
\subsection{Context-sensitive ...}

context-sensitive things: maps and loops

% this sort of feels like it should be its own chapter...
\section{Code generation}
% compilation pipeline, transformations then lowering then more transformations

\subsection{\pyop3 transformations}

expand loop contexts

pack unpack transforms

\subsection{Lowering to loopy}
% inames etc

\subsection{Loopy transformations}

don't think we actually do any of these currently...

\section{PETSc integration}

also not sure that this is the right place for this section, maybe put inside an "other features" section?

Arguably this could even go after where we discuss parallel. Most of the content is focused on axis tree arrays so this is arguably a confusing distraction. Should probably have a section per chapter on PETsc matrices as there are relevant bits per chapter that apply.

Need to discuss raxes, caxes, and tabulating the rmap and cmap, they are materialised indexed things (like we have for temporaries).

\end{document}
