\documentclass[thesis]{subfiles}

\begin{document}

\chapter{The execution model}
\label{chapter:execution_model}

% TODO: Discuss how data dependencies are (not) handled.
% TODO: It's written in Python, but that's fine because it's a compiler - Julia/another faster language would not help us
% TODO: embedded DSL is used for interoperability with other libraries, and Python is everywhere, can also use parser of parent language (reference UFL for its thoughts on this?)

Thus far we have established a new abstraction for mesh-like data structures, and an approach for symbolically representing smaller ``packed" parts of them.
In order for \pyop3 to be a usable library, rather than just an interesting abstraction to consider, three problems remain:

\begin{itemize}
  \item How are the actual data structures stored in memory?
  \item How does one express operations to be executed?
  \item How are these operations executed?
\end{itemize}

These questions will be answered in this chapter, giving us a fully capable, serial-only, \pyop3 library.

\section{Data structures}
\label{sec:data_structures}

Thus far we have only discussed the \textit{specification} of how data is stored in \pyop3 and not the actual implementation.
For continuum mechanics problems one typically needs to have representations for scalars, vectors and matrices.
In \pyop3, recycling the terminology from \pyop2, we call scalars \pycode{Globals}, vectors \pycode{Dats} and matrices \pycode{Mats}.
All of these data structures work in parallel, and their parallel implementation is deferred to \cref{chapter:parallel}.

\subsection{Scalars (\pycode{Globals})}

\pycode{Globals} are the simplest of \pyop3's data structures.
They wrap a single scalar value, which may be of any data type (e.g. \pycode{int32}, \pycode{float64}, \pycode{complex128}) and thus have a trivial data layout, hence they have no need for axis trees.
It is not valid to index into a \pycode{Global} (\cref{chapter:indexing}).

\subsection{Vectors (\pycode{Dats})}

Thus far, all of the data structures that we have encountered would be stored as \pycode{Dats}.
\pycode{Dats} are constructed with a single axis tree that provides the information necessary to address the underlying flat array that carries the data.
Having a single axis tree, \pycode{Dats} may be indexed using a single index (\cref{chapter:indexing}).

Currently \pycode{Dats} use numpy arrays as the underlying data storage mechanism, but we intend to permit further array types to enable targeting accelerator architectures like CUDA GPUs (see \cref{sec:future_work}).

\subsection{Matrices (\pycode{Mats})}

\pycode{Mats} require 2 axis trees: one for the rows of the matrix and one for the columns.
They rely on PETSc \ccode{Mat} objects for the underlying data storage.
To improve performance one should preallocate the matrix by constructing a \pycode{Sparsity} object and doing a simulated run of all the loop expressions so that non-zeros are put in the right places.

% NOTE: This sort of thing needs to be clearer - we index axis trees, so they have two...
Since \pycode{Mats} have two axis trees, two indices are needed when indexing.

\section{The domain-specific language}
\label{sec:dsl}

% forms a tree, show a more complex example

\subsection{Loop expressions}
% loop expressions, made of a loop index and statements
% can be context-sensitive

% how do they manifest? how do we generate code from them?
% the axes turn into loops - this can be done without mentioning loopy

\begin{figure}[h]
  \centering

  \hfill
  %
  \begin{subfigure}[t]{.3\textwidth}
    \centering
    \begin{minipage}{.8\textwidth}
      \begin{minted}{python}
        loop(
          p := axes.index(),
          do_something(...),
        )
      \end{minted}
    \end{minipage}
    %
    \vspace{1em}
  \end{subfigure}
  %
  \hfill
  %
  \begin{subfigure}[t]{.38\textwidth}
    \centering
    \begin{minipage}{\textwidth}
      \centering
      \includegraphics{loop_expr_axes.pdf}
    \end{minipage}
  \end{subfigure}
  %
  \hfill
  %
  \begin{subfigure}[t]{.3\textwidth}
    \centering
    \begin{minipage}{.8\textwidth}
      \begin{minted}{python}
        for ia in range(5):
          for ib in range(3):
            do_something(...)
      \end{minted}
    \end{minipage}
    %
    \vspace{1.5em}
  \end{subfigure}
  %
  \hfill

  \caption{
    A simple example of generating code from a loop expression.
    The loop expression (left) loops over axis tree \pycode{axes} and performs some computation (\pycode{do_something}) for each point in the iteration.
    \pycode{axes} is defined to be a two-dimensional linear axis tree with shape \pycode{(5, 3)} and axis labels $a$ and $b$ (middle).
    Pseudocode for the code that is generated from this expression is shown to the right.
    Loop indices \pycode{ia} and \pycode{ib} correspond to looping over axes $a$ and $b$ respectively.
  }
  \label{fig:loop_expr}
\end{figure}

\subsubsection{Context-sensitive loops}

\begin{figure}[h]
  \centering

  \begin{subfigure}[t]{.32\textwidth}
    \centering
    % extra lines to improve position on page
    \begin{minted}{python}
loop(
  p := axes.index(), ...
)


    \end{minted}
    %
    \caption{TODO}
    \label{fig:multi_component_loop_expr_init}
  \end{subfigure}
  %
  \begin{subfigure}[t]{.32\textwidth}
    \centering
    \includegraphics{multi_component_loop_expr_axes.pdf}
    %
    \caption{TODO}
    \label{fig:multi_component_loop_expr_axes}
  \end{subfigure}
  %
  \begin{subfigure}[t]{.32\textwidth}
    \centering
    % extra lines to improve position on page
    \begin{minted}{python}
for ia in range(2):
  for ib in range(3):
    ...

for ia in range(2):
  for ic in range(2):
    ...
    \end{minted}
    %
    \caption{TODO}
    \label{fig:multi_component_loop_expr_codegen}
  \end{subfigure}

  \caption{TODO}
  \label{fig:multi_component_loop_expr}
\end{figure}

% terminals! be more generic
\subsection{Kernels}

% kernels
% access descriptors (vague)

\section{Code generation}
\label{sec:codegen}

\begin{figure}[h]
  \centering
  \includegraphics{codegen_flowchart.pdf}
  \caption{
    The code generation pipeline for the compilation of a loop expression into a callable function.
    The input (``Loop expression") and output (``Compiled function") are shown in blue whilst the intermediate processes are red.
    The dashed line from input expression to output function is included to represent the fact that the compiled function additionally requires data from the input loop expression in order to be usable.
  }
  \label{fig:codegen_flowchart}
\end{figure}

% transformations can be split into functional ones and optimisation ones

% dual role of loop expression: codegen spec *and* data to pass through (ref. figure)

To aid with the explanation, we will take a straightforward loop expression typical of finite element style codes and demonstrate the transformation and lowering stages of the compilation pipeline as they apply to it:

\begin{center}
  \begin{minipage}{.38\textwidth}
    \begin{minted}{python}
      loop(
        p := a.index(),
        kernel(dat0[map0(p), :], dat1[p]),
      )
    \end{minted}
  \end{minipage}
\end{center}

In this loop expression a number of terms require further explanation.
Firstly, \pycode{dat0} and \pycode{dat1} are defined to be arrays with shape \pycode{(8, 3)} and \pycode{(5,)} respectively, with their axis trees appearing as follows:

\begin{center}
  \includegraphics{codegen_example_dat_axes.pdf}
\end{center}

The axis tree used to construct \pycode{dat1} (right) is the same as the one used in the loop index \pycode{p := a.index()}.
We therefore expect that the generated loop will have the following structure (since axis $a$ has 5 entries):

\begin{center}
  \begin{minipage}{.25\textwidth}
    \begin{minted}{python}
      for ia in range(5):
        kernel(...)
    \end{minted}
  \end{minipage}
\end{center}

\pycode{dat1} is entirely indexed by the loop index \pycode{p}, and so the indexed array \pycode{dat1[p]} only has size 1.
The situation for \pycode{dat0} is more complicated.
\pycode{map0} is a map from axis $a$ to axis $x$ with arity 2, and the slice notation ``\pycode{:}" indicates a full slice over the inner axis $y$.
The indexed object \pycode{dat0[map0(p), :]} passed through to \pycode{kernel} therefore has size 6.

Lastly, the local kernel (\pycode{kernel}) is defined to be some function taking two arguments with size 6 and intent \pycode{READ}, and size 1 and intent \pycode{INC} respectively.

\subsection{Loop expression transformations}

\begin{table}
  \centering

  \begin{tabular}{|c|l|l|}
    \hline
    \textbf{Intent} & \textbf{Pack instruction} & \textbf{Unpack instruction} \\
    \hline
    \pycode{READ} & \ccode{write(temporary, indexed)} & \tableDash \\
    \hline
    \pycode{WRITE} & \tableDash & \ccode{write(indexed, temporary)} \\
    \hline
    \pycode{RW} & \ccode{write(temporary, indexed)} & \ccode{write(indexed, temporary)} \\
    \hline
    \pycode{INC} & \ccode{write(temporary, 0)} & \ccode{inc(indexed, temporary)} \\
    \hline
    \pycode{MIN_WRITE} & \tableDash & \ccode{min(indexed, temporary)} \\
    \hline
    \pycode{MIN_INC} & \ccode{write(temporary, 0)} & \ccode{min(indexed, temporary)} \\
    \hline
    \pycode{MAX_WRITE} & \tableDash & \ccode{max(indexed, temporary)} \\
    \hline
    \pycode{MAX_INC} & \ccode{write(temporary, 0)} & \ccode{max(indexed, temporary)} \\
    \hline
  \end{tabular}

  \caption{
    Intent values supported by \pyop3 kernels and their corresponding pack/unpack instructions.
    In the instructions, the variable ``\ccode{indexed}" is used to represent the indexed view of some piece of global data (e.g. \pycode{dat0[map0(p)]}) and the variable ``\ccode{temporary}" is the temporary buffer for storing the materialised data.
    Table entries marked with a ``\pycode{-}" indicate that no pack/unpack instruction is emitted for this intent.
  }
  \label{tab:intents}
\end{table}

\begin{figure}
  \centering
  \includegraphics{loop_expr_tree_transform.pdf}

  \caption{
    The expression tree transformation expanding implicit pack and unpack instructions for the example loop expression (\cref{sec:codegen}).
    The input loop expression is shown on the left and the output, expanded, loop expression is shown on the right.
    \pycode{kernel} has argument intents of \pycode{READ} and \pycode{INC} for its first and second argument respectively and so the transformed expression contains \pycode{write} and \pycode{inc} instructions where appropriate.
  }
  \label{fig:loop_expr_tree_transform}
\end{figure}

% explain implicit pack/unpack expansion, reference table (and explain notation) and reference figure (demo)

\subsection{Lowering loop expressions to loopy kernels}

\begin{figure}
  \centering
  \begin{minipage}{.8\textwidth}
    \inputminted{text}{./scripts/artefacts/codegen_example_loopy_kernel_tidy.txt}
  \end{minipage}
  \caption{
    Abbreviated textual representation of the loopy kernel generated for the example expression (\cref{sec:codegen}).
  }
  \label{lst:codegen_example_loopy_kernel}
\end{figure}

% single pass
% tree is traversed in a pre-order fashion (parents visited first)
% loops emit domains and perhaps assignments (ragged)
% then terminals (i.e. function calls and assignments) also emit domains and assignments

% explain substituted target expressions

\subsection{Compilation and execution of loopy kernels}
\label{sec:codegen_loopy_kernel}

\begin{figure}
  \centering
  \begin{minipage}{.8\textwidth}
    \inputminted{c}{./scripts/artefacts/codegen_example_c_code_tidy.c}
  \end{minipage}
  \caption{
    TODO
  }
  \label{lst:codegen_example_c_code}
\end{figure}

Once at the level of a loopy kernel, the rest of the compilation becomes straightforward.
Depending on the \textit{target} attribute belonging to a kernel, loopy can generate an appropriate string of C code that \pyop3 writes to a file and compiles with a traditional C compiler (e.g. gcc).
Once compiled, \pyop3 can load the function pointer from the shared object file, allowing it to be executed.
This process is unchanged from \pyop2.

For our demo the generated C code can be seen in \cref{lst:codegen_example_c_code}.
Unsurprisingly it looks very similar to the input loopy kernel (\cref{lst:codegen_example_loopy_kernel}).

\paragraph{Note}

Despite being included in the compilation flowchart (\cref{fig:codegen_flowchart}), no transformations are currently applied at the level of the loopy kernel.
Transformations at this level would include operations like intra-element vectorisation~\cite{sunStudyVectorizationMatrixfree2020}, or enabling the generated code to run on GPUs~\cite{fenics2021-kulkarni}.
These things are considered to be future work (see \cref{sec:future_work}).

\end{document}
