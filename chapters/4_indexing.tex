\documentclass[thesis]{subfiles}

\begin{document}

\chapter{Indexing}
\label{chapter:indexing}

Just like \numpy{} (\cref{sec:numpy_indexing_arrays}), data structures in \pyop3 can be indexed, extracting portions of the full structure and returning them as a new object.
However, unlike \numpy{} where indexing is accomplished with a list of index-like objects, one for each axis, \pyop3 needs a more complicated solution to accommodate for the fact that data structures are now tree-like.
Instead of lists of indices, \pyop3 instead uses \emph{index trees}.

\section{Index trees}
\label{sec:index_trees}

In the same way that axis trees are built from labelled axis objects and axis components, index trees are built from labelled \emph{index} objects and \emph{index components}.
Whereas axes and axis components encode levels in data layouts, indices in index trees encode \emph{axis transformations} of the form
\begin{equation}
  \textnormal{axis} \quad \circ \quad \textnormal{index} \quad \to \quad (\textnormal{axis tree},\ \textnormal{index information}).
  \label{eq:index_transform}
\end{equation}
That is, each index takes an input axis and transforms it, returning a new axis tree alongside additional information encoding the relation between the old and new axes.

When dealing with index trees, as opposed to individual indices, this process expands to index entire axis trees, giving the transformation
\begin{equation}
  \label{eq:index_tree_apply}
  \textnormal{axis tree} \quad \circ \quad \textnormal{index tree} \quad \to \quad (\textnormal{axis tree},\ \textnormal{index information}).
\end{equation}
In other words, to index an axis tree, an index tree must be applied to it, with the result being an \emph{indexed axis tree} consisting of a new axis tree plus additional indexing information mapping the new axis tree back to the original one.

\begin{algorithm}
  \caption{
    Algorithm that indexes an axis tree using an index tree.
    The function is initially called passing the root of the index tree.
  }

  \begin{flushright}
    \begin{minipage}{.96\textwidth}
  \begin{pyalg2}
    def index_axes(index: Index, input_axis_tree: AxisTree):
      # process the current index
      axis_tree, index_info = index_handler(index, input_axis_tree)?\label{code:index_handler}?

      # recursively visit child indices
      for component in index.components:
        if has_subindex(index, component):
          subindex = get_subindex(index, component)
          subaxis_tree, subindex_info = index_axes(subindex,
                                                   input_axis_tree)
          axis_tree.add_subtree(subaxis_tree)
          index_info.update(subindex_info)

      return axis_tree, index_info
  \end{pyalg2}
    \end{minipage}
  \end{flushright}

  \label{alg:index_axis_tree}
\end{algorithm}

Pseudocode for the indexing transformation that occurs when indexing an axis tree with an index tree (\cref{eq:index_tree_apply}) is shown in \cref{alg:index_axis_tree}.
The index tree is traversed and the function \pycode{index_handler()}, representing the transformation in \cref{eq:index_transform}, is called on each index (line~\ref{code:index_handler}), with the resulting axis tree and index information collected together and returned.
Together these attributes are called an \emph{indexed axis tree}.

\subsubsection{Mapping between axis trees}

The index information returned by the indexing transformation allows one to relate the old and new axis trees.
In particular, it enables one to take multi-indices given for the new axis tree (the \emph{source}) and map them to multi-indices for the old tree (the \emph{target}).
The index information consists of two mappings:
\begin{description}
  \item[Target path] A map connecting labels of the source tree with labels of the target tree.
  \item[Target expressions] A map connecting index expressions of the source tree with those of the target tree.
\end{description}

With these mapping, one has sufficient information to:
(a) select the right layout function from the target axis tree, and
(b) substitute index expressions into the layout function, resulting in a \emph{substituted layout function} that accepts multi-indices of the source axis tree instead of the target.
By taking this symbolic approach to indexing, \pyop3 can represent all supported indexing operations as views, regardless of the indexing operation applied.
This contrasts with \numpy{} where only `simple' indexing operations are supported without copying the array.

\subsubsection{Example 1: Linear index trees} \label{example:linear_index_tree}

\begin{figure}
  \centering
  \begin{subfigure}{.9\textwidth}
    \begin{pyalg2}
      # set up the axis tree and array
      axis_a = Axis(5, "a")
      axis_b = Axis(3, "b")
      axes = AxisTree.from_iterable([axis_a, axis_b])
      dat = Dat(axes)

      # index the array
      indexed_dat = dat[::2, 1::]?\label{code:index_linear_apply}?
    \end{pyalg2}

    \caption{
      \pyop3 indexing code.
      The syntax used to index \pycode{dat} (\pycode{[::2, 1::]}) is syntactic sugar that is expanded to a complete index tree.
      The syntax is identical to the equivalent \numpy{} code for an array with shape \pycode{(5, 3)}.
    }
    \label{fig:index_linear_code}
  \end{subfigure}

  \vspace{1em}

  \begin{subfigure}{\textwidth}
    \centering
    \includegraphics[width=\textwidth]{index_linear_tree_data_layout.pdf}
    \caption{
      The axis tree transformation.
      The relation between old and new axis indices are indicated by the labels on the indexed axis tree (right).
      For example, the notation $c_2/a_4$ means that axis entry $c_2$ maps to $a_4$ in the original axis tree.
    }
    \label{fig:index_linear_transform}
  \end{subfigure}

  \caption{
    The axis tree transformation resulting from indexing a linear axis tree with shape \pycode{(5, 3)} with slices \pycode{[::2]} and \pycode{[1::]} on axes $a$ and $b$ respectively.
    The resulting axis tree has shape \pycode{(3, 2)} and different labels: $c$ and $d$.
  }
  \label{fig:index_linear}
\end{figure}

To illustrate axis tree indexing with a simple example, we index a linear axis tree, consisting of two axes labelled $a$ and $b$, using two slices, where slices have the same semantics as they do in \numpy{} (\cref{sec:numpy_indexing_arrays}).
The resulting axis tree is smaller than the original and has new labels: $c$ and $d$.
Code for the transformation is shown in \cref{fig:index_linear_code}, and a diagramatic representation is shown in \cref{fig:index_linear_transform}.

As well as creating a new axis tree, indexing the array produces the following index information:
\begin{center}
  \begin{tblr}{|[1pt]c|[1pt]c|[1pt]}
    \hline[1pt]
    \textbf{Target path} & \textbf{Target expressions} \\
    \hline[1pt]
    $\{ c, d \} \to \{a, b\}$ & $\{i_a = 2 i_c,\ i_b = i_d+1\}$ \\
    \hline[1pt]
  \end{tblr}
\end{center}
The target path encodes the fact that the path through the source tree $\{c,d\}$ is equivalent to the path through the target tree $\{a,b\}$ and the target expressions capture the semantics of the slices.
For example, indexing axis $a$ with the slice \pycode{[::2]} means that entries in the new axis $c$ map to even entries in $a$ - which is exactly the index expression $i_a = 2i_c$.

Given this information, we may now construct a \emph{substituted layout function} for addressing the original array using multi-indices of the indexed axis tree.
First, the target path tells us which layout function to use from the original array, which in this case is
\begin{equation}
  \textnormal{offset}(i_a, i_b) = 3 i_a + i_b.
\end{equation}
Then, applying the substitutions from the target expressions, this produces the substituted layout function
\begin{equation}
  \textnormal{offset}(i_c, i_d) = 6 i_c + i_d + 1,
\end{equation}
which maps multi-indices in the source axis tree to offsets in the original array as desired.

\subsubsection{Example 2: Multi-component index trees}

\begin{figure}
  \centering

  \begin{subfigure}{.9\textwidth}
    \begin{pyalg2}
      # set up the axis tree and array
      axis_a = Axis({"x": 3, "y": 2}, "a")
      axis_b = Axis(2, "b")
      axis_c = Axis(3, "c")
      axes = AxisTree.from_nest({axis_a: [axis_b, axis_c]})
      dat = Dat(axes)

      # create the index tree
      index_a = Slice("a", [Subset("x", [0, 2]),
                            AffineSliceComponent("y", start=1)])
      index_b = ScalarIndex("b", 1)
      index_c = Slice("c", [Subset([1, 2])])
      index_tree = IndexTree.from_nest({index_a: [index_b, index_c]})?\label{code:multi_component_make_index_tree}?

      # index the array
      indexed_dat = dat[index_tree]
    \end{pyalg2}

    \caption{\pyop3 indexing code.}
    \label{fig:multi_component_slice_code}
  \end{subfigure}

  \vspace{1em}

  \begin{subfigure}{\textwidth}
    \centering
    \includegraphics[scale=.9]{multi_component_slice_transform.pdf}
    \caption{
      The axis tree transformation.
    }
    \label{fig:multi_component_slice_transform_flowchart}
  \end{subfigure}

  \caption{
    The axis tree transformation resulting from indexing a multi-component axis tree with a multi-component index tree.
  }
  \label{fig:multi_component_slice}
\end{figure}

As a further example of indexing transformations we consider a multi-component index tree applied to a multi-component axis tree.
Shown in \cref{fig:multi_component_slice}, there are a number of changes compared with the linear case:
\begin{itemize}
  \item
    The slice over axis $a$ now has two components, one slicing the $x$ axis component and one the $y$ component.
    This means that the indexed axis tree (\cref{fig:multi_component_slice_transform_flowchart}, right) has two components also.

  \item
    Some of the slices now include \emph{subsets}.
    Instead of providing \pycode{start}, \pycode{stop}, and \pycode{step} arguments an integer array is provided.
    This is equivalent to `integer array indexing' for \numpy{} (\cref{sec:numpy_indexing_arrays})\footnote{A possible source of confusion between \pyop3 and \numpy{} slices is that in \pyop3 slices refer generically to the operation of indexing a subset of an axis, either affinely (equivalent to a regular \numpy{} slice) or using a lookup table (equivalent to \numpy{}'s integer array indexing).}.

  \item
    Instead of a slice, axis $b$ is indexed with a \emph{scalar index}.
    This accesses just one of the entries in $b$ and so the index transformation \emph{consumes} the axis and it does not appear in the indexed axis tree.

  \item
    Unlike the linear case, the multi-component nature of the index tree means that it is no longer possible to hide its instantiation behind syntactic sugar like in \cref{example:linear_index_tree} and it must instead be explicitly constructed (\cref{fig:multi_component_slice_code}, line~\ref{code:multi_component_make_index_tree}).
\end{itemize}

Indexing the array produces the following target path and target expressions:
\begin{center}
  \begin{tblr}{|[1pt]c|[1pt]c|[1pt]}
    \hline[1pt]
    \textbf{Target path} & \textbf{Target expressions} \\
    \hline[1pt]
    $\{d^t\} \to \{a^x,b\}$ & $\{i_a = \pycode{[0,2][?$i_d$?]},\ i_b = 1\}$ \\
    \hline[1pt]
    $\{d^u,e\} \to \{a^y,c\}$ & $\{i_a = i_d + 1,\ i_c = \pycode{[1,2][?$i_e$?]}\}$ \\
    \hline[1pt]
  \end{tblr}
\end{center}
Unlike the linear case above, having subsets introduces index expressions like $\pycode{[0,2][?$i_d$?]}$, and the scalar index of axis $b$ consumes an axis, meaning that $i_b$ is simply a constant instead of an expression of some other axis index.

With these target paths and expressions it is now straightforward to again determine appropriate substituted layouts for the indexed axis tree using the same process as before:
\begin{center}
  \begin{tblr}{|[1pt]c|[1pt]l|[1pt]l|[1pt]}
    \hline[1pt]
    \textbf{Path} & \SetCell{c}\textbf{Original layout} & \SetCell{c}\textbf{Substituted layout} \\
    \hline[1pt]
    $\{a^x, b\}$ & $\textnormal{offset}(i_a, i_b) = 2 i_a + i_b$ & $\textnormal{offset}(i_d) = 2 ( \pycode{[0,2][?$i_d$?]} ) + 1$ \\
    \hline[1pt]
    $\{a^y, c\}$ & $\textnormal{offset}(i_a, i_c) = 3 i_a + i_c + 6$ & $\textnormal{offset}(i_d, i_e) = 3(i_d+1) + \pycode{[1,2][?$i_e$?]} + 6$ \\
    \hline[1pt]
  \end{tblr}
\end{center}

\subsection{Index composition}
\label{sec:index_composition}

Since indexing an axis tree just gives back another axis tree, it is possible to repeat the process and index the axis tree again.
This is called \emph{index composition}.
An example composition operation is shown in \cref{fig:index_composition}.
Using the indexed array from \cref{fig:index_linear} we index it again to produce another indexed axis tree that remains capable of mapping multi-indices back to the original target axis tree.

\begin{figure}
  \centering

  \begin{subfigure}{.9\textwidth}
    \begin{pyalg2}
      # use indexed_dat from ?\cref{fig:index_linear}?
      indexed_dat = ...

      # index the array again
      indexed_dat2 = indexed_dat[1::, 1]
    \end{pyalg2}

    \caption{\pyop3 indexing code.}
    \label{fig:index_composition_code}
  \end{subfigure}

  \vspace{1em}

  \begin{subfigure}{\textwidth}
    \centering
    \includegraphics{index_composition_transform.pdf}
    \caption{The axis tree transformation.}
    \label{fig:index_composition_transform}
  \end{subfigure}

  \caption{
    The composition of an already indexed axis tree (from \cref{fig:index_linear}) with another index tree.
  }
  \label{fig:index_composition}
\end{figure}

If we disregard any prior indexing information and simply treat the transformation as we would a `normal' indexing operation, treating the input axis tree as an unindexed axis tree with labels $c$ and $d$, we get the following index information:
\begin{center}
  \begin{tblr}{|[1pt]c|[1pt]c|[1pt]}
    \hline[1pt]
    \textbf{Target path} & \textbf{Target expressions} \\
    \hline[1pt]
    $\{e\} \to \{ c, d \}$ & $\{ i_c = i_e + 1,\ i_d = 1 \}$ \\
    \hline[1pt]
  \end{tblr}
\end{center}
From \cref{example:linear_index_tree} we know that the index information for the input (indexed) axis tree is given by:
\begin{center}
  \begin{tblr}{|[1pt]c|[1pt]c|[1pt]}
    \hline[1pt]
    \textbf{Target path} & \textbf{Target expressions} \\
    \hline[1pt]
    $\{ c, d \} \to \{a, b\}$ & $\{i_a = 2 i_c,\ i_b = i_d+1\}$ \\
    \hline[1pt]
  \end{tblr}
\end{center}
Which may now be \emph{composed} with the new indexing information to produce the new set of index information:
\begin{center}
  \begin{tblr}{|[1pt]c|[1pt]c|[1pt]}
    \hline[1pt]
    \textbf{Target path} & \textbf{Target expressions} \\
    \hline[1pt]
    $\{e\} \to \{a, b\}$ & $\{i_a = 2 (i_e+1),\ i_b = 2\}$ \\
    \hline[1pt]
  \end{tblr}
\end{center}
We can now perform the same sort of layout substitution as before:
\begin{center}
  \begin{tblr}{|[1pt]c|[1pt]l|[1pt]l|[1pt]}
    \hline[1pt]
    \textbf{Path} & \SetCell{c}\textbf{Original layout} & \SetCell{c}\textbf{Substituted layout} \\
    \hline[1pt]
    $\{a,b\}$ & $\textnormal{offset}(i_a, i_b) = 3 i_a + i_b$ & $\textnormal{offset}(i_e) = 6 (i_e+1) + 2$ \\
    \hline[1pt]
  \end{tblr}
\end{center}
Giving us a way to map from the doubly-indexed axis tree back to the original.

\section{Outer loops}
\label{sec:outer_loops}

\begin{figure}
  \centering

  \begin{subfigure}{.9\textwidth}
    \begin{pyalg2}
      # set up the axis tree and array
      axis_a = Axis(5, "a")
      axis_b = Axis(3, "b")
      axes = AxisTree.from_iterable([axis_a, axis_b])
      dat = Dat(axes)

      # create the "outer" loop index
      p = axis_a.index()?\label{code:loop_index_init}?

      # index the array
      indexed_dat = dat[p, :]?\label{code:loop_index_apply}?
    \end{pyalg2}

    \caption{\pyop3 indexing code.}
    \label{fig:loop_index_code}
  \end{subfigure}

  \vspace{1em}

  \begin{subfigure}{\textwidth}
    \centering
    \includegraphics{loop_index_linear_tree_data_layout.pdf}
    \caption{
      The axis tree transformation.
    }
    \label{fig:loop_index_linear_tree_data_layout}
  \end{subfigure}

  \vspace{1em}

  \begin{subfigure}{.9\textwidth}
    \begin{pyalg2}
      # create an empty array with shape (5, 3)
      array = numpy.empty((5, 3))
      for p in range(5):           # the "outer" loop
        array[p, :]                # index one axis of the array with p
    \end{pyalg2}
    \caption{\numpy{} code performing an equivalent operation.}
    \label{fig:loop_index_numpy}
  \end{subfigure}
  \caption{
    The axis tree transformation resulting from indexing with a loop index.
  }
  \label{fig:loop_index_linear_tree_data_layout_all}
\end{figure}

For mesh stencil calculations it is often the case that one has `outer loops' over some set of mesh entities, accessing global data relative to the current iterate.
\pyop3 has full support for expressing this sort of iteration.

In \pyop3, outer loops may be created over any axis tree via the instantiation of a \emph{loop index}, constructed by calling the \pycode{index()} method.
Indexing axis trees with loop indices is similar to using scalar indices because the loop index consumes an axis.
However, they differ in the fact that a scalar index sets the index expression to a scalar value, whereas loop indices pin the index to the value of the outer loop iterate.

An example demonstrating loop index usage is shown in \cref{fig:loop_index_linear_tree_data_layout_all}.
In it one indexes a \pycode{Dat} (\pycode{dat}) with shape \pycode{(5, 3)} and labels $a$ and $b$ using a loop index over the axis $a$ (line~\ref{code:loop_index_init}), and a full slice over axis $b$.
\numpy{} code for an equivalent operation is shown in \cref{fig:loop_index_numpy}.

In order to map between source and target axes we follow the same process as before.
This time, the index information is:
\begin{center}
  \begin{tblr}{|[1pt]c|[1pt]c|[1pt]}
    \hline[1pt]
    \textbf{Target path} & \textbf{Target expressions} \\
    \hline[1pt]
    $\{b\} \to \{a, b\}$ & $\{i_a = L^p_a,\ i_b = i_b\}$ \\
    \hline[1pt]
  \end{tblr}
\end{center}
Where the index expression $L^p_a$ has been introduced to represent loop index $p$'s loop over axis $a$.
Note that axis $b$ has not been relabelled in the indexed array as a full slice was taken.
Determining a substituted layout for the indexed array is then straightforward:
\begin{center}
  \begin{tblr}{|[1pt]c|[1pt]l|[1pt]l|[1pt]}
    \hline[1pt]
    \textbf{Path} & \SetCell{c}\textbf{Original layout} & \SetCell{c}\textbf{Substituted layout} \\
    \hline[1pt]
    $\{a,b\}$ & $\textnormal{offset}(i_a, i_b) = 3 i_a + i_b$ & $\textnormal{offset}(L^p_a,i_b) = 3 L^p_a + i_b$ \\
    \hline[1pt]
  \end{tblr}
\end{center}

It is important to note that, unlike with scalar indices and slices, arrays that have been indexed with loop indices have loop index terms (e.g. $L^p_a$) in their substituted layouts.
Practically this means that it is not possible to evaluate offsets - and thus interpret - the array without knowing the value of these terms.
Since the terms are provided externally, we say that arrays indexed with loop indices are \emph{context-sensitive}, as they only make sense within the context of an outer loop.

\section{Maps}

\begin{figure}
  \centering

  \begin{subfigure}{.9\textwidth}
    \begin{pyalg2}
      # set up the axis tree and array
      axis_a = Axis(5, "a")
      axis_b = Axis(3, "b")
      axes = AxisTree.from_iterable([axis_a, axis_b])
      dat = Dat(axes)

      # create the "outer" loop
      axis_x = Axis(10, "x")?\label{code:make_axis_x}?
      p = axis_x.index()

      # prepare a mapping from axis "x" to "a"
      f = Map({"x": [MapComponent("a", arity=3, ...)]})?\label{code:make_map}?

      # index the array
      indexed_dat = dat[f(p), :]
    \end{pyalg2}

    \caption{\pyop3 indexing code.}
    \label{fig:index_map_code}
  \end{subfigure}

  \vspace{1em}

  \begin{subfigure}{\textwidth}
    \centering
    \includegraphics{index_map_data_layout.pdf}
    \caption{The axis tree transformation.}
    \label{fig:index_map_data_layout}
  \end{subfigure}

  \caption{Index transformation representing the packing of an axis tree using the map $f$, which has arity 3.}
  \label{fig:index_map}
\end{figure}

Indexing with loop indices is an important step towards generating expressions for stencil codes.
However, they are not sufficient to produce `packed' temporaries of the sort shown in \cref{fig:scott_vogelius_element_dof_layout_packed} - for that we need \emph{maps}.

In \pyop3, maps are functions that accept a single loop index as an argument (or another map, \cref{sec:indexing_map_composition}) and yield values that are parametrised by the loop index.
The number of values returned by the map for each iterate is called the \emph{arity}; for example, the map from cells to incident edges ($\cone(p)$, \cref{sec:dmplex_queries}) for a mesh with triangular cells has an arity of 3.

An example of indexing an array using a map is shown in \cref{fig:index_map}.
We declare an outer loop over an axis labelled $x$ (\ref{fig:index_map_code}, line~\ref{code:make_axis_x}) as well as an arity 3 map, $f$, mapping from $x$ to $a$ (line~\ref{code:make_map}).
\pycode{dat} is then indexed with $f$, which has been \emph{called} by passing it the loop index (i.e. $f(p)$).
The resulting axis tree (\ref{fig:index_map_data_layout}, right) has lost axis $a$ but gained a new axis in its place, labelled $c$, containing the 3 entries indicated in the map for a given $p$.
Unlike loop indices, maps contribute new axes to the indexed axis tree.

The treatment of the indexing information to retain view-like semantics remain exactly the same as before.
For this example the returned index information is:
\begin{center}
  \begin{tblr}{|[1pt]c|[1pt]c|[1pt]}
    \hline[1pt]
    \textbf{Target path} & \textbf{Target expressions} \\
    \hline[1pt]
    $\{c, b\} \to \{a, b\}$ & $\{i_a = f(L^p_x, i_c),\ i_b = i_b\}$ \\
    \hline[1pt]
  \end{tblr}
\end{center}
Where $f(L^p_x, i_c)$ is used to represent the map.
$L^p_x$ is the outer loop index over axis $x$ and $i_c$ is an index for the new axis $c$.
The final, substituted layout function can then be generated:
\begin{center}
  \begin{tblr}{|[1pt]c|[1pt]l|[1pt]l|[1pt]}
    \hline[1pt]
    \textbf{Path} & \SetCell{c}\textbf{Original layout} & \SetCell{c}\textbf{Substituted layout} \\
    \hline[1pt]
    $\{a,b\}$ & $\textnormal{offset}(i_a, i_b) = 3 i_a + i_b$ & $\textnormal{offset}(L^p_x,i_b, i_c) = 3 f(L^p_x, i_c) + i_b$ \\
    \hline[1pt]
  \end{tblr}
\end{center}

Note that we have not specified the form of the map function $f$.
This is intentional because the function could either be a lookup with an indirection map, something like $f(L^p_x,i_c) \coloneq \pycode{[...][?$L^p_x$?,?$i_c$?]}$, or it could instead be some affine function of the form $f(L^p_x,i_c) \coloneq C \cdot L^p_x + i_c + D$ (for constants $C$ and $D$).
By treating maps more generally instead of eagerly committing to a particular representation \pyop3 should be able to work without change for both of these cases.
This is useful in cases where both methods would ideally be supported, such as when dealing with both structured (affine function) and unstructured (indirection map) meshes, behind a single interface.

\subsection{Ragged maps}
\label{sec:indexing_ragged_maps}

\begin{figure}
  \centering

  \begin{subfigure}{.9\textwidth}
    \begin{pyalg2}
      # set up the axis tree and array
      dat = Dat(Axis(4, "a"))

      # create the "outer" loop and loop index
      axis_x = Axis(3, "x")
      p = axis_x.index()

      # create a ragged map mapping from "x" to "a"
      arity_dat = Dat(axis_x, data=[2, 0, 1])
      f = Map({"x": [MapComponent("a", arity=arity_dat, ...)]})?\label{code:arity_dat}?

      # index the array
      indexed_dat = dat[f(p)]
    \end{pyalg2}

    \caption{\pyop3 indexing code.}
    \label{fig:index_ragged_map_code}
  \end{subfigure}

  \vspace{1em}

  \begin{subfigure}{\textwidth}
    \centering
    \includegraphics{index_ragged_map_transform.pdf}
    \caption{The axis tree transformation.}
    \label{fig:index_ragged_map_transform}
  \end{subfigure}

  \caption{
    Example indexing an array with a ragged map with arity \pycode{[2, 0, 1]}.
    The resulting array has variable size depending on the value of the loop index.
  }
  \label{fig:index_ragged_map}
\end{figure}

Thus far we have only considered the case where maps have a constant value arity.
This is not always the case.
To use the same example from \cref{sec:ragged_axis_trees}, the map relating vertices in a mesh to their supported edges has variable arity because the size of this relation is different for each vertex.
As with axis trees, maps that have a variable size are also called \emph{ragged}.

To represent ragged maps a simple change in the code is required: one must pass a \pycode{Dat} instead of an integer as the \pycode{arity} argument to the map.
This is demonstrated in \cref{fig:index_ragged_map_code} (line~\ref{code:arity_dat}).
As a result of this non-constant arity, the size of the indexed array is no longer constant and instead depends on the value of the loop index.
This is shown in \cref{fig:index_ragged_map_transform} (right).

A discussion of layout functions and target paths/expressions is omitted here because it is identical to the example shown in \cref{fig:index_map} above.

\subsection{Map composition}
\label{sec:indexing_map_composition}

\begin{figure}
  \centering

  \begin{subfigure}{.9\textwidth}
    \begin{pyalg2}
      # set up the axis tree and array
      axes = AxisTree.from_iterable([Axis(5, "a"), Axis(3, "b")])
      dat = Dat(axes)

      # create the "outer" loop and loop index
      axis_x = Axis(10, "x")
      p = axis_x.index()

      # create map f, mapping "x" to "y"
      f = Map({"x": [MapComponent("y", arity=3, ...)]})

      # create map g, mapping "y" to "a"
      g = Map({"y": [MapComponent("a", arity=2, ...)]})

      # index the array
      indexed_dat = dat[g(f(p)), :]
    \end{pyalg2}

    \caption{\pyop3 indexing code.}
    \label{fig:index_map_composition_code}
  \end{subfigure}

  \vspace{1em}

  \begin{subfigure}{\textwidth}
    \centering
    \includegraphics{index_map_composition_transform.pdf}
    \caption{
      The axis tree transformation.
      The composed map $g(f(p))$ produces the two-dimensional partial axis tree with labels $c$ and $d$.
    }
    \label{fig:index_map_composition_transform}
  \end{subfigure}
  \caption{Indexing an axis tree using composed maps.}
  \label{fig:index_map_composition}
\end{figure}

As discussed in \cref{sec:dmplex_queries}, it is occasionally desirable to compose maps such that the output of one feeds directly into another.
This is natural to express in \pyop3.

As an example, consider the indexing operation shown in \cref{fig:index_map_composition}.
Axis $a$ is indexed with the composed map $g(f(p))$ where $p$ is a loop index over axis $x$, $f$ maps $x$ to $y$, and $g$ maps $y$ to $a$.
As the arities of $f$ and $g$ are 3 and 2 respectively, two axes with sizes 3 and 2 (labelled $c$ and $d$) are produced in the resulting indexed array (\ref{fig:index_map_composition_transform}, right).

Indexing the array with the composed map gives the following index information:
\begin{center}
  \begin{tblr}{|[1pt]c|[1pt]c|[1pt]}
    \hline[1pt]
    \textbf{Target path} & \textbf{Target expressions} \\
    \hline[1pt]
    $\{c, d, b\} \to \{a, b\}$ & $\{i_a = g(f(L^p_x, i_c), i_d),\ i_b = i_b\}$ \\
    \hline[1pt]
  \end{tblr}
\end{center}
Where the index corresponding to axis $c$ ($i_c$) is passed as an argument to $f$, and the one for axis $d$ ($i_d$) passed to $g$.
With this, the substituted layouts are then:
\begin{center}
  \begin{tblr}{|[1pt]c|[1pt]l|[1pt]l|[1pt]}
    \hline[1pt]
    \textbf{Path} & \SetCell{c}\textbf{Original layout} & \SetCell{c}\textbf{Substituted layout} \\
    \hline[1pt]
    $\{a, b\}$ & $\textnormal{offset}(i_a, i_b) = 3 i_a + i_b$ & $\textnormal{offset}(L^p_x,i_c, i_d, i_b) = 3 g(f(L^p_x, i_c), i_d) + i_b$ \\
    \hline[1pt]
  \end{tblr}
\end{center}

\section{Indexing with alternative data layouts}
\label{sec:indexing_data_layout_transformations}

In \cref{sec:axis_tree_alternative_layouts} we introduced the concept of representing alternative data layouts using axis trees and claimed that \pyop3 does not care about the specific ordering of axes.
Whilst this is true internally to \pyop3, more thought is required when interfacing with external code.
In particular, the local kernels evaluated during mesh stencil calculations assume a specific ordering of the input and output DoFs.
In other words, the global data structures do not have a prescribed layout, but the local ones do.

Conveniently, \pyop3's approach to indexing entirely obviates this issue.
From \cref{alg:index_axis_tree} it can be seen that the structure of the new, indexed axis tree \emph{comes entirely from the \emph{index} tree, and not from the input \emph{axis} tree}.
This means that we can prescribe the local DoF ordering through careful construction of the index tree, independent of any global data layout.

\begin{figure}
  \centering
  \begin{subfigure}{.9\textwidth}
    \begin{pyalg2}
      # set up the (flipped) axis tree and array
      axes = AxisTree.from_iterable([Axis(3, "b"), Axis(5, "a")])
      dat = Dat(axes)

      # create the "outer" loop index
      axis_x = Axis(10, "x")
      p = axis_x.index()

      # create a map mapping from axis "x" to "a"
      f = Map({"x": [MapComponent("a", arity=3, ...)]})

      # index the array
      indexed_dat = dat[f(p), :]
    \end{pyalg2}

    \caption{\pyop3 indexing code.}
    \label{fig:index_map_swap_code}
  \end{subfigure}

  \vspace{1em}

  \begin{subfigure}{\textwidth}
    \centering
    \includegraphics{index_map_swap_transform.pdf}
    \caption{
      The axis tree transformation.
      Note that the ordering of the indexed axis tree (right) is the same as \cref{fig:index_map_data_layout} even though the input data layouts are different.
    }
    \label{fig:index_map_swap_transform}
  \end{subfigure}

  \caption{
    Indexing an array with a map.
    The transformation shown is identical to \cref{fig:index_map} except the axes of the input array have been swapped.
  }
\end{figure}

To demonstrate this behaviour we consider an identical packing transformation to that shown in \cref{fig:index_map}, but with a different axis ordering for the unindexed array: swapping axes $a$ and $b$.
All other code, including the indexing code, is unchanged.

Keeping the indexing code the same produces the same axis tree and index information as \cref{fig:index_map}:
\begin{center}
  \begin{tblr}{|[1pt]c|[1pt]c|[1pt]}
    \hline[1pt]
    \textbf{Target path} & \textbf{Target expressions} \\
    \hline[1pt]
    $\{c, b\} \to \{a, b\}$ & $\{i_a = f(L^p_x, i_c),\ i_b = i_b\}$ \\
    \hline[1pt]
  \end{tblr}
\end{center}
But, since the layout functions for the unindexed object are different to before, the substituted layout is also necessarily different:
\begin{center}
  \begin{tblr}{|[1pt]c|[1pt]l|[1pt]l|[1pt]}
    \hline[1pt]
    \textbf{Path} & \SetCell{c}\textbf{Original layout} & \SetCell{c}\textbf{Substituted layout} \\
    \hline[1pt]
    $\{a, b\}$ & $\textnormal{offset}(i_a, i_b) = 5 i_b + i_a$ & $\textnormal{offset}(L^p_x,i_b, i_c) = 5 i_b + f(L^p_x, i_c)$ \\
    \hline[1pt]
  \end{tblr}
\end{center}
This demonstrates that even though the packed DoFs are exactly the same and in the same order, \emph{they have been drawn from different places in the global array}.

\section{Outlook}

At this point we now have all of the necessary ingredients for axis trees to be used as part of a mesh stencil package.
In particular, this chapter establishes:
(a) the ability to symbolically represent views of a larger array, in an analogous manner to views of \numpy{} arrays, and
(b) the ability to express `outer loops' using loop indices.
These are both essential qualities for mesh data in a stencil computation.
Loop indices let us declare loops over mesh entities and axis tree views (i.e. indexed axis trees) are appropriate for representing pack/unpack operations.

In the following chapters we take the new abstractions of axis trees and index trees and provide an implementation of them in the new library \pyop3.

\end{document}
