\documentclass[thesis]{subfiles}

\begin{document}

\chapter{Indexing}
\label{chapter:indexing}

\todo[inline]{It seems a bit clunky to have "example" sections only at the start when the whole chapter is built from examples.}

Just like \numpy, data structures in \pyop3 can be indexed.
Compared with \numpy where indexing is accomplished with a list of index-like objects, one for each axis (\cref{sec:numpy_indexing_arrays}), \pyop3 needs a more complicated solution to accomodate for the fact that data structures are now tree-like.
Instead of lists of indices, \pyop3 instead uses \emph{index trees}.

\section{Index trees}
\label{sec:index_trees}

\begin{algorithm}
  \caption{
    Algorithm that indexes an axis tree using an index tree.
    The function is initially called passing the root of the index tree.
  }

  \begin{pyalg2}
    def index_axes(index: Index, input_axis_tree: AxisTree):
      # process the current index
      axis_tree, index_info = index_handler(index, input_axis_tree)

      # recursively visit child indices
      for component in index.components:
        if has_subindex(index, component):
          subindex = get_subindex(index, component)
          subaxis_tree, subindex_info = index_axes(subindex, input_axis_tree)

          axis_tree.add_subtree(subaxis_tree)
          index_info.update(subindex_info)

      return axis_tree, index_info
  \end{pyalg2}

  \label{alg:index_axis_tree}
\end{algorithm}

In the same way that axis trees are built from labelled axis objects and axis components, index trees are built from labelled \emph{index} objects and \emph{index components}.
However, whereas axis and axis components encode the dimensions of an array, indices in index trees instead encode \emph{axis transformations} of the form

\begin{equation}
  \textnormal{axis} \quad \circ \quad \textnormal{index} \quad \to \quad (\textnormal{axis tree},\ \textnormal{index information}).
  \label{eq:index_transform}
\end{equation}

\noindent
That is, each index takes an input axis and transforms it, returning a new axis tree alongside additional information encoding the relation between the old and new axes.

This process may then be extended to indexing an entire axis tree, giving the transformation

\begin{equation*}
  \textnormal{axis tree} \quad \circ \quad \textnormal{index tree} \quad \to \quad (\textnormal{axis tree},\ \textnormal{index information}),
\end{equation*}

\noindent
this time returning a new axis tree and aggregated index information from each of the indices.

Code for this operation is shown in \cref{alg:index_axis_tree}.
The index tree is traversed and the function \pycode{index_handler()}, representing the transformation in \cref{eq:index_transform}, is called on each index, with the resulting axis tree and index information aggregated together and returned.

\subsubsection{Mapping from indexed to unindexed axis trees}

In addition to having a new axis tree with new sizes and labels, for indexing to work \pyop3 needs to know how the new and old axis trees relate.
In particular, it needs to be able to take multi-indices given for the new axis tree (the \emph{source}) and map them to multi-indices for the old tree (the \emph{target}).
Represented by the `index information' shown above, the multi-index relations are represented by two mappings:

\begin{itemize}
  \item
    \textbf{Target paths} \\
    A map connecting labels of the source tree with labels of the target.
  \item
    \textbf{Target expressions} \\
    A map connecting index expressions of the source tree with those of the target.
\end{itemize}

With these, \pyop3 has sufficient information to:
(1) select the right layout function from the target axis tree, and
(2) substitute index expressions into the layout function, resulting in a \emph{substituted layout function} that accepts multi-indices of the source axis tree instead of the target.
This approach allows \pyop3 to represent indexing operations symbolically as views, regardless of the indexing operation applied.
This contrasts with \numpy where only `simple' indexing operations are supported without copying the array.

\begin{example}{Linear index trees} \label{example:linear_index_tree}

\begin{figure}
  \centering
  \begin{subfigure}{.9\textwidth}
    \begin{pyalg2}
      # set up the axis tree and array
      axis_a = Axis(5, "a")
      axis_b = Axis(3, "b")
      axes = AxisTree.from_iterable([axis_a, axis_b])
      dat = Dat(axes)

      # index the array
      indexed_dat = dat[::2, 1::]?\label{code:index_linear_apply}?
    \end{pyalg2}

    \caption{\pyop3 indexing code.}
    \label{fig:index_linear_code}
  \end{subfigure}

  \vspace{1em}

  \begin{subfigure}{\textwidth}
    \centering
    \includegraphics{index_linear_tree_data_layout.pdf}
    \caption{
      The axis tree transformation.
    }
    \label{fig:index_linear_transform}
  \end{subfigure}

  \caption{
    The axis tree transformation resulting from indexing a linear axis tree with shape \pycode{(5, 3)} with slices \pycode{[::2]} and \pycode{[1::]} on axes $a$ and $b$ respectively.
    The resulting axis tree has shape \pycode{(3, 2)} and different labels: $c$ and $d$.
  }
  \label{fig:index_linear}
\end{figure}

To illustrate axis tree indexing with a simple example, we consider the operation shown in \cref{fig:index_linear_transform}.
In it we index a linear axis tree using two slices, one for each of the axes (labelled $a$ and $b$).
Code for the transformation is shown in \cref{fig:index_linear_code}.

From \cref{fig:index_linear} we may make several observations:

\begin{itemize}
  \item
    The indexed axis tree (\cref{fig:index_linear_transform}, right) is smaller than the unindexed one and it has new labels: $c$ and $d$ instead of $a$ and $b$.

  \item
    The mapping between old and new axis entries are indicated by their labels.
    For example, the notation $c_2/a_4$ means that axis entry $c_2$ maps to $a_4$ in the original axis tree.

  \item
    Both slices in the index tree (\cref{fig:index_linear_transform}, middle) have a single component.
    No component labels are specified as the input axis tree is linear and so there is no ambiguity.

  \item
    The \pycode{Dat} in \cref{fig:index_linear_code} is not explicitly indexed with an index tree.
    Instead it is indexed with the pair of slices \pycode{[::2, 1::]} (line~\ref{code:index_linear_apply}) that are internally converted into an index tree.
    This syntactic sugar facilitates writing concise code and means that the syntax here is identical to the equivalent \numpy code for an array with shape \pycode{(5, 3)}.
\end{itemize}

Alongside the creation of a new axis tree, indexing the array produces the following index information:

\begin{center}
  \begin{tblr}{|[1pt]c|[1pt]c|[1pt]}
    \hline[1pt]
    \textbf{Target path} & \textbf{Target expressions} \\
    \hline[1pt]
    $\{ c, d \} \to \{a, b\}$ & $\{i_a = 2 i_c,\ i_b = i_d+1\}$ \\
    \hline[1pt]
  \end{tblr}
\end{center}

Here, the target path store the fact that the path through the source tree $\{c,d\}$ is equivalent to the path through the target tree $\{a,b\}$.
The target expressions capture the semantics of the slices.
Indexing axis $a$ with the slice \pycode{[::2]}, for example, means that entries in the new axis ($c$) map to even entries in $a$ - which is exactly the index expression $i_a = 2i_c$.

Given this information, we may now construct a substituted layout function for addressing the original array with multi-indices of the indexed axis tree.
From the target path we know that the correct, unsubstituted, layout function is

\begin{equation*}
  \textnormal{offset}(i_a, i_b) = 3 i_a + i_b,
\end{equation*}

\noindent
which, applying the target expression substitutions, produces the substituted layout function

\begin{equation*}
  \textnormal{offset}(i_c, i_d) = 6 i_c + i_d + 1
\end{equation*}

\noindent
as desired.

\end{example}

\begin{example}{Multi-component index trees}

\begin{figure}
  \centering

  \begin{subfigure}{.9\textwidth}
    \begin{pyalg2}
      # set up the axis tree and array
      axis_a = Axis({"x": 3, "y": 2}, "a")
      axis_b = Axis(2, "b")
      axis_c = Axis(3, "c")
      axes = AxisTree.from_nest({axis_a: [axis_b, axis_c]})
      dat = Dat(axes)

      # create the index tree
      index_a = Slice("a", [Subset("x", [0, 2]),
                            AffineSliceComponent("y", start=1)])
      index_b = ScalarIndex("b", 1)
      index_c = Slice("c", [Subset([1, 2])])
      index_tree = IndexTree.from_nest({index_a: [index_b, index_c]})?\label{code:multi_component_make_index_tree}?

      # index the array
      indexed_dat = dat[index_tree]
    \end{pyalg2}

    \caption{\pyop3 indexing code.}
    \label{fig:multi_component_slice_code}
  \end{subfigure}

  \vspace{1em}

  \begin{subfigure}{\textwidth}
    \centering
    \includegraphics[scale=.9]{multi_component_slice_transform.pdf}
    \caption{
      The axis tree transformation.
    }
    \label{fig:multi_component_slice_transform_flowchart}
  \end{subfigure}

  \caption{
    The axis tree transformation resulting from indexing a multi-component axis tree with a multi-component index tree.
  }
  \label{fig:multi_component_slice}
\end{figure}

As a further example of indexing transformations we consider a multi-component index tree (\cref{fig:multi_component_slice}).
Compared with the linear case there are a number of changes:

\begin{itemize}
  \item
    The slice over axis $a$ now has two components, one slicing the $x$ axis component and one the $y$ component.
    This means that the indexed axis tree (\cref{fig:multi_component_slice_transform_flowchart}, right) has two components also.

  \item
    Some of the slices now include \emph{subsets}.
    Instead of providing \pycode{start}, \pycode{stop}, and \pycode{step} arguments an integer array is provided.
    This is equivalent to `integer array indexing' for \numpy (\cref{sec:numpy_indexing_arrays}).

  \item
    Instead of a slice, axis $b$ is indexed with a \emph{scalar index}.
    This accesses just one of the indices in $b$ and so the index transformation does not need to produce a new axis tree.

  \item
    Unlike the linear case, it is no longer possible to hide the index tree behind syntactic sugar and it must instead be explicitly constructed (\cref{fig:multi_component_slice_code}, line~\ref{code:multi_component_make_index_tree}).
\end{itemize}

Indexing the array as described produces the following target path and target expressions:

\begin{center}
  \begin{tblr}{|[1pt]c|[1pt]c|[1pt]}
    \hline[1pt]
    \textbf{Target path} & \textbf{Target expressions} \\
    \hline[1pt]
    $\{d^t\} \to \{a^x,b\}$ & $\{i_a = \pycode{[0,2][?$i_d$?]},\ i_b = 1\}$ \\
    \hline
    $\{d^u,e\} \to \{a^y,c\}$ & $\{i_a = i_d + 1,\ i_c = \pycode{[1,2][?$i_e$?]}\}$ \\
    \hline[1pt]
  \end{tblr}
\end{center}

Unlike the linear case above, having subsets introduces index expressions like $\pycode{[0,2][?$i_d$?]}$, and the scalar index of axis $b$ means that $i_b$ is simply a constant, rather than an expression of some other axis index.
With these target paths and expressions it is now straightforward to again determine appropriate substituted layouts for the indexed axis tree:

\begin{center}
  \begin{tblr}{|[1pt]c|[1pt]l|[1pt]l|[1pt]}
    \hline[1pt]
    \textbf{Path} & \SetCell{c}\textbf{Original layout} & \SetCell{c}\textbf{Substituted layout} \\
    \hline[1pt]
    $\{a^x, b\}$ & $\textnormal{offset}(i_a, i_b) = 2 i_a + i_b$ & $\textnormal{offset}(i_d) = 2 ( \pycode{[0,2][?$i_d$?]} ) + 1$ \\
    \hline
    $\{a^y, c\}$ & $\textnormal{offset}(i_a, i_c) = 3 i_a + i_c + 6$ & $\textnormal{offset}(i_d, i_e) = 3(i_d+1) + \pycode{[1,2][?$i_e$?]} + 6$ \\
    \hline[1pt]
  \end{tblr}
\end{center}

\end{example}

\subsection{Index composition}
\label{sec:index_composition}

\begin{figure}
  \centering

  \begin{subfigure}{.9\textwidth}
    \begin{pyalg2}
      # use indexed_dat from ?\cref{fig:index_linear}?
      indexed_dat = ...

      # index the array again
      indexed_dat2 = indexed_dat[1::, 1]
    \end{pyalg2}

    \caption{\pyop3 indexing code.}
    \label{fig:index_composition_code}
  \end{subfigure}

  \vspace{1em}

  \begin{subfigure}{\textwidth}
    \centering
    \includegraphics{index_composition_transform.pdf}
    \caption{The axis tree transformation.}
    \label{fig:index_composition_transform}
  \end{subfigure}

  \caption{
    The composition of an already indexed axis tree (from \cref{fig:index_linear}) with another index tree.
  }
  \label{fig:index_composition}
\end{figure}

Since indexing an axis tree just gives back another axis tree, it is straightforward to repeat the process and index the axis tree again.
An example composition operation is shown in \cref{fig:index_composition}.
Using the indexed array from \cref{fig:index_linear} we index it again to produce another indexed axis tree that remains capable of mapping multi-indices back to the original target axis tree.

If we disregard the prior indexing information and simply treat the transformation as we would a `normal' indexing operation, we get the following index information:

\begin{center}
  \begin{tblr}{|[1pt]c|[1pt]c|[1pt]}
    \hline[1pt]
    \textbf{Target path} & \textbf{Target expressions} \\
    \hline[1pt]
    $\{e\} \to \{ c, d \}$ & $\{ i_c = i_e + 1,\ i_d = 1 \}$ \\
    \hline[1pt]
  \end{tblr}
\end{center}

\noindent
The circumstances here are different to previous examples because the `source' axis tree (\cref{fig:index_composition_transform}, left) is already indexed, and hence more work is needed to fully map back to the unindexed target axis tree.
From \cref{example:linear_index_tree} we know that the index information for the `source' tree is given by:

\begin{center}
  \begin{tblr}{|[1pt]c|[1pt]c|[1pt]}
    \hline[1pt]
    \textbf{Target path} & \textbf{Target expressions} \\
    \hline[1pt]
    $\{ c, d \} \to \{a, b\}$ & $\{i_a = 2 i_c,\ i_b = i_d+1\}$ \\
    \hline[1pt]
  \end{tblr}
\end{center}

\noindent
Which may now be \emph{composed} with the new indexing information to produce:

\begin{center}
  \begin{tblr}{|[1pt]c|[1pt]c|[1pt]}
    \hline[1pt]
    \textbf{Target path} & \textbf{Target expressions} \\
    \hline[1pt]
    $\{e\} \to \{a, b\}$ & $\{i_a = 2 (i_e+1),\ i_b = 2\}$ \\
    \hline[1pt]
  \end{tblr}
\end{center}

\noindent
Enabling one to produce appropriate substituted layouts relating the source and target axis trees:

\begin{center}
  \begin{tblr}{|[1pt]c|[1pt]l|[1pt]l|[1pt]}
    \hline[1pt]
    \textbf{Path} & \SetCell{c}\textbf{Original layout} & \SetCell{c}\textbf{Substituted layout} \\
    \hline[1pt]
    $\{a,b\}$ & $\textnormal{offset}(i_a, i_b) = 3 i_a + i_b$ & $\textnormal{offset}(i_e) = 6 (i_e+1) + 2$ \\
    \hline[1pt]
  \end{tblr}
\end{center}

This ability to compose indexing operations means that \pyop3 is able to to represent \emph{all} indexed arrays symbolically, as views.
This contrasts with \numpy which is only capable of doing this for `simple' indexing operations and otherwise resorts to a copy.

\section{Outer loops}
\label{sec:outer_loops}

\begin{figure}
  \centering

  \begin{subfigure}{.9\textwidth}
    \begin{pyalg2}
      # set up the axis tree and array
      axis_a = Axis(5, "a")
      axis_b = Axis(3, "b")
      axes = AxisTree.from_iterable([axis_a, axis_b])
      dat = Dat(axes)

      # create the "outer" loop index
      p = axis_a.index()?\label{code:loop_index_init}?

      # index the array
      indexed_dat = dat[p, :]?\label{code:loop_index_apply}?
    \end{pyalg2}

    \caption{\pyop3 indexing code.}
    \label{fig:loop_index_code}
  \end{subfigure}

  \vspace{1em}

  \begin{subfigure}{\textwidth}
    \centering
    \includegraphics{loop_index_linear_tree_data_layout.pdf}
    \caption{
      The axis tree transformation.
    }
    \label{fig:loop_index_linear_tree_data_layout}
  \end{subfigure}
  \caption{
    The axis tree transformation resulting from indexing with a loop index.
  }
  \label{fig:loop_index_linear_tree_data_layout_all}
\end{figure}

For mesh stencil calculations it is typically the case that one has `outer loops' - loops over some iteration set - with the stencils defined relative to the current iterate.
In \pyop3, outer loops are represented by a new type of index called a \emph{loop index}.
Loop indices are constructed from axis trees by calling the \pycode{index()} method.
They represent a loop over all entities in the tree and both nested and multi-component axis trees are supported.
Symbolically, loop indices are represented by the tuple $(L^p_{x_0}, L^p_{x_1}, \dots)$, where $L^p$ is a unique identifier for the index and $x_i$ represents the set of axis labels of the axis tree getting looped over.

An example demonstrating loop index usage is shown in \cref{fig:loop_index_linear_tree_data_layout_all}.
Given two axes, $a$ and $b$, the loop index is created only for axis $a$ (line~\ref{code:loop_index_init}), which is then used to index into \pycode{dat} (line~\ref{code:loop_index_apply}).
It is equivalent to the following \numpy code:

\begin{pyinline}
  array = numpy.empty((5, 3))  # create an empty array with shape (5, 3)
  for p in range(5):           # the "outer" loop
    array[p, :]                # index one axis of the array with p
\end{pyinline}

In order to map between source and target axes we follow the same process as before.
This time, the index information is:

\begin{center}
  \begin{tblr}{|[1pt]c|[1pt]c|[1pt]}
    \hline[1pt]
    \textbf{Target path} & \textbf{Target expressions} \\
    \hline[1pt]
    $\{b\} \to \{a, b\}$ & $\{i_a = L^p_a,\ i_b = i_b\}$ \\
    \hline[1pt]
  \end{tblr}
\end{center}

\noindent
Where the index expression $L^p_a$ has been introduced to represent loop index $p$'s loop over axis $a$.
Note that axis $b$ has not been relabelled in the indexed array as a full slice was taken.

Determining a substituted layout for the indexed array is then straightforward:

\begin{center}
  \begin{tblr}{|[1pt]c|[1pt]l|[1pt]l|[1pt]}
    \hline[1pt]
    \textbf{Path} & \SetCell{c}\textbf{Original layout} & \SetCell{c}\textbf{Substituted layout} \\
    \hline[1pt]
    $\{a,b\}$ & $\textnormal{offset}(i_a, i_b) = 3 i_a + i_b$ & $\textnormal{offset}(L^p_a,i_b) = 3 L^p_a + i_b$ \\
    \hline[1pt]
  \end{tblr}
\end{center}

It is important to note that, unlike with scalar indices and slices, arrays that have been indexed with loop indices will have loop index terms (i.e. $L^p_a$) in their substituted layouts.
Practically this means that it is not possible to evaluate offsets - and thus interpret - the array without knowing the value of these terms.
Since the terms are provided externally, we say that arrays indexed with loop indices are \emph{context-sensitive}, as they only make sense within the context of an outer loop.

\section{Maps}

\begin{figure}
  \centering

  \begin{subfigure}{.9\textwidth}
    \begin{pyalg2}
      # set up the axis tree and array
      axis_a = Axis(5, "a")
      axis_b = Axis(3, "b")
      axes = AxisTree.from_iterable([axis_a, axis_b])
      dat = Dat(axes)

      # create the "outer" loop
      axis_x = Axis(10, "x")?\label{code:make_axis_x}?
      p = axis_x.index()

      # prepare a mapping from axis "x" to "a"
      f = Map({"x": [MapComponent("a", arity=3, ...)]})?\label{code:make_map}?

      # index the array
      indexed_dat = dat[f(p), :]
    \end{pyalg2}

    \caption{\pyop3 indexing code.}
    \label{fig:index_map_code}
  \end{subfigure}

  \vspace{1em}

  \begin{subfigure}{\textwidth}
    \centering
    \includegraphics{index_map_data_layout.pdf}
    \caption{The axis tree transformation.}
    \label{fig:index_map_data_layout}
  \end{subfigure}

  \caption{Index transformation representing the packing of an axis tree using the map $f$, which has arity 3.}
  \label{fig:index_map}
\end{figure}

Indexing with loop indices is an important step towards generating expressions for stencil codes.
However, they are not sufficient to produce `packed' temporaries of the sort shown in \cref{fig:scott_vogelius_element_dof_layout_packed} - for that we need \emph{maps}.

In \pyop3, maps are functions that accept a single loop index as an argument (or another map, \cref{sec:indexing_map_composition}) to produce a new axis tree parametrised by the loop index.
The size of this axis, equivalently the number of values returned by the map, is termed the \emph{arity}.
For example, the map from cells to incident edges ($\cone(p)$, \cref{sec:dmplex_queries}) for a mesh with triangular cells has an arity of 3.

An example of indexing an array using a map is shown in \cref{fig:index_map}.
We declare some `outer' loop over an axis labelled $x$ (\ref{fig:index_map_code}, line~\ref{code:make_axis_x}) as well as an arity 3 map, $f$, mapping from $x$ to $a$ (line~\ref{code:make_map}).
The array is then indexed with the \emph{called map} $f(p)$.
The resultant axis tree (\ref{fig:index_map_data_layout}, right) has lost axis $a$ but gained a new axis in its place, labelled $c$, containing the 3 entries indicated in the map for a given $p$.

The treatment of the indexing information to retain view-like semantics remain exactly the same as before.
For the example in \cref{fig:index_map} the returned index information is:

\begin{center}
  \begin{tblr}{|[1pt]c|[1pt]c|[1pt]}
    \hline[1pt]
    \textbf{Target path} & \textbf{Target expressions} \\
    \hline[1pt]
    $\{c, b\} \to \{a, b\}$ & $\{i_a = f(L^p_x, i_c),\ i_b = i_b\}$ \\
    \hline[1pt]
  \end{tblr}
\end{center}

\noindent
Where $f(L^p_x, i_c)$ is used to represent the map; $L^p_x$ is the outer loop index over axis $x$ and $i_c$ is an index for the new `arity' axis $c$.

The final layout function is again accessible via straightforward substitution:

\begin{center}
  \begin{tblr}{|[1pt]c|[1pt]l|[1pt]l|[1pt]}
    \hline[1pt]
    \textbf{Path} & \SetCell{c}\textbf{Original layout} & \SetCell{c}\textbf{Substituted layout} \\
    \hline[1pt]
    $\{a,b\}$ & $\textnormal{offset}(i_a, i_b) = 3 i_a + i_b$ & $\textnormal{offset}(L^p_x,i_b, i_c) = 3 f(L^p_x, i_c) + i_b$ \\
    \hline[1pt]
  \end{tblr}
\end{center}

Note that in the example above we did not specify the form of the map function $f$.
This is intentional as the function could either be a lookup with an indirection map, like $f(L^p_x,i_c) \coloneq \pycode{[...][?$L^p_x$?,?$i_c$?]}$, or it could instead be some affine function of the form $f(L^p_x,i_c) \coloneq C \cdot L^p_a + i_c + D$ (for some constant $C$ and $D$).
By treating maps more generally, instead of eagerly committing to a particular representation, \pyop3 should be able to work without change for all cases.
This will be useful in cases where one approach is known to be better than another, such as when dealing with structured meshes where indirections are expressible as affine maps.

\subsection{Ragged maps}
\label{sec:indexing_ragged_maps}

\begin{figure}
  \centering

  \begin{subfigure}{.9\textwidth}
    \begin{pyalg2}
      # set up the axis tree and array
      dat = Dat(Axis(4, "a"))

      # create the "outer" loop and loop index
      axis_x = Axis(3, "x")
      p = axis_x.index()

      # create a ragged map mapping from "x" to "a"
      arity_dat = Dat(axis_x, data=[2, 0, 1])
      f = Map({"x": [MapComponent("a", arity=arity_dat, ...)]})?\label{code:arity_dat}?

      # index the array
      indexed_dat = dat[f(p)]
    \end{pyalg2}

    \caption{\pyop3 indexing code.}
    \label{fig:index_ragged_map_code}
  \end{subfigure}

  \vspace{1em}

  \begin{subfigure}{\textwidth}
    \centering
    \includegraphics{index_ragged_map_transform.pdf}
    \caption{The axis tree transformation.}
    \label{fig:index_ragged_map_transform}
  \end{subfigure}

  \caption{
    Example indexing an array with a ragged map.
    The resulting array has variable size depending on the value of the loop index.
  }
  \label{fig:index_ragged_map}
\end{figure}

It is not always the case that maps will have a constant value arity.
To use the same example from \cref{sec:ragged_axis_trees}, the map relating vertices in a mesh to their supported edges has variable arity because the size of this relation is different for each vertex.

To represent ragged maps a simple change in the code is required: one must pass a \pycode{Dat} as the \pycode{arity} argument to the map, instead of an integer.
This is demonstrated in \cref{fig:index_ragged_map_code} (line~\ref{code:arity_dat}).
As a result of using a non-constant arity, the size of the indexed array is no longer constant but instead depends on the value of the loop index.
This is shown in \cref{fig:index_map_data_layout} (right).

A discussion of layout functions and target paths/expressions is omitted here because it is identical to the example above.

\subsection{Map composition}
\label{sec:indexing_map_composition}

\begin{figure}
  \centering

  \begin{subfigure}{.9\textwidth}
    \begin{pyalg2}
      # set up the axis tree and array
      axes = AxisTree.from_iterable([Axis(5, "a"), Axis(3, "b")])
      dat = Dat(axes)

      # create the "outer" loop and loop index
      axis_x = Axis(10, "x")
      p = axis_x.index()

      # create map f, mapping "x" to "y"
      f = Map({"x": [MapComponent("y", arity=3, ...)]})

      # create map g, mapping "y" to "a"
      g = Map({"y": [MapComponent("a", arity=2, ...)]})

      # index the array
      indexed_dat = dat[g(f(p)), :]
    \end{pyalg2}

    \caption{\pyop3 indexing code.}
    \label{fig:index_map_composition_code}
  \end{subfigure}

  \vspace{1em}

  \begin{subfigure}{\textwidth}
    \centering
    \includegraphics{index_map_composition_transform.pdf}
    \caption{
      The axis tree transformation.
    }
    \label{fig:index_map_composition_transform}
  \end{subfigure}
  \caption{Indexing an axis tree using composed maps.}
  \label{fig:index_map_composition}
\end{figure}

As discussed in \cref{sec:dmplex_queries}, it is occasionally desirable to compose maps such that the output of one feeds directly into another.
This is natural to express in \pyop3.

As an example, consider the indexing operation shown in \cref{fig:index_map_composition}.
Axis $a$ is indexed with the composed map $g(f(p))$ where $p$ is a loop index over axis $x$, $f$ maps $x$ to $y$ and finally $g$ maps $y$ to $a$.
As the arities of $f$ and $g$ are 3 and 2 respectively, two axes with sizes 3 and 2 (labelled $c$ and $d$) are produced in the resultant indexed array (\ref{fig:index_map_composition_transform}, right).

Indexing the array with the composed map gives the following index information:

\begin{center}
  \begin{tblr}{|[1pt]c|[1pt]c|[1pt]}
    \hline[1pt]
    \textbf{Target path} & \textbf{Target expressions} \\
    \hline[1pt]
    $\{c, d, b\} \to \{a, b\}$ & $\{i_a = g(f(L^p_x, i_c), i_d),\ i_b = i_b\}$ \\
    \hline[1pt]
  \end{tblr}
\end{center}

\noindent
Where the index corresponding to the `arity' axis $c$ is passed as an argument to $f$ and $d$ passed to $g$.
With this, the resultant substituted layouts are then:

\begin{center}
  \begin{tblr}{|[1pt]c|[1pt]l|[1pt]l|[1pt]}
    \hline[1pt]
    \textbf{Path} & \SetCell{c}\textbf{Original layout} & \SetCell{c}\textbf{Substituted layout} \\
    \hline[1pt]
    $\{a, b\}$ & $\textnormal{offset}(i_a, i_b) = 3 i_a + i_b$ & $\textnormal{offset}(L^p_x,i_c, i_d, i_b) = 3 g(f(L^p_x, i_c), i_d) + i_b$ \\
    \hline[1pt]
  \end{tblr}
\end{center}

\section{Indexing with alternative data layouts}
\label{sec:indexing_data_layout_transformations}

\begin{figure}
  \centering
  \begin{subfigure}{.9\textwidth}
    \begin{pyalg2}
      # set up the (flipped) axis tree and array
      axes = AxisTree.from_iterable([Axis(3, "b"), Axis(5, "a")])
      dat = Dat(axes)

      # create the "outer" loop index
      axis_x = Axis(10, "x")
      p = axis_x.index()

      # create a map mapping from axis "x" to "a"
      f = Map({"x": [MapComponent("a", arity=3, ...)]})

      # index the array
      indexed_dat = dat[f(p), :]
    \end{pyalg2}

    \caption{\pyop3 indexing code.}
    \label{fig:index_map_swap_code}
  \end{subfigure}

  \vspace{1em}

  \begin{subfigure}{\textwidth}
    \centering
    \includegraphics{index_map_swap_transform.pdf}
    \caption{This axis tree transformation.}
    \label{fig:index_map_swap_transform}
  \end{subfigure}

  \caption{
    Indexing an array with a map.
    The transformation shown is identical to \cref{fig:index_map} except the axes of the input array have been swapped.
  }
\end{figure}

In \cref{sec:axis_tree_alternative_layouts} we introduced the concept of alternative data layouts and claimed that \pyop3 does not care about the specific ordering of axes.
Whilst this is true internally to \pyop3, more thought is required when interfacing with external code.
In particular, the local kernels evaluated during the mesh stencil calculation assume a specific ordering of the input and output DoFs.
In other words, the global data structures do not have a prescribed layout, but the local ones do.

Conveniently, \pyop3's indexing approach entirely obviates this issue.
From \cref{alg:index_axis_tree} it can be seen that the structure of the indexed axis tree \emph{comes entirely from the index tree, not the input axis tree}.
This means that we can prescribe the local DoF ordering independently of any global data layout.

To demonstrate this behaviour we consider an identical packing transformation to that shown in \cref{fig:index_map}, but with a different axis ordering for the unindexed array: axes $a$ and $b$ have been swapped.
All other code - including the indexing code - is unchanged.

By keeping the indexing code the same we therefore produce the same axis tree and index information as \cref{fig:index_map}:

\begin{center}
  \begin{tblr}{|[1pt]c|[1pt]c|[1pt]}
    \hline[1pt]
    \textbf{Target path} & \textbf{Target expressions} \\
    \hline[1pt]
    $\{c, b\} \to \{a, b\}$ & $\{i_a = f(L^p_x, i_c),\ i_b = i_b\}$ \\
    \hline[1pt]
  \end{tblr}
\end{center}

\noindent
But, since the layout functions for the unindexed object are different to before, the substituted layout is also necessarily different:

\begin{center}
  \begin{tblr}{|[1pt]c|[1pt]l|[1pt]l|[1pt]}
    \hline[1pt]
    \textbf{Path} & \SetCell{c}\textbf{Original layout} & \SetCell{c}\textbf{Substituted layout} \\
    \hline[1pt]
    $\{a, b\}$ & $\textnormal{offset}(i_a, i_b) = 5 i_b + i_a$ & $\textnormal{offset}(L^p_x,i_b, i_c) = 5 i_b + f(L^p_x, i_c)$ \\
    \hline[1pt]
  \end{tblr}
\end{center}

\noindent
This demonstrates that even though the DoFs in the local array are exactly the same as before, \emph{they have been extracted from totally different parts of the input array}.

\end{document}
