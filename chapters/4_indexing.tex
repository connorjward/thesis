\documentclass[thesis]{subfiles}

\begin{document}

% TODO: Somewhere talk about ragged temporaries?

\chapter{Indexing}

% motivation
Why do we want to index arrays? Why do we need a flexible system?

\section{Indexing axis trees}

% basic intro
Begin by giving an example of indexing a linear axis tree (10, 3)[::2, 1:] (and explain this terminology).
Explain/show that the axis tree needs to necessarily shrink but the expressions needed to index it go from a*3+b to (2*x)*3+(y+1).
Show the (relabelled) axis tree to demonstrate shrinkage.

Describe the process of indexing an array or axis tree: parse index tree, traverse index tree and generate new axes and collect index expressions, produce a new set of axes with the same layout as before but new axes and index expressions.
(note that both axis trees and arrays are indexable, since we can loop over slices of axes)
Demonstrate this using the same example as above.
Use the ASCII view of the trees to demonstrate?
Note that when we index we keep the layouts and index expressions separate, see index composition section for the reason.

% maybe slightly higher up
Explain why we need to use index *trees* here: to build a wider variety of axes.
If we want to build a new tree like set of axes (e.g. for closure packing) then we need to be able to pack different axis components together.

\section{Loops}

% needs to be discussed here because maps dont make sense otherwise, can talk about how
% we just wrap index exprs in loop index vars so they can be differentiated.

A loop index is constructed by doing axis[???].index().
This loop index can then be used inside an index tree like a slice or map.

A loop index has no shape. e.g. axes[p] where p is a loop index is scalar.

Perhaps provide a motivating example where we have array[p, :], or something, show that the layout expression contains p and hence we can substitute in an iname or index value.

\section{Maps}

Maps differ from slices because they add additional shape. They have a from index

How to build a map.

Give closure as an example. Index tree.

\subsection{Ragged maps}

Ragged maps are also supported. e.g. support, star

\subsection{Map composition}

e.g. g(f(p))

\section{Index composition}

i.e. indexing an indexed thing

Give an example indexing Axis(10, "a")[::2][1:].
The required subst layout targets [2, 4, 6, 8], which is 2*(i+1), not (2*i)+1

cannot directly insert as it breaks composition, need to store index_exprs separately

\end{document}
