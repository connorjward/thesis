\documentclass[thesis]{subfiles}

\begin{document}

\chapter{Conclusions}
\label{chapter:conclusions}

Over the course of this thesis we have introduced a novel suite of abstractions that together may be used to represent general-purpose mesh stencil calculations.
Compared with existing frameworks, the new abstractions are better able to represent the semantics of the problem domain, enabling a greater range of algorithms, transformations, and optimisations to be expressed and applied.
In addition to this theoretical work, the abstractions have been implemented as part of the new Python library \pyop3.
\pyop3 takes the abstractions and uses code generation, DSL + compiler, to achieve high performance without sacrificing expressiveness and flexibility  % FIXME:
Running in parallel in distributed computing environments is also supported (\cref{chapter:parallel).

The core abstractions introduced by this work are the axis tree (\cref{chapter:axis_trees}) and index tree (\cref{chapter:indexing}).
Axis trees... %that can flexibly describe data layouts and be indexed.
Index trees transform axis trees...
Together these...
% (compared to existing frameworks, wheremesh numbering or topology information are lost, bad composability, limited applicability
% and we wanted to support the multi-representation style of dmplex, where data can be viewed in multiple equivalent ways.

\section{Future research directions}
\label{sec:future_work}

% with only a small number of exceptions what we have shown so far is basically a bad pyop2
% the abstractions of pyop3 were created expressly to enable novel methods/future work by capturing more semantic information than existing frameworks

\subsubsection{Complete Firedrake integration}

% highest priority is a complete pyop3 obvs.!
% improve performance of python loops via generating code or cython
% key inspector-exec optimisations (expression compression)
% completely merge into Firedrake

% data layout transformations exposed in UFL

\subsubsection{GPU support}

% like vectorisation, loopy transforms
% PETSc support GPUs
% the groundwork is laid.

\subsubsection{Novel data structures}

% new matrix types, custom sparse array type, libXSMM (check spelling)
% good for small problems

% redevelop extruded mesh functionality - explain why dropped, maybe important for GPU performance with big packing things. - fewer scalar instructions
% structured meshes, partially structured (e.g. that multigrid framework), cubed-sphere

% dynamic data structures - maybe good for AMR or particle methods

\subsubsection{Expanded stencil algorithms}

% patchPC
% WENO slope limiters?
% finite volume?

% reimplement SLATE and pcpatch using pyop3.

\subsubsection{Runtime transformations}

% Hanging nodes: AMR? hp-adaptivity
% orientations

\subsubsection{Compiler bits}

% sparse tiling
% loop fusion and subexpression elimination over cells and facets

\subsubsection{Better parallel}

% Minimal DG? cell loops for DG do not need to exchange halo data even though its there

\end{document}
