\documentclass[thesis]{subfiles}

\begin{document}

% PERFORMANCE OPTIMISATIONS?

% e.g. can do things like \cite{dasSlicingAnalysisIndirect1994} and "flatten" repeated indexing

\chapter{Summary}
\label{chapter:summary}

%pyop3 is like Ebb and Simit, but the data model is chosen to truly unify PETSc and N-dim arrays. We get maths-y optimisations that the others don't have, like mesh numbering and storing edge and vertex data together and mixed-type things. (plus obvs distributed parallel)
%
%It solves the same problem but from a different direction.
%
%Primary objectives: integration with PETSc, distributed parallel - data model that is compatible with DMPlex etc.
% we don't need our own mesh implementation - so distribution, I/O etc we get "for free"
%
%Simit is also very similar, but not distributed parallel and pyop3 unifies the hypergraph and linear algebra data views.
%
% and Liszt? Just quite old, reimplements a lot

\section{Future work}
\label{sec:future_work}

% highest priority is a complete pyop3 obvs.!

% compression transformation
% sparse data structures - already done a proof of concept, provide index array, we are already CSR so this is straightforward to do
% GPUs, vectorisation (\cref{sec:codegen_loopy_kernel})
% Hanging nodes: AMR? hp-adaptivity
% orientations?
% WENO slope limiters?
% finite volume?
% patchPC

\section{Conclusions}

\end{document}
