\documentclass[thesis]{subfiles}

\begin{document}

\chapter{Introduction}
\label{chapter:introduction}

% * (numerically) solving PDEs is very important for lots of things:
%   * climate
%   * manufacturing
%   * fluids...
%   * ???

\section{A motivating example: solving the Stokes equations using the finite element method}
\label{sec:stokes_equations}

As an introductory example to a calculation requiring iterating over a mesh, we consider solving the Stokes equations using the \gls{fem}.
Our exposition will focus on the aspects of the computation that are relevant for \pyop3, for a more complete review of \gls{fem} we refer the reader to~\cite{brennerMathematicalTheoryFinite2008} and~\cite{larsonFiniteElementMethod2013}.

The Stokes equations are a linearisation of the Navier-Stokes equations and are used to describe fluid flow for laminar (slow and calm) media.
For domain $\Omega$ and boundary $\Gamma$, omitting any forcing terms for simplicity, they are given by

\begin{subequations}
  \begin{align}
    - \nu \Delta u + \nabla p &= 0 \quad \textrm{in} \ \Omega, \\
    %
    \nabla \cdot u &= 0 \quad \textrm{in} \ \Omega, \\
    %
    u &= g \quad \textrm{on} \ \Gamma.
  \end{align}
  %
  \label{eq:strong_stokes}
\end{subequations}

with $u$ the fluid velocity, $p$ the pressure and $\nu$ the viscosity.
We also prescribe Dirichlet boundary conditions for the velocity across the entire boundary, setting $u$ to the value of function $g$.
Since we have a coupled system of two variables ($u$ and $p$), we refer to the Stokes system as being a \textit{mixed} problem.

\subsection{Deriving a weak formulation}

% TODO: Add a forcing term $f$ so the RHS is non-zero so we can happily show the assembly diagram

For the finite element method we seek the solution to the \textit{variational}, or \textit{weak}, formulation of these equations.
These are obtained by multiplying each equation by a suitable \textit{test function} and integrating over the domain.
For \cref{eq:strong_stokes}, using $v$ and $q$ as the test functions, drawn from function spaces $\hat V$ and $Q$ respectively, and integrating by parts this gives

% taken from 
% https://nbviewer.org/github/firedrakeproject/firedrake/blob/master/docs/notebooks/06-pde-constrained-optimisation.ipynb
% and Larson and Bengzon (pg. 293)
\begin{subequations}
  \begin{align}
    \int \nu \nabla u : \nabla v \, \textrm{d}\Omega
    - \int p \nabla \cdot v \, \textrm{d}\Omega
    - \int \nu (\nabla u \cdot n) \cdot v \, \textrm{d}\Gamma
    - \int p n \cdot v \, \textrm{d}\Gamma
    &= 0
    &\forall \ v \in \hat V
    \label{eq:weak_stokes_extra_V} \\
    %
    \int q \, \nabla \cdot u \, \textrm{d}\Omega
    &= 0
    &\forall \ q \in Q.
    \label{eq:weak_stokes_extra_Q}
  \end{align}
  \label{eq:weak_stokes_extra}
\end{subequations}

From these weak forms it is now possible to classify the function spaces for $u$ and $p$.
For $u$, we already know that the space must be vector-valued, since it stores a velocity, and constrained to $g$ on the boundary.
\Cref{eq:weak_stokes_extra_V} further shows us that $u$ must have at least one weak derivative.
We can therefore say that $u \in V$ where

\begin{equation}
  V = \{ \ v \in [H^1(\Omega)]^d : v |_{\Gamma} = g \ \}  \\
  \label{eq:stokes_velocity_space}
\end{equation}

$p$ is scalar-valued, no derivatives of $p$ are present in the weak formulation, nor are any boundary conditions applied to it and so we can write that $p \in Q$ where

\begin{equation}
  Q = \{ \ q \in L^2(\Omega) \ \}  \\
  \label{eq:stokes_pressure_space}
\end{equation}

Since the values of $u$ at the boundary are already prescribed, the function space of the test function $v$ is defined to be zero at those nodes

\begin{equation*}
  \hat V = \{\ v \in [H^1(\Omega)]^d : v|_{\Gamma} = 0 \ \}.
\end{equation*}

This allows us to drop some terms from \cref{eq:weak_stokes_extra_V}, allowing us to state the final problem as follows:

\vspace{1em}

%TODO: Not sure how to format this best
Find $(u, p) \in V \times Q$ such that

\begin{subequations}
  \begin{align}
    \int \nu \nabla u : \nabla v \, \textrm{d}\Omega
    - \int p \nabla \cdot v \, \textrm{d}\Omega
    &= 0
    &\forall \ v \in \hat V
    \label{eq:weak_stokes_V} \\
    %
    \int q \, \nabla \cdot u \, \textrm{d}\Omega
    &= 0
    &\forall \ q \in Q.
    \label{eq:weak_stokes_Q}
  \end{align}
  \label{eq:weak_stokes}
\end{subequations}

\subsection{Discretising the system of equations}

In order to solve this weak formulation using the finite element method we discretise the function spaces in use by replacing them with a finite dimensional equivalent:

\begin{equation*}
  V \to V_h \subset V,
  \quad
  \hat V \to \hat V_h \subset \hat V,
  \quad
  Q \to Q_h \subset Q.
\end{equation*}

Each of these discrete spaces is spanned by a set of basis functions so any function can be expressed as a linear combination of the basis functions and their coefficients.
For example, we can write the function $u_h \in V_h$ as

\begin{equation*}
  u_h = \Sigma^N_{i=1} \hat u_i \psi^{V_h}_i
\end{equation*}

for basis functions $\psi^{V_h}_i$ and coefficients $\hat u_i$.

Substituting these discrete function spaces back into \cref{eq:weak_stokes}, and discarding the basis coefficients for the arbitrary functions $v_h$ and $q_h$, we obtain the discrete problem:

\vspace{1em}

%TODO: Not sure how to format this best
Find $(\hat u, \hat p)$ such that

\begin{subequations}
  \begin{align}
    \int \nu \hat u \nabla \psi^{V_h} : \nabla \psi^{\hat V_h} \, \textrm{d}\Omega
    - \int \hat p \psi^Q \nabla \cdot \psi^{\hat V_h} \, \textrm{d}\Omega
    &= 0
    &\forall \ \psi^{\hat V} \\
    %
    \int \psi^Q \, \nabla \cdot \hat u \psi^{V_h} \, \textrm{d}\Omega
    &= 0
    &\forall \ \psi^{Q}
  \end{align}
  \label{eq:weak_stokes_discrete}
\end{subequations}

This can be reformulated as the saddle point linear system

\begin{equation}
  \left (
  \begin{array}{c|c}
    \int \nu \nabla \psi^{V_h} : \nabla \psi^{\hat V_h} \, \textrm{d}\Omega
    &
    - \int \psi^Q \nabla \cdot \psi^{\hat V_h} \, \textrm{d}\Omega \\
    \hline
    \int \psi^Q \, \nabla \cdot \psi^{V_h} \, \textrm{d}\Omega
    &
    0
  \end{array}
  \right )
  \left (
  \begin{array}{c}
    \hat u \\
    \hline
    \hat p
  \end{array}
  \right )
  =
  \left (
  \begin{array}{c}
    0 \\ \hline 0
  \end{array}
  \right )
  %
  \label{eq:stokes_linear_system}
\end{equation}

Solving the Stokes equations using the finite element method therefore boils down to constructing, or \textit{assembling}, the left-hand-side matrix and the, here trivial, right-hand-side vector before solving for the coefficients $\hat u$ and $\hat p$.

\subsection{Choosing a basis}

\begin{figure}
  \centering
  \begin{subfigure}{.3\textwidth}
    \centering
    \includegraphics{lagrange_element_2.pdf}
    \caption{P2, TODO}
    \label{fig:lagrange_element_2}
  \end{subfigure}
  \begin{subfigure}{.68\textwidth}
    \centering
    \begin{tabular}{c c c}
      \includegraphics[width=.3\textwidth]{element_Lagrange_2_dof0.pdf}
      &
      \includegraphics[width=.3\textwidth]{element_Lagrange_2_dof1.pdf}
      &
      \includegraphics[width=.3\textwidth]{element_Lagrange_2_dof2.pdf}
      \\
      \includegraphics[width=.3\textwidth]{element_Lagrange_2_dof3.pdf}
      &
      \includegraphics[width=.3\textwidth]{element_Lagrange_2_dof4.pdf}
      &
      \includegraphics[width=.3\textwidth]{element_Lagrange_2_dof5.pdf}
    \end{tabular}
    %
    \caption{P2 basis functions, TODO}
    \label{fig:lagrange_element_2_basis}
  \end{subfigure}
  %
  \caption{The $P_2$ (Lagrange, degree 2) finite element, \cite{defelement}}
\end{figure}

% TODO: Resolve \hatV_h vs V_h, think one goes away...
In order to numerically evaluate the integrals in \cref{eq:stokes_linear_system} the basis functions $\psi^{V_h}$ and $\psi^Q$ must be known.
In the finite element method these are discovered from the choice of \textit{finite element}.

First formalised by Ciarlet~\parencite{ciarletElement2002}, a finite element is the triple $(K, P, N)$, where:

\begin{itemize}
  \item $K$ is a bounded closed set, or \textit{cell}, with non-empty interior and piecewise smooth boundary,
  \item $P$ is a finite-dimensional space of functions on $K$, and
  \item $N$ is a set of linear functionals that form a basis for the dual space of $P$.
\end{itemize}

% TODO: Reword this because the point evaluation stuff is a bit confusingly worded
A simple example of a finite element, the degree 2 Lagrange element, is shown in \cref{fig:lagrange_element_2}.
For this element $K$ (the cell) is a triangle, $P$ (the function space) is the space of order 2 polynomials, and $N$ (the dual basis) is defined to be point evaluation at each of the nodes:

\begin{equation*}
  l_i(v) = v(x_i),
\end{equation*}

where $l_i$ is the linear functional associated with node $i$, $v$ is some function in $P$ and $x_i$ are the coordinates of the $i$-th node.

From these attributes, it is possible to determine a basis for $P$ by imposing that

% TODO: define n_k
\begin{equation*}
  l_i(\psi_j) = \delta_{ij} \quad i, j = 0, 1, \dots, n_k.
\end{equation*}

In the case of the $P_2$ element this yields the basis functions

% see https://defelement.com/elements/examples/triangle-lagrange-equispaced-2.html
\begin{align*}
  &\psi_0 = 2x^2 + 4xy - 3x + 2y^2-3y+1,
  &
  &\psi_3 = 4xy, \\
  %
  &\psi_1 = x(2x-1),
  &
  &\psi_4 = 4y(-x-y+1), \\
  %
  &\psi_2 = y(2y-1),
  &
  &\psi_5 = 4x(-x-y+1),
\end{align*}

shown in \cref{fig:lagrange_element_2_basis}.

As described, these basis functions belong to just the cell $K$.
However, this is just the \textit{reference} cell.
In the finite element method the problem domain $\Omega$ is broken apart into many disjoint cells - $\Omega = \Sigma_K \Omega_K$ - and the integrals are evaluated per cell.
This tesselation of the domain is termed a \textit{mesh}.

From \cref{fig:lagrange_element_2} one can see that the linear functionals, and hence the basis functions, are associated with different topological entities in the cell and this is emphasised by showing them in different colours.
Inter-cell continuity is achieved by having \glspl{dof} on the vertices and edges, as these are shared between adjacent cells.
If all \glspl{dof} are associated to the cell alone then the function space is said to be \textit{discontinuous}.

\begin{figure}
  \centering
  %
  \hfill
  %
  \begin{subfigure}{.4\textwidth}
    \includegraphics{lagrange_element_3_vec.pdf}
    \label{fig:scott_vogelius_element_P3}
  \end{subfigure}
  %
  \hfill
  %
  \begin{subfigure}{.4\textwidth}
    \includegraphics{lagrange_element_2_dg.pdf}
  \end{subfigure}
  %
  \hfill
  %
  \caption{Scott-Vogelius element (degree 3?), $[P_3]^2 \oplus P_2^\mathrm{disc}$}
  \label{fig:scott_vogelius_element}
\end{figure}

The choice of basis functions used by the function spaces has significant implications for the convergence and stability of the model.
For the Stokes equations in 2D, a common choice of element pair, or \textit{mixed} element, with properties matching the constraints given in \cref{eq:stokes_velocity_space} and \cref{eq:stokes_pressure_space} is the Scott-Vogelius element~\cite{scottNormEstimatesMaximal1985}.
Shown in \cref{fig:scott_vogelius_element}, the element consists of a continuous vector-valued degree $k$ Lagrange element for the velocity space, and a discontinuous Lagrange element of degree $k-1$.
Note that the Scott-Vogelius element is known to be inf-sup stable for degree $\geq 4$ but we only show degree 3 here for brevity~\cite{guzmanScottVogeliusFiniteElements2018}.

\subsection{Finite element assembly}

If we consider assembly the top left block of the matrix in \cref{eq:stokes_linear_system} we see that we need to evaluate

\begin{equation*}
  \int \nu \nabla \psi^{V_h}_i : \nabla \psi^{\hat V_h}_j \, \textrm{d}\Omega
\end{equation*}

for each possible $i$ and $j$.

However, from \cref{fig:lagrange_element_2_basis} it can be seen that finite element basis functions are only non-zero in a small region around its assigned mesh entity - we say that they have \textit{local support}.
This is significant for constructing matrices such as that in \cref{eq:stokes_linear_system} because most of the integrals are zero \textit{by definition}.
This has two consequences:

\begin{itemize}
  \item
    The assembled matrix is \textit{sparse}.

  \item 
    Assembly is more efficient done cell-wise. That is, one evaluates

    \begin{equation*}
      \int_K \nu \nabla \psi^{V_h}_K : \nabla \psi^{\hat V_h}_K \, \textrm{d}\Omega_K
    \end{equation*}

    for each cell $K$ and adds the results together.
\end{itemize}

% TODO: Decide on a consistent algorithm format
% TODO: I should introduce the term "stencil"
\begin{algorithm}
  \begin{verbatim}
    FOR EACH cell IN mesh.cells:
      FOR EACH coefficient IN expression:
        collect the coefficients of basis functions that have non-zero support over cell
      compute the integral numerically
      add the values of the computed integrals into the global matrix or vector
  \end{verbatim}
  \caption{TODO}
  \label{alg:fem_assembly}
\end{algorithm}

\begin{figure}
  \centering
  \includegraphics[scale=1.2]{fem_assembly.pdf}
  \caption{TODO}
  \label{fig:fem_assembly}
\end{figure}

The algorithm for this cell-wise assembly process is summarised in \cref{alg:fem_assembly}.
Given some code that evaluates the integral for a single cell the global algorithm is responsible for marshalling ``local" data in and out of it and hence we term it a \textit{stencil} algorithm.
An example is shown in \cref{fig:fem_assembly}.
Local viscosity data for a particular cell is \textit{packed} into a small temporary and the cell integral is evaluated.
This produces a small dense matrix whose values must be scattered to entries in the global sparse matrix.

\subsection{Finite element data structures}

\begin{figure}
  \centering
  \includegraphics{scott_vogelius_element_dof_layout_packed.pdf}
  \caption{TODO}
  \label{fig:scott_vogelius_element_dof_layout_packed}
\end{figure}

For the finite problem in \cref{eq:stokes_linear_system} one must store various arrays of unknowns in memory ($\nu$, $(\hat u, \hat p)^T$, etc).
The vector $(\hat u, \hat p)^T$, for a 2-dimensional domain, contains

\begin{equation*}
  2 ( n_\textnormal{cells} \times 1 + n_\textnormal{edges} \times 2 + n_\textnormal{vertices} \times 1 ) + n_\textnormal{cells} \times 6
\end{equation*}

unknowns, which are packed into a local temporary of

\begin{equation*}
  2 ( 1 \times 1 + 3 \times 2 + 3 \times 1 ) + 6 = 26
\end{equation*}

numbers (per row and column) during assembly.
This process is illustrated in \cref{fig:scott_vogelius_element_dof_layout_packed}.

In this example it can be seen that both the global and local arrays have some non-trivial structure.
Representing this structure is one of the main challenges of this thesis and will be described in more detail in \cref{chapter:axis_trees,chapter:indexing}.

\section{Execution models for mesh stencil calculations}

% TODO: start with "so this is the algorithm we want to execute, we want to automate its
% application"
% this crops up with subtle variations all over the place, changes to multitude of variables constitute a rewrite - also cannot layer UFL on top?
% BECAUSE PRODUCTIVITY!

% TODO:: cite dune, deal.ii, Firedrake, FENIcs, devito...?
At this point we have established, excluding the local kernels and global solve, the algorithms and data structures necessary to solve a finite element problem.
Subsequently, we are interested in how to manifest these in software.
Writing these codes by hand is prohibitively difficult: writing a performant and scalable simulation would take months or years of programmer effort and any changes to the \glspl{pde}, discretisation or hardware might constitute a substantial rewrite.
To counter this, numerous frameworks exist providing the building blocks from which a domain specialist, without expertise in high performance computing nor months of programmer time, might build a simulation.
This creates a separation of concerns between the framework maintainers, who specialise in low-level optimisation, and the users, who can instead reason about the problem in terms of the mathematics.

% talk about BLIS, Spiral, FFTW, Halide etc?
In addition to the step-change in programmer productivity, high-level abstractions also facilitate advanced performance optimisations that would be very difficult to implement for a low-level code.
Sometimes, high-level algorithmic changes (discretisation, solver, etc) are required to achieve acceptable performance on a given machine and having a high-level of abstraction means that tweaking these options is minimally invasive \parencite{betteridgeCodeGenerationProductive2021}.
Further, having a high-level representation of the problem enables optimisations best expressed at the level of the mathematics that would otherwise be very challenging to implement (e.g.~\cite{homolyaExposingExploitingStructure2017}).

% inspector-executor model?
% inspector-executor model. cite Saltz and Strout
% two programs, an inspector that generates a schedule, and an executor that uses it. Executor
% is a transformed original program.
% these aim to improve data locality and parallelisation opportunities.
% important point is that I/E strategies utilise runtime information to generate optimal schedules
% this is very important for unstructured applications where the compilers would have a really hard time!
\cite{stroutSparsePolyhedralFramework2018} % review article
\cite{mirchandaneyPrinciplesRuntimeSupport1988} % old (general purpose) example
\cite{arenazInspectorExecutorAlgorithmIrregular2004} % fem example but specifically parallelisation
% perhaps also cite Luporini for sparse tiling? yes I think that would be good.
% Interesting note: composing inspector-executor transformations is difficult.
% see "The Sparse Polyhedral Framework: Composing Compiler-Generated Inspector-Executor Code"
% DSLs like loopy and pyop3 can make this easier to handle.
% mesh numbering is an example of an inspector-executor thing.
% so is determining core and owned to overlap communication and computation
% DSLs help a lot to implement this sort of thing because transformations can be a lot easier to
% express using a high-level representation.

\subsection{Design requirements}

% NOTE: This needs a rewrite, key idea is to enumerate interface choices that impact productivity
% Mention expressivity power?
For the unstructured mesh traversal operation we are interested in, we need an abstraction that:

\begin{itemize}
  \item
    Expresses operations in terms of loops, compact kernels and restricted data structures,
  \item
    Supports the indirection mappings necessary for unstructured meshes (e.g. the map from cells to supported DoFs), and
\end{itemize}


% we want things to be both performant and expressive - here we care about performance really

% assembly can be a bottleneck (when?)
% can be slow because of: data movement, payload or matrix insertion.

\subsubsection{Distributed memory parallelism}
\label{sec:intro_mpi}

% large scale simulations need to be run on massively parallel machines, necessitates dist. memory (MPI)

In massively parallel simulations the data structures are often too large to be stored in the memory of any individual node.
Instead, they are \textit{distributed} between all of the processes, with each process owning, and seeing, only a small piece of the entire structure.

In order to run simulations at scale global data structures are broken apart and each process stores and operates on its own local piece.

% TODO: do not fade some out - just the shared ones.
% NOTE: closure is novel concept...
\begin{figure}
  \centering
  \includegraphics{split_mesh.pdf}
  \caption{
    \pyop2 entity classes for a mesh distributed between 2 processes.
    Points belonging to process 1 (left) are shown in red and points belonging to process 2 (right) are shown in blue.
    \textit{Ghost} points are indicated by points whose colour match the other process, \textit{owned} points are faded and the rest are labelled \textit{core}.
    We assume that one is only computing cell-wise integrals on the mesh and so the mesh overlap need only contain enough ghost points to ensure that all the cells have a complete closure.
  }
  \label{fig:pyop2_split_mesh}
\end{figure}

% TODO: I think it is better to not bother talking about "closed" meshes here. Then can move
% some things back to DMPlex section. Hmm but overlap is important for PyOP2 exposition!
% Can phrase as "local computations may require adjacent DoFs that "belong" on other processes - these are called "ghost" points and care must be taken to ensure that the values are kept up-to-date across processes.

For unstructured meshes and vectors of mesh data it is usual to \textit{partition} the mesh.
Since one needs all DoFs incident on the cell to compute things (need a closed local mesh) boundary values may be duplicated on adjacent processes (\cref{fig:star_forest}).
These are termed \textit{ghost} points.
The amount of overlap required depends on both the stencil and also the amount of redundant computation/frequency of transfers (look into this, Fabio's thesis, Devito?).

The global matrix is also partitioned, usually by row, so each process only sees a portion of it

% show the mesh figure here? not in chapter 2?
It is often the case when solving PDEs that unknowns must be stored on multiple processes simultaneously.
Consider the partitioned mesh shown in \cref{fig:???}.
In order to be able to compute the cell-wise integrals for the finite element method, the cell stencils overlap with each other and the shared overlap region must be stored on both processes.
Since unknowns must have only one owning process, their representation on other ranks are known as ``ghosts".


% Now how to get good performance? reduce amount of communication + overlap computation and communication (non-blocking), avoid expensive comm patterns (like all to one rank)

\subsubsection{Performance portability}

% ie GPUs - they are now in the biggest supercomputers, good energy efficiency, bandwidth, FLOP/s

% potentially very effective for this sort of work - very parallel

% challenging programming model, different programming languages (and vendor specific) (apart from Kokkos etc), kernels need to be launched (can be slow, latency), memory transers between host and device

% OpenMP? hybrid?
% As with Firedrake and PETSc, MPI is chosen as the sole parallel abstraction; hybrid models also using shared memory libraries like OpenMP (cite) are not used because the posited performance advantages are contentious \parencite{knepleyExascaleComputingThreads2015} and would increase the complexity of the code.

\subsubsection{Interoperability with existing software}

% PETSc is used in MFEM, FEniCS, MOOSE, deal.ii? so many, definitely a "standard"

% high level algorithmic changes are essential for performance (JB paper) - want to use top quality solvers

% also, need to reimplement a lot of things best done by existing packages (partitioning, I/O etc)

% certainly hard to do in parallel, sparse matrices etc
% PETSc is great, name check

\subsubsection{Data locality optimisations}
\label{sec:intro_mesh_numbering}

% mesh numbering
% --------------
% cite PyOP2 paper, also perhaps extruded paper

% Numbering can reference the FEM pseudo-code - we need cellwise data to be "close"
% "put closure data close together"

% \subsubsection{Data layout transformations}

\begin{figure}
  \centering
  \includegraphics{scott_vogelius_element_dof_layout_swap.pdf}
  \caption{
    TODO
  }
  \label{fig:scott_vogelius_element_dof_layout_swap}
\end{figure}

% many packages "commit early" to a particular representation, potentially sacrificing performance

% TODO: cite AoS/SoA performance paper or something.

% NOTE: do not talk about OP2/PyOP2 here, mentioned first below

% Lastly, OP2 and \pyop2 both commit eagerly to storing the data in a particular format.
% Depending on the problem at hand, better performance may be achieved by rearranging the way in which DoFs are stored such that data is streamed from memory in the most efficient manner possible.
% An example demonstrating a different data layout for the data shown in \cref{fig:scott_vogelius_element_dof_layout} is shown in \cref{fig:scott_vogelius_element_dof_layout_swap}.
% In the latter the vector component part of the data layout has been ``lifted" such that, for the $V_h$ function space, all \glspl{dof} corresponding with the 0th component are stored before all \glspl{dof} for the 1st component.
% This is equivalent to transforming from an ``array-of-structs" (AoS) layout to a ``struct-of-arrays" (SoA) layout.
% Neither layout is inherently faster than the other and different applications may want to use one over another.
% Both OP2 and \pyop2 only support storing data in the AoS format (\cref{fig:scott_vogelius_element_dof_layout_pyop2}) and so cannot do this.

% TODO: mention taichi in this chapter somewhere as a similar project, maybe also TACO (and SPF?)

% % AoS-SoA etc? (taichi)
% \textbf{Taichi} is a programming language embedded in C\+\+ for operating on complex data structures~\cite{huTaichiLanguageHighperformance2019}.
% Just like \pyop3, Taichi declares data structures hierarchically and the data layout is kept distinct from the operations applied to them.
% Taichi has no concept of a mesh, and it does not work on distributed machines.
% https://www.taichi-lang.org/
% very successful!

% TODO: This should actually be covered in the indexing chapter.
% \section{Renumbering for data locality}
% \label{sec:renumbering}
%
% For memory-bound codes, performance is synonymous with data locality.
% In the case of stencil codes like finite element assembly, one should aim to arrange the data such that the data required for a single stencil calculation is contiguous in memory and can be read from memory into cache with only a single instruction.
%
% For simulations involving unstructured meshes, data reorderings that provide perfect streaming access to memory are not possible and so renumbering strategies have been developed to try and maximise locality.
% For example, the data layout shown in \cref{fig:two_cell_mesh_lagrange_data_layout_flat} approximates the strategy taken by \pyop2, cells are traversed according to some RCM ordering and the cell closures are packed next to the cell~\cite{langeEfficientMeshManagement2016}.
% The is effective for finite element codes because finite element assembly (usually) involves iterating over cells and accessing the data in their closures.
%
% In \pyop3, we choose a simple approach and defer to PETSc to provide us with an appropriate RCM numbering for the points.
% This is communicated to the axis tree by giving an axis, in this case the \pycode{"mesh"} one, a \pycode{numbering} argument.
% This numbering consists of the flat indices of the axis and is exactly the object given to us from PETSc.
% This is not quite the case in parallel (see \cref{chapter:parallel}).

\subsection{Existing software}

% NOTE: A checklist comparing different packages was considered but I think it
% shouldn't be included because there are too many unknowns. I don't know enough
% about Liszt, Simit or Ebb and it makes things a bit more complicated to explain.

A number of packages exist that already meet most or all of these requirements:

\paragraph{Liszt}{
  is a \gls{dsl} embedded in Scala~\cite{devitoLisztDomainSpecific2011}.
  Mesh connectivity is expressed through built-in topological relations and mesh data is associated with specific topological entities.
  Liszt uses a custom mesh implementation with support for parallel partitioning and hence works in a distributed memory environment.
  Liszt is also capable of generating code for use in a multi-threaded or GPU context.
}

\paragraph{Simit}{
  is another \gls{dsl} for mesh simulations~\cite{kjolstadSimitLanguagePhysical2016}.
  It has a unique design where mesh data structures have a dual representation: they can either be viewed as a hypergraph or as a multi-dimensional tensor.
  This enables for both mesh-like queries to be applied to the data structures as well as enabling linear algebra operations to be expressed.
  Simit is capable of targeting both CPUs and GPUs without needing to change the input code, though distributed memory computing is not available.
}

\paragraph{Ebb}{
  is another \gls{dsl} embedded in Lua~\cite{bernsteinEbbDSLPhysical2016}.
  It uses a relational database model to describe the mesh and has a 3-layer infrastructure that separates simulation code from data structure specification and different code generation targets.
  It has support for execution on GPUs but distributed memory computing is not available.
}

% TODO: mention for these two that they are less expressive than the others.
% TODO: How to write "C++"?
\paragraph{OP2}{
  Unlike the frameworks mentioned above, OP2 is an \textit{active library} that provides source-to-source translation from C, C++ or Fortran to target a range of different backends including OpenMP, CUDA and OpenCL~\cite{mudaligeOP2ActiveLibrary2012}.
  OP2 uses a simplified model of a mesh where entities are represented as \textit{sets}.
  One can store data on these sets and mappings exist between sets.
  Computations are termed \textit{kernels} and are provided by the user.
  Distributed memory computing is possible and OP2 is even able to interleave computation and communication to provide improved scaling.
}

\paragraph{\pyop2}{
  is a reimplementation of OP2 in Python~\cite{rathgeberPyOP2HighLevelFramework2012}.
  The same core abstraction of sets, mappings and kernels is used but runtime code generation is used instead of source-to-source translation.
  \pyop2 currently only targets execution on CPUs though a proof-of-concept GPU extension has been created~\cite{fenics2021-kulkarni}.
  Distributed memory computing is supported and \pyop2 can also assemble sparse matrices with PETSc~\cite{petsc-web-page,petsc-user-ref,petsc-efficient}.
}
% Not flexible:
  % "Whilst appropriate for a great many operations, there are occasions where one needs to be able to execute multiple kernels or have nested loops - for example when developing certain types of preconditioners (e.g. \cite{gibsonSlateExtendingFiredrake2020}, \cite{farrellPCPATCHSoftwareTopological2021}).
  % "In these cases one has to extend the compiler in a sui-generis manner to achieve the desired result, resulting in code that is harder to maintain and not composable with other features.

\subsection{Data layout limitations}

\begin{figure}
  \centering
  \includegraphics{scott_vogelius_element_dof_layout_pyop2.pdf}
  \caption{
    The data layout matching \cref{fig:scott_vogelius_element_dof_layout} as it would be stored by \pyop2.
    Data for each function space ($V_h$ and $Q_h$) are stored in separate arrays.
  }
  \label{fig:scott_vogelius_element_dof_layout_pyop2}
\end{figure}

\begin{figure}
  \centering
  \includegraphics{scott_vogelius_element_dof_layout_topological.pdf}
  \caption{
    The closest possible data layout for \cref{fig:scott_vogelius_element_dof_layout} for a library that associates unknowns with topological entities.
    Data for each topological entity are stored in separate arrays.
  }
  \label{fig:scott_vogelius_element_dof_layout_topological}
\end{figure}

% * Liszt, Simit and Ebb are very flexible but they all tie data structures to topological entities. Data layouts
%   like SV element one are not expressible as it requires interleaving/mesh numbering is not possible.
%   * also unnatural to treat function spaces like this.
% * old:
  % "All of the frameworks described above have had to make a tradeoff between expressivity and flexibility (?).
  % Liszt, Simit and Ebb are considerably more flexible than either of OP2 or \pyop2 because they can execute any program written in their respective \glspl{dsl}."

% * OP2/PyOP2 (and Simit and Ebb?) use a set model that does not have this restriction (i.e. node sets) (ref figure)
%   * however this makes the code hard to reason about as info is getting thrown away, also makes things hard to compose.
% * old:
% Further, the treatment of data as belonging to sets can lead to information being discarded.
% Considering the data layout shown in \cref{fig:scott_vogelius_element_dof_layout}, \pyop2 would store it as a tuple of two \textit{node sets} (\cref{fig:scott_vogelius_element_dof_layout_pyop2}).
% Compared with the ``lossless" representation in \cref{fig:scott_vogelius_element_dof_layout}, this approach simplifies the implementation because the data can now be stored as 2 rectangular arrays with shapes \pycode{(10, 2)} and \pycode{(6, 1)}.
% However, this approach \textit{discards topological information}: the ``Mesh" layer of the representation is gone.
% This means that the library user has to do the bookkeeping to correctly handle the mappings from mesh entities to nodes.

\subsection{The missing abstraction}

% This mostly belongs in chapter 3
Clearly, there is something missing here.
The designs of the existing libraries all require that one either use topological information in a simplified way - associating data with particular mesh entities only - or that one take ownership of the data, discarding topological information that is helpful for having a composable abstraction.
To get around this difficulty we have developed a new abstraction for data layouts, termed \textit{axis trees}, that bridges the gap between these worlds.
Axis trees allow the user to describe complex data layouts of the sort shown in \cref{fig:scott_vogelius_element_dof_layout} fully, without needing to discard any of the topological information.
As a convenient side benefit, expressing data layout transformations (\cref{sec:data_layout_transformations}) becomes natural to do.

The axis tree abstraction is included in the new Python library \pyop3.
\pyop3 is a near-total rewrite of \pyop2 that aims to substantially improve its expressivity power and composability.
It has support for distributed memory parallelism and integrates with PETSc.

% NOTE: I don't like including this but I think I have to be honest about this.
% put in a box
Disclaimer:
Whilst all of the functionality of \pyop3 in the following thesis has been tested and shown to work, the time constraints of the PhD mean that \pyop3 is unfinished.
Thus not all of the provided functionality has been merged into the \texttt{main} branch yet.
This is considered very high priority future work for the project.

\section{Thesis outline}

The remainder of this work is structured as follows\dots

\end{document}
