\documentclass[thesis]{subfiles}

\begin{document}

\chapter{Introduction}
\label{chapter:introduction}

% where does this go>?
A common criticism of domain specific languages is that they are superfluous since any code they generate could have been written in C, C++ or Fortran by hand instead.
This is especially true for \pyop3 since it generates a simple subset of C code focussed on reading/writing from array-like data structures.
This is the sort of code that is bread-and-butter for undergraduate courses on high-performance computing; classical optimisations like loop tiling and AoS-SoA are very simple to write by hand.

The counter point, of course, is that \pyop3 is not primarily intended to be used as a simulation front-end, but instead as an intermediate representation of some higher level abstraction.
DSLs can more or less only express problems that are also expressible in all of their IRs.
The only exception to this is if the abstractions provide an escape hatch where one can inject custom code to perform a particular task.
Escape hatches, however, are difficult to produce and ``hacky".
Also, the cost of implementing one can be prohibitive.
It would be preferable for problems to be fully expressible in all IRs.

The key contribution of \pyop3 is that it increases the number of representable states of a finite-element-like code (whilst also cutting down on boilerplate).
This increased expressiveness is beneficial as it enables higher-level DSLs (e.g. UFL) to express more problems without having to resort to abstraction-breaking internal code changes.
As a consequence, implementing novel numerical methods that would have previously been infeasible are now tractable.

% WHAT IS THE PROBLEM THAT I AM TRYING TO SOLVE AND WHY DOES IT MATTER?

% solving PDEs is very important for lots of things
% * climate
% * manufacturing
% * fluids...
% * ???

% 1. what is the problem area? what is it important for?
% - stencil calcs?
% - mesh iteration?

% * mesh iteration is ubiquitous in numerical solution of PDEs

% -----------------

% WHAT ARE THE KEY ISSUES THAT ONE ENCOUNTERS WHEN ATTEMPTING TO SOLVE THIS PROBLEM?

% solving them can be very complicated, and solving them efficiently can
% be VERY complicated. So many things to consider: portability, performance, scaling,
% discretisation/convergence...

% code generation frameworks rely on having a data layout abstraction
% the data layout abstraction needs to be able to interface with the generated code in some
% (performant) way.
% assumptions about the data layout (e.g. regularity, depth) can prohibit numerical methods.
% requires a major redesign in order to get things right/sufficiently flexible.

% "One of the major roadblocks to implementing mesh based simulation code is in"
% "finangling the data structures. Complex algorithms like mesh adaptivity with"
% "hanging nodes require a fundamental rewrite of the code base because it usually"
% "violates a number of assumptions about what the underlying data looks like..."

% i.e. solving PDEs with FEM is characterised by irregular data access patterns
% and complex data structures. Assumptions about data structures inhibit the implementation
% of certain numerical methods.

% 2. Why is it a hard problem to solve?

% other considerations?
% - MPI?
% - portable?

% n. What problems exist with current implementations?
% by hand
% - portability
% automatic
% - like libceed - think this is wrong

% * portability
% * to address limitations in PyOP2's abstraction
% * parallel is hard
%    - pyop3 provides a transparent parallel abstraction
%    - libCEED requires user input

% Data layout choices are "baked-in" at the start of a simulation and implementation must follow suit
% e.g. raggedness or constant inner shape? dual implementation?

% -----------------

% WHAT ARE THE CONCEPTS THAT ARE INTEGRAL TO MY WORK?

% * Praise DSLs!
% Code generation and DSLs provide a separation of concerns.
% Domain specialists can focus
% on the things they do best. Crucially it lowers the barrier to entry for scientists
% to do novel work.

% with codegen, i.e. compilers, one can do some really amazing optimisations not available
% to library interfaces and traditional compilers. In particular cross-loop optimisations?
% Also data layout/memory bandwidth minimising optimisations.

% this is because DSLs provide a high-level representation of the problem where complicated problems may be straightforwardly expressed.

% It is an example of an inspector-executor application.

% key principle: optimisations are easy, synthesis is hard. Level of abstraction
% facilitates memory optimisations.

% -----------------

% "IN THIS THESIS..."

The remainder of this paper is laid out as follows\dots


% ===== NOTES =====

% See M. Homolya's thesis for a nice example of a good introduction. It is basically
% a longer form abstract which also describes the rest of the thesis.

\end{document}
