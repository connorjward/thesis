\documentclass[thesis]{subfiles}

\begin{document}

\chapter{Introduction}
\label{chapter:introduction}

The numerical solution of partial differential equations (PDEs) is ubiquitous across a wide range of applications, from climate modelling to structural mechanics and manufacturing.
In order to solve them numerically, depending on the numerical method chosen, different elements of the geometry and/or function spaces of the physical problem must be \emph{discretised} such that they may be represented on a computer.

\begin{figure}
  \centering
  \includegraphics[scale=1.5]{mesh_stencil.pdf}
  \caption{
    An example of a mesh stencil operation where cells are looped over and adjacent entities are accessed by the stencil (grey).
    DoFs stored on the cells, edges, and vertices are represented by blue, red, and green dots respectively.
  }
  \label{fig:mesh_stencil}
\end{figure}

For a number of numerical methods such as the finite volume, finite difference, and finite element methods, the geometry of the problem is discretised to form a \emph{mesh} composed of topological entities termed cells, with each having sub-entities such as faces, edges and vertices.
Depending on the method and implementation choices, this mesh may be either \textit{structured}, where adjacent entities may be addressed by applying offsets, or \textit{unstructured}, where an indirection map of the form \ccode{A[B[i]]} is needed.
To these topological entities are associated unknowns, termed degrees of freedom (DoFs), representing physical quantities in the PDE such as the pressure.

For all of these methods, an integral part of their usage is the application of a computational kernel, or \emph{stencil}.
To apply one, one loops over a particular collection of mesh entities and, for each entity, computes a contribution to some global data structure using the DoFs that lie in the neighbourhood of the current iterate.
An example of a mesh stencil operation is shown in \cref{fig:mesh_stencil}.

Mesh stencil computations crop up repeatedly in continuum mechanics simulations and developing methods for their application is the central focus of this thesis.

\section{Mesh stencils in practice: the finite element method}
\label{sec:stokes_equations}

As a motivating example for a mesh stencil calculation, we consider solving the Stokes equations, a linearisation of the Navier-Stokes equations in common use in the fluid dynamics community, using the finite element method (FEM).
Our exposition will focus on the aspects of the computation that are relevant for mesh stencils, for a more complete review of FEM we refer the reader to~\cite{brennerMathematicalTheoryFinite2008} and~\cite{larsonFiniteElementMethod2013}.

The Stokes equations are given as follows: for domain $\Omega$ with boundary $\Gamma$, find the fluid velocity $u$ and pressure $p$ such that
\begin{equation} \label{eq:strong_stokes}
  \begin{aligned}
    - \nu \Delta u + \nabla p &= 0 \quad \textrm{in} \ \Omega, \\
    %
    \nabla \cdot u &= 0 \quad \textrm{in} \ \Omega, \\
    %
    u &= g \quad \textrm{on} \ \Gamma,
  \end{aligned}
\end{equation}
where $\nu$ is the viscosity and $g$ is some prescribed Dirichlet boundary condition fixing the velocity across the entire boundary.
Forcing terms are omitted for simplicity.
Since this is a coupled system of two variables ($u$ and $p$), we refer to it as being \emph{mixed}.

\subsection{Deriving a weak formulation}

For the finite element method we seek the solution to the \emph{variational}, or \emph{weak}, formulation of these equations.
These are obtained by multiplying each equation by a suitable \emph{test function} and integrating over the domain.
For \cref{eq:strong_stokes}, using $v$ and $q$ as the test functions for the velocity and pressure spaces respectively, and integrating by parts this gives
% taken from 
% https://nbviewer.org/github/firedrakeproject/firedrake/blob/master/docs/notebooks/06-pde-constrained-optimisation.ipynb
% and Larson and Bengzon (pg. 293)
\begin{equation} \label{eq:weak_stokes_full}
  \begin{aligned}
    \int \nu \nabla u : \nabla v \, \textrm{d}\Omega
    - \int p \nabla \cdot v \, \textrm{d}\Omega
    - \int \nu (\nabla u \cdot n) \cdot v \, \textrm{d}\Gamma
    - \int p n \cdot v \, \textrm{d}\Gamma
    &= 0
    &\forall v
    \\
    \int q \, \nabla \cdot u \, \textrm{d}\Omega
    &= 0
    &\forall q
  \end{aligned}
\end{equation}

\subsubsection{Choosing function spaces}

From this weak form it is now possible to classify the various function spaces for $u$, $p$, $v$ and $q$.
For $u$, we already know that the space must be vector-valued and constrained to $g$ on the boundary.
\Cref{eq:weak_stokes_full} further shows us that $u$ must have at least one weak derivative.
We can therefore say that $u \in V$ where, for dimension $d$,
\begin{equation}
  V = \{ \ v \in [H^1(\Omega)]^d : v |_{\Gamma} = g \ \}.
\end{equation}

The function space of the test function $v$ is similar, but, since the values of $u$ at the boundary are already prescribed, it is defined to be zero at those nodes
\begin{equation} \label{eq:vspace}
  V_0 = \{\ v \in [H^1(\Omega)]^d : v|_{\Gamma} = 0 \ \}.
\end{equation}

Unlike $u$ and $v$, $p$ and $q$ are scalar-valued and need no derivatives.
Further, since no boundary conditions are applied to the space of $p$, we can draw $p$ and $q$ from the same space
\begin{equation}
  Q = \{ \ q \in L^2(\Omega) \ \}.
  \label{eq:stokes_pressure_space}
\end{equation}

Given the definition of these spaces and in particular the fact that $v$ is defined to vanish on the boundary, which allows us to drop the surface integrals, we may restate the problem posed in \cref{eq:weak_stokes_full} as follows: find $(u, p) \in V \times Q$ such that
\begin{equation} \label{eq:weak_stokes_no_surface_terms}
  \begin{aligned}
    \int \nu \nabla u : \nabla v \, \textrm{d}\Omega
    - \int p \nabla \cdot v \, \textrm{d}\Omega
    &= 0
    &\forall v \in V_0 \\
    %
    \int q \, \nabla \cdot u \, \textrm{d}\Omega
    &= 0
    &\forall q \in Q.
  \end{aligned}
\end{equation}

\subsubsection{Resolving boundary conditions}

To better represent the boundary conditions on space $V$, we can split it in two:
\begin{equation*}
  V = V_0 \oplus V_\Gamma,
\end{equation*}
where $V_0$ has the usual definition from \cref{eq:vspace}, and $V_\Gamma$ is defined as the space spanned by the functions of $V$ that are non-zero on the boundary.
With this split we may then rewrite the function $u$ as the sum of the functions $u_0 \in V_0$ and $u_\Gamma \in V_\Gamma$ to give
\begin{equation*}
  u = u_0 + u_\Gamma.
\end{equation*}
$u_\Gamma$ is prescribed by the boundary condition and hence known.
This allows us to move it to the right hand side of \cref{eq:weak_stokes_no_surface_terms} giving: find $(u_0, p) \in V_0 \times Q$ such that
\begin{equation} \label{eq:weak_stokes}
  \begin{aligned}
    \int \nu \nabla u_0 : \nabla v \, \textrm{d}\Omega
    - \int p \nabla \cdot v \, \textrm{d}\Omega
    &=
    - \int \nu \nabla u_\Gamma : \nabla v \, \textrm{d}\Omega
    &\forall v \in V_0 \\
    %
    \int q \, \nabla \cdot u \, \textrm{d}\Omega
    &= 0
    &\forall q \in Q.
  \end{aligned}
\end{equation}

\subsection{Discretising the system of equations}

In order to solve this weak formulation using the finite element method we discretise the function spaces in use by replacing them with finite-dimensional equivalents:
\begin{equation*}
  V_0 \to V_{0,h} \subset V_0,
  \quad
  \quad
  Q \to Q_h \subset Q.
\end{equation*}
Each of these discrete spaces is spanned by a set of basis functions, allowing functions to be expressed as a linear combination of basis functions and coefficients.
For example, we can write the function $u_{0,h} \in V_{0,h}$ as
\begin{equation*}
  u_{0,h} = \sum^N_{i=1} \hat u_i \psi^{V_{0,h}}_i,
\end{equation*}
where $\psi^{V_{0,h}}_i$ are the basis functions that span $V_{0,h}$ and $\hat u_i$ the coefficients.

If we substitute these discrete spaces into \cref{eq:weak_stokes}, and discard the basis coefficients for the arbitrary functions $v_h$ and $q_h$, we obtain the discrete problem: find the vector $(\hat u, \hat p)$ such that
\small
\begin{equation}
  \begin{aligned}
    \int \nu \hat u \nabla \psi^{V_{0,h}} : \nabla \psi^{V_{0,h}} \, \textrm{d}\Omega
    - \int \hat p \psi^{Q_h} \nabla \cdot \psi^{V_{0,h}} \, \textrm{d}\Omega
    &=
    - \int \nu \nabla u_\Gamma : \nabla \psi^{V_{0,h}} \, \textrm{d}\Omega
    \quad
    &\forall \psi^{V_{0,h}}, \\
    %
    \int \psi^{Q_h} \, \hat u \nabla \cdot \psi^{V_{0,h}} \, \textrm{d}\Omega
    &= 0
    \quad
    &\forall \psi^{Q_h}.
  \end{aligned}
\end{equation}
\normalsize
This can be written as the (saddle point) linear system
\small
\begin{equation} \label{eq:stokes_linear_system}
  \left (
  \begin{array}{c|c}
    \int \nu \nabla \psi^{V_{0,h}} : \nabla \psi^{V_{0,h}} \, \textrm{d}\Omega
    &
    - \int \psi^{Q_h} \nabla \cdot \psi^{V_{0,h}} \, \textrm{d}\Omega \\
    \hline
    \int \psi^{Q_h} \, \nabla \cdot \psi^{V_{0,h}} \, \textrm{d}\Omega
    &
    0
  \end{array}
  \right )
  \left (
  \begin{array}{c}
    \hat u \\
    \hline
    \hat p
  \end{array}
  \right )
  =
  \left (
  \begin{array}{c}
    - \int \nu \nabla u_\Gamma : \nabla \psi^{V_{0,h}} \, \textrm{d}\Omega \\
    \hline
    0
  \end{array}
  \right ).
\end{equation}
\normalsize

Solving the Stokes equations using the finite element method therefore boils down to constructing, or \emph{assembling}, the left hand side matrix and right hand side vector, before solving for the coefficients $\hat u$ and $\hat p$.

\subsection{Choosing a basis}

\begin{figure}
  \centering
  \begin{subfigure}[b]{.4\textwidth}
    \centering
    \includegraphics{lagrange_element_2.pdf}
    \caption{
      The dual basis for a $P_2$ element consists of point evaluation at the vertices (green) and edges (red).
    }
    \label{fig:lagrange_element_2_dofs}
  \end{subfigure}
  \begin{subfigure}[b]{.58\textwidth}
    \centering
    \includegraphics[width=.3\textwidth]{element_Lagrange_2_dof0.pdf}\,%
    \includegraphics[width=.3\textwidth]{element_Lagrange_2_dof1.pdf}\,%
    \includegraphics[width=.3\textwidth]{element_Lagrange_2_dof2.pdf}

    \includegraphics[width=.3\textwidth]{element_Lagrange_2_dof3.pdf}\,%
    \includegraphics[width=.3\textwidth]{element_Lagrange_2_dof4.pdf}\,%
    \includegraphics[width=.3\textwidth]{element_Lagrange_2_dof5.pdf}
    \caption{
      The basis functions.
      The images were sourced from \href{https://defelement.com}{defelement.com}~\cite{defelement}.
    }
    \label{fig:lagrange_element_2_basis}
  \end{subfigure}

  \caption{The $P_2$ (Lagrange, polynomial degree 2) finite element.}
  \label{fig:lagrange_element_2}
\end{figure}

In order to numerically evaluate the integrals of \cref{eq:stokes_linear_system}, the basis functions $\psi^{V_{0,h}}$ and $\psi^{Q_h}$ must be known.
In the finite element method these are generated from the choice of \emph{finite element}.

First formalised by Ciarlet~\parencite{ciarletElement2002}, a finite element is the triple $(K, P, N)$, where:
\begin{itemize}
  \item $K$ is a bounded closed set, or \emph{cell}, with non-empty interior and piecewise smooth boundary,
  \item $P$ is a finite-dimensional space of functions on $K$, and
  \item $N$ is a set of linear functionals that form a basis for the dual space of $P$.
\end{itemize}

A simple example of a finite element - the Lagrange element with polynomial degree 2, written $P_2$ - is shown in \cref{fig:lagrange_element_2}.
For this element $K$ is a triangle, $P$ is the space of order 2 polynomials, and the linear functionals of $N$ are defined as the point evaluations
\begin{equation*}
  l_i(v) = v(x_i)
  \quad
  i = 0,1,\dots,5
\end{equation*}
for each of the 6 nodes shown in \cref{fig:lagrange_element_2_dofs}, with $x_i$ being the coordinates of the $i$-th node.

Given this definition, one can determine a basis, $\psi_j$, for $P$ by imposing that
\begin{equation*}
  l_i(\psi_j) = \delta_{ij} \quad i, j = 0, 1, \dots, 5
\end{equation*}
which for the $P_2$ element yields the basis functions:
\begin{align*}
  % see https://defelement.com/elements/examples/triangle-lagrange-equispaced-2.html
  &\psi_0 = 2x^2 + 4xy - 3x + 2y^2-3y+1,
  &
  &\psi_3 = 4xy, \\
  %
  &\psi_1 = x(2x-1),
  &
  &\psi_4 = 4y(-x-y+1), \\
  %
  &\psi_2 = y(2y-1),
  &
  &\psi_5 = 4x(-x-y+1),
\end{align*}
that are shown in \cref{fig:lagrange_element_2_basis}.

Note that the basis functions determined from the finite element have \emph{local support}.
That is, they are only defined to be non-zero in a small region around their associated entities.

\subsubsection{Choosing a basis for the Stokes equations}

\begin{figure}
  \centering
  \begin{subfigure}{.35\textwidth}
    \centering
    \includegraphics{lagrange_element_3_vec.pdf}
    \label{fig:scott_vogelius_element_P3}
  \end{subfigure}
  %
  \begin{subfigure}{.35\textwidth}
    \centering
    \includegraphics{lagrange_element_2_dg.pdf}
  \end{subfigure}
  %
  \caption{
    The degree 3 Scott-Vogelius element ($[P_3]^2 \oplus P_2^\mathrm{disc}$).
    The lower-order space (right) is discontinuous and hence all of the DoFs are marked as belonging to the cell (blue).
  }
  \label{fig:scott_vogelius_element}
\end{figure}

The choice of basis functions used by the function spaces has significant implications for the convergence and stability of the method.
For the two-dimensional Stokes equations, a common choice of element pair, or \emph{mixed element}, with properties matching the constraints given in \cref{eq:vspace} and \cref{eq:stokes_pressure_space} is the Scott-Vogelius element~\cite{scottNormEstimatesMaximal1985}.
The element consists of a continuous vector-valued degree $k$ Lagrange element for the velocity space, written $[P_k]^d$, and a discontinuous Lagrange element of degree $k-1$ for the pressure space, written $P_{k-1}^{\textnormal{disc}}$.
These\footnotemark are shown in \cref{fig:scott_vogelius_element} for $k = 3$.

\footnotetext{In two dimensions the Scott-Vogelius element is technically known to be stable only for $k \geq 4$~\cite{guzmanScottVogeliusFiniteElements2018}, but we show $k = 3$ for simplicity.}

%
% % connectivity
%
% From \cref{fig:lagrange_element_2_dofs} it can be seen that the nodes of the function space are associated with different topological entities: 3 with the vertices and 3 with the edges.
% DoFs on these entities are thus shared between adjacent cells.
% This connectivity can be seen in \cref{fig:mesh_stencil}.

\subsection{Finite element assembly}

The finite element basis functions from the Ciarlet definition above are given in \emph{reference space}.
For them to be usable for assembling the Stokes system of \cref{eq:stokes_linear_system} we must first draw a connection between the domain of the problem, $\Omega$, with the domain of the finite element, $K$.
To do this, we partition $\Omega$ into a set of disjoint cells, $\mathcal{K}$, called a \emph{mesh}, where
\begin{equation*}
  \bigcup \mathcal{K} = \Omega.
\end{equation*}

Owing to the local support property of finite element basis functions, one can now rewrite the integrals in \cref{eq:stokes_linear_system} as a sum of cell-wise contributions.
Using the top left block of the matrix as an example, the integral can be rewritten as
\begin{equation}
  \sum_{\mathcal{K}} \int \nu \nabla \psi_i^{\mathcal{K}} : \nabla \psi_j^{\mathcal{K}} \, \textnormal{d}\Omega_{\mathcal{K}},
\end{equation}
with $\phi^{\mathcal{K}}$ the basis functions over the cell and $i$ and $j$ indicating the different element nodes.
Since the basis functions have only local support the resulting matrix is \emph{sparse}.

Subject to the addition of some Jacobian-like terms, called \emph{pullbacks}, to map between the geometry of the physical and reference cells, these cell-wise integrals can now be evaluated numerically using a quadrature scheme.
This evaluation is termed \emph{local assembly}, and code that performs it is called the \emph{local kernel}.

\subsubsection{Global assembly}

Given such a local kernel, the final piece required for finite element assembly is an outer loop over the cells, $\mathcal{K}$, that adds cell-wise contributions into the global matrix or vector.
This process is called \emph{global assembly} and is an example of a mesh stencil algorithm.

\begin{figure}
  \centering
  \includegraphics[scale=1.4]{fem_assembly.pdf}
  \caption{
    Diagram of the finite element global assembly process.
    For each cell in the mesh a coefficient is packed into a temporary array before the local kernel is evaluated.
    The result, a dense matrix, is then scattered to the global sparse matrix.
  }
  \label{fig:fem_assembly}
\end{figure}

\begin{algorithm}
  \caption{
    Algorithm for assembling a finite element data structure with a single coefficient in the expression.
  }
  %
  \begin{algorithmic}[1]
    \Require \textit{coefficient}, \textit{output} \Comment{Input and output variables}
    \Require \textit{temp0}, \textit{temp1} \Comment{Work arrays}

    \For{\textit{cell} \textbf{in} \textit{mesh.cells}}
      \State \Call{Read}{\textit{temp0}, \textit{coefficient}, \textit{cell}} \Comment{Pack the coefficient}\label{code:pack_demo}
      \State \Call{Zero}{\textit{temp1}} \Comment{Reset \textit{temp1}}
      \State \Call{LocalKernel}{\textit{temp0}, \textit{temp1}} \Comment{Evaluate the integral for each cell}
      \State \Call{Increment}{\textit{output}, \textit{temp1}, \textit{cell}} \Comment{Scatter (unpack) the result}\label{code:unpack_demo}
    \EndFor
  \end{algorithmic}
  \label{alg:fem_assembly}
\end{algorithm}

Since the local kernel operates at the level of a single cell, it expects data to be provided to it as a small dense array, containing only those DoFs necessary for the calculation.
However, due to the complex nature of unstructured mesh data layouts, the global data structures will have arranged the DoFs differently, and so preparing the data for the local kernel requires an explicit transformation between these representations.
Transforming from global to local representations is called \emph{packing}, and from local to global \emph{unpacking}.

A diagram showing the global assembly loop process, including explicit pack/unpack steps, is shown in \cref{fig:fem_assembly} and an algorithm for its execution in \cref{alg:fem_assembly}.
For the latter there is a single pack instruction, \textsc{Read}, on line~\ref{code:pack_demo}, and a single unpack instruction, \textsc{Increment}, on line~\ref{code:unpack_demo}.
Both instructions take the current iterate, \textit{cell}, as an argument because the data that is packed/unpacked is dependent upon it.

\subsection{Finite element data structures}

\begin{figure}
  \centering
  \includegraphics[width=\textwidth]{scott_vogelius_element_dof_layout_packed.pdf}
  \caption{
    The packing transformation taking global data (top) to packed local data (bottom) for a two-dimensional Scott-Vogelius function space.
    The global data stores all DoFs for the entire function space whilst the packed data only contains the DoFs present in the current stencil iterate.
    DoFs associated with vertices, edges, and cells are coloured green, red, and blue respectively.
  }
  \label{fig:scott_vogelius_element_dof_layout_packed}
\end{figure}

Determining the right pack/unpack operations for a finite element global assembly loop is challenging because:
(a) unstructured mesh data has a non-trivial structure that is difficult to reason about, and
(b) DoFs corresponding to different types of mesh entity (e.g. edges and vertices) must be packed together contiguously.
These issues are jointly demonstrated in \cref{fig:scott_vogelius_element_dof_layout_packed}, which shows a packing operation for a Scott-Vogelius function space suitable for solving the Stokes equations.
Both the local and global representations of the data have some hierarchical structure.
For instance:
\begin{itemize}
  \item All of the DoFs in space $V_h$ precede those of $Q_h$.
  \item The \labelComponent{0} and \labelComponent {1} vector components of $V_h$ are stored adjacently in memory.
  \item For $V_h$, vertices and cells have one node, whilst edges have two.
\end{itemize}
Since the packed local data (bottom) only exists as a temporary its contents are parametrised by the current cell iterate, $c_i$.
For example, the first vertex that is packed for a particular cell is represented by the function $f^v_0(c_i)$ where $f^v_i$ is the map relating cells to incident vertices.

Developing abstractions to representing this hierarchical structure is challenging, and forms one of the main contributions of this thesis.

\section{Software for continuum mechanics}
\label{sec:introduction_software}

\todo[inline]{This section is a bit off. The key points are there but the phrasing is not great.}

Developing simulation software from scratch is very difficult, requiring multiple person-years of developer time and expertise from a range of disciplines.
Instead, it is usual to use an existing `toolbox' package from which the simulation may be built.
Using pre-existing packages substantially lower the barrier-to-entry for scientists and provide excellent performance and scalability whilst also being productive.
% NOTE: Bit unclearly worded
These packages may be broadly split into two categories: `hand-written' and those that rely on domain-specific compilers and high-level abstractions.

\subsubsection{Hand-written packages}

By far the simpler to produce, hand-written packages are composed of a set of functions and types, usually written in a compiled language like C or \cplusplus, that are called by the user via the package's API.
Hand-written packages have seen tremendous success in the scientific community with packages such as BLAS~\cite{lawsonBasicLinearAlgebra1979}, a suite of linear algebra routines used in most scientific software.
However, the successes of such packages has in part relied on the simplicity of the features that they implement.
As the software gets more and more complex, more and more variants of the same program must be written in order to get optimal performance - an infeasible proposition.
Further, hand-written codes necessarily combine the mathematical details of the problem with low-level optimisations in the code.
This makes collaboration difficult, as code developers require expertise in both the mathematics and software optimisation, and also means that numerous assumptions about architecture, data layout, and many more are `baked-in' and exceedingly difficult to change at a later date (e.g. if porting the code to a GPU).

For continuum mechanics problems, providing a general purpose toolbox package with a hand-written library is especially challenging.
Most of the performance-critical code - the `hot loops' - is specialised to the choice of discretisation and PDE.
In order to support a wide range of simulation types, PDE-solving packages like \texttt{deal.II}~\cite{dealII95,dealii2019design} and libCEED~\cite{brownLibCEEDFastAlgebra2021} require the user to provide custom kernels.

\todo[inline]{This doesn't read very well, but I want to make a point that abstractions are "zero-cost" later on.}

To detangle different pieces of code, libraries may use higher-level abstractions such as object-oriented programming.
Adding abstractions at this point leads to an increased overhead though as the compiled code needs to follow more pointers to access methods and adding functions and methods creates `black boxes' that the compiler struggles to optimise through.

\subsubsection{Domain-specific compiler packages}

As an alternative approach, a number of packages instead use code generation to automate the process.
Code generation operates on alternative, domain-specific high-level representations of the problem that may then be lowered, possibly via a series of intermediate representations, to an executable.
Unlike hand-written codes, these high-level abstractions have `zero-cost' as they only exist at compile-time.
Having multiple representations of the problem has a number of benefits:

% TODO: I think performance should be its own paragraph, the others can be lumped together
\begin{itemize}
  \item
    It allows for a separation of concerns between users and developers, facilitating contributions from users without a background in performance optimisation.
  \item
    High level representation is more productive to develop in.
  \item
    By representing the problem at a high level, it is sometimes possible for the compiler to apply specialised optimisations that would be difficult to apply by hand.
    Whilst always the case that any automatically generated code could also have been written by hand, compilers frequently generate code with comparable or even superior~\cite{ragan-kelleyHalideLanguageCompiler} performance to hand-written implementations.
    An example of this is sum factorisation, a FLOP-reduction technique applicable to finite element simulations with tensor-product structure~\cite{homolyaExposingExploitingStructure2017}.
    The compiler can even sometimes do auto-tuning, where a space of possible solutions is explored to find the fastest one.
    Examples include ATLAS (dense linear algebra)~\cite{whaleyAutomatedEmpiricalOptimizations2001}, SPIRAL (signal processing)~\cite{puschelSPIRALCodeGeneration2005}, FFTW (discrete Fourier transforms)~\cite{frigoDesignImplementationFFTW32005}, Halide (image processing)~\cite{ragan-kelleyHalideLanguageCompiler}, and the Sparse Polyhedral Framework (sparse linear algebra)~\cite{stroutSparsePolyhedralFramework2018}.
    If done at runtime, auto-tuning is also sometimes referred to as an \emph{inspector-executor strategy}.
\end{itemize}

Examples of continuum mechanics simulation packages using domain-specific compilers include the finite element libraries Dune-Fem~\cite{dednerGenericInterfaceParallel2010}, FEniCSx~\cite{barattaDOLFINxNextGeneration2023}, and Firedrake~\cite{FiredrakeUserManual}, all using the same DSL for expressing weak forms UFL~\cite{alnaesUnifiedFormLanguage2014a}, and the finite difference library Devito~\cite{devito-api,luporiniArchitecturePerformanceDevito2020}, that uses Sympy~\cite{10.7717/peerj-cs.103} as its top-level DSL.

Despite these benefits, DSLs have not been adopted wholesale into the continuum mechanics simulation community for a number of reasons:

\begin{itemize}
  \item
    Vast resources have been expended to produce hand-coded implementations with exceptional performance.
    Current performance of generated code is not competitive and harder to improve given that it is hidden behind a compiler.
  \item
    Compilation at run-time can be slow.
  \item
    Abstractions are fundamentally limiting, numerical methods currently being researched may not be expressible in the abstraction and hence may be impossible to implement without redesigning the abstraction.
\end{itemize}

Clearly, further research to address these concerns is required.

\todo[inline]{The takeaway from this section \textit{should} be that DSLs are great for continuum mechanics but that designing the right abstractions is both important and difficult.}
% "good codegen relies on capturing behaviour via powerful abstractions"
% else need sui-generis solutions that do not generalise/extend
% "getting the abstraction right (and at the right level) lets us do things like sum factorisation (FInAT) and automatic adjoint (pyadjoint)"

\subsection{Requirements for a mesh stencil compiler}

\todo[inline]{As David stressed, we need to talk about what a stencil language should look like/what it needs!}
% i.e. the *essential* parts of the abstraction.
% e.g. "ability to represent a wide range of kernels and stencil definitions" - execution model
% also needs an appropriate "data model" (i.e. function spaces etc)
% "stencil flexibility" and "data flexibility"
% e.g. PyOP2 pre-commits to particular stencil patterns - leads to sui-generis...
% TODO: Include these in the checklist!!!

In addition to having the right high-level abstractions and backing code generation infrastructure, there a number of essential requirements that a mesh stencil package must meet in order for it to be effective.
These requirements are:

\todo[inline]{No, these are \emph{not} requirements, but desirable characteristics}

\subsubsection{Distributed memory parallelism}
\label{sec:intro_parallelism}

\begin{figure}
  \centering
  \includegraphics{split_mesh.pdf}
  \caption{
    Diagram showing the DoFs for a function distributed between 2 processes.
    DoFs existing on multiple processes are represented by diamonds, with filled diamonds being those belonging to that process and hollow ones being ghost DoFs.
    DoFs that are not duplicated across processes are shown as circles.
    Sample (cell-wise) stencils are shown in grey to demonstrate that the ghost DoFs are required for the stencils to be computable.
  }
  \label{fig:pyop2_split_mesh}
\end{figure}

In massively parallel simulations the data structures are often too large to be stored in the memory of any individual node.
Instead, they are \textit{distributed} between all of the processes, with each process owning, and seeing, only a small piece of the entire structure.
For vectors it is usual to store the DoFs according to some partitioning of the mesh (\cref{fig:pyop2_split_mesh}).
Depending on the stencil calculation, processes often need to store DoFs at their boundaries belonging to different processes; these entities are frequently referred to as \textit{ghost points}.
For matrices the DoF partitioning process is less straightforward.
Linear algebra libraries such as PETSc~\cite{petsc-user-ref,petsc-web-page,petsc-efficient} provide routines for accessing parallel matrices in a manner such that off-process DoF accesses are hidden.
\textbf{Mesh stencil packages need to have support for distributed data structures as well as have the right routines for ensuring the correctness of the ghost points.}

\subsubsection{Performance portability}

\todo[inline]{Need to make it clear in the structure of this section that we want performance portability to be \emph{possible}, not necessarily delivered!}
% motivates codegen, perhaps needn't be mentioned here but only above in the "abstraction/codegen" bit

Due to the search for ever-increasing computational power and energy efficiency, it is no longer possible to write a piece of low-level code and deploy it on all supercomputers.
GPU machines now beginning to dominate the HPC landscape.
Also ARM (RISC) CPUs are popular (e.g. Fugaku).
Programming for these disparate architectures requires both new programming languages/models, in the case of GPUs, as well as high-level algorithmic changes~\cite{betteridgeCodeGenerationProductive2021} to get acceptable performance.
The situation is further complicated by the programming models being vendor-specific with languages such as CUDA, HIP and Sycl all being specially designed to get performance from a particular vendor's accelerators.
\textbf{A successful mesh stencil code should be able to switch between backends with only minimal changes to both user and internal code, and be reasonably performant for them all.}

\begin{figure}
  \centering
  \begin{subfigure}{\textwidth}
    \centering
    \includegraphics{two_cell_mesh_lagrange.pdf}
    \vspace{1em}
  \end{subfigure}
  \begin{subfigure}{\textwidth}
    \centering
    \includegraphics{two_cell_mesh_lagrange_data_layout_flat.pdf}
  \end{subfigure}
  \caption{
    Data layout for a $P_3$ function space on a two cell mesh.
    The mesh data has been renumbered such that edge and vertex DoFs are `close' to their incident cell.
  }
  \label{fig:mesh_renumbering_demo}
\end{figure}

\subsubsection{Mesh renumbering}

During the execution of a mesh stencil the input and output global data structures are accessed with indirection maps.
Accessing the data indirectly incurs a performance penalty because it defeats hardware-implemented memory optimisations such as prefetching, where adjacent blocks of memory are loaded into a cache line together.
To improve data locality, a common optimisation is to \textit{renumber the mesh entities} such that, say, edge DoFs are `close' to the DoFs of their incident cell (e.g. Algorithm 3 of~\cite{langeEfficientMeshManagement2016}).
This increases both the chance that adjacent required DoFs will be loaded onto the same cache line (spatial locality), and also the chance that DoFs shared between successive iterates will already be in cache (temporal locality).
An example of such a renumbering is shown in \cref{fig:mesh_renumbering_demo}; the data for the vector is arranged such that incident edge and vertex DoFs are adjacent in memory to their cell.
Whilst not necessary for low-order methods, where DoFs are only stored on a single entity type, \textbf{a truly generic mesh stencil code needs to have support for `interleaving' DoFs of different entity types.}

\begin{figure}
  \centering
  \includegraphics{scott_vogelius_element_dof_layout_swap.pdf}
  \caption{
    An alternative, `struct-of-array' data layout for the Scott-Vogelius function space of \cref{fig:scott_vogelius_element_dof_layout_packed}.
    Notice how the components of the vector space $V_h$ are now stored apart from one another.
  }
  \label{fig:scott_vogelius_element_dof_layout_swap}
\end{figure}

\todo[inline]{does referencing this label work?}
\subsubsection{Data layout flexibility}
\label{sec:intro_data_layout_flex}

In order to get optimal performance on different platforms, especially when transitioning from CPU to GPU architectures, the way in which data is laid out must be altered (e.g.~\cite{markallFiniteElementAssembly2013,sulyokImprovingLocalityUnstructured2018}).
This poses a challenge to many simulation packages as they often commit to a particular representation of the data, typically one optimised for CPU execution, posing difficulties for achieving performance portability without a substantial rewrite of much of the software.
One possible way in which this manifests is whether or not to use a `struct-of-array' (SoA) or `array-of-structs' (AoS) data layout.
Applied to a finite element function space these data layouts can manifest as, say, having vector components stored adjacent to one another (AoS, \cref{fig:scott_vogelius_element_dof_layout_packed}) or separately (SoA, \cref{fig:scott_vogelius_element_dof_layout_swap}).

\textbf{As a consequence, we naturally want mesh stencil codes to be able to support a variety of data layout representations, ideally without any major changes to the code.}
Taking a high-level abstraction approach is useful for this as a separation of concerns may be drawn between the semantics of the data and its materialisation.
Examples of such abstractions providing high-level interfaces for constructing differing data layouts include Taichi~\cite{huTaichiLanguageHighperformance2019}, a DSL for graphics simulations, and TACO~\cite{kjolstadTacoToolGenerate2017}, a compiler for tensor algebra.
Both packages separate the specification of the data layouts from the implementation of the algorithms and have been shown to have excellent performance.

\subsection{Existing mesh stencil packages}
\label{sec:intro_existing_software}

\todo[inline]{Maybe have macros to make sure the headings agree with above (they currently don't!).}
\begin{table}
  \centering

  \begin{tblr}{|[1pt]l|[1pt]c|[1pt]c|[1pt]c|[1pt]c|[1pt]c|[1pt]c|[1pt]}
    \hline[1pt]
    & \textbf{Liszt} & \textbf{Simit} & \textbf{Ebb} & \textbf{MeshTaichi} & \textbf{OP2} & \textbf{PyOP2} \\
    \hline[1pt]
    Distributed memory parallelism & \mytick & \mycross & \mycross & \mycross & \mytick & \mytick \\
    \hline
    Performance portability & \mytick & \mytick & \mytick & \mytick & \mytick & \mycross\footnotemark \\
    \hline
    Mesh renumbering\footnotemark & \mycross & \mytick & \mycross & \mycross & \mytick & \mytick \\
    \hline
    Dynamic data layouts & \mycross & \mytick & \mytick & \mytick & \mycross & \mycross \\
    \hline
    Actively maintained & \mycross & \mycross & \mycross & \mytick & \mytick & \mytick \\
    \hline[1pt]
  \end{tblr}
  \caption{Comparison of the features of some pre-existing mesh stencil packages.}
  \label{tab:existing_stencil_code_capabilities}
\end{table}

% FIXME: These don't label correctly
\footnotetext{Proof-of-concept GPU support has been demonstrated for PyOP2~\cite{fenics2021-kulkarni}.}
\footnotetext{Simit, OP2, and PyOP2 all support storing data on generic set-like structures so any DoF ordering may be achieved.}

A number of mesh stencil packages already exist, each meeting a subset of the requirements laid out above (see \cref{tab:existing_stencil_code_capabilities}).
\textbf{Liszt}~\cite{devitoLisztDomainSpecific2011}, \textbf{Simit}~\cite{kjolstadSimitLanguagePhysical2016}, \textbf{Ebb}~\cite{bernsteinEbbDSLPhysical2016}, and \textbf{MeshTaichi}~\cite{yuMeshTaichiCompilerEfficient2022} are DSLs for mesh stencil computations embedded in Scala, \cplusplus, Lua, and Python respectively.
Each provides a rich API for expressing a variety of loop constructs and stencil operations that may be compiled and executed on a selection of different backends.
Liszt has support for shared-memory, distributed-parallel, and GPU backends whereas Simit, Ebb, and MeshTaichi only support shared-memory and GPUs.

In contrast with these expressive DSLs, \textbf{OP2}~\cite{mudaligeOP2ActiveLibrary2012} and \textbf{PyOP2}~\cite{rathgeberPyOP2HighLevelFramework2012} are mesh stencil packages with much more restrictive interfaces.
Rather than providing a full problem-solving environment, OP2 and PyOP2 programs have a single entrypoint, termed a \textit{parallel loop}, that takes an externally provided local kernel along with an \textit{iteration set} and directly or indirectly addressed \textit{arguments}.
This approach maps cleanly to FEM-style assembly loops such as that shown in \cref{alg:fem_assembly}, which in PyOP2 would be expressed as

\begin{pyinline}
  par_loop(kernel,                                       # local kernel
           cells,                                        # iteration set
           coefficient(READ, cell_node_map),             # arguments
           output(INC, (cell_node_map, cell_node_map)))
\end{pyinline}

OP2 is an \textit{active library} that provides source-to-source translation from C, \cplusplus or Fortran to a range of different backends.
By contrast, PyOP2 is a reimplementation of the OP2 API in Python, using run-time code generation to compile the parallel loops.
By using code generation, PyOP2 is easier to integrate with other code generation packages; in particular it is a core component of the Firedrake finite element framework~\cite{FiredrakeUserManual}, which is written in Python.

Whilst straightforward to implement, having a single entrypoint to these libraries can be restrictive.
Occasionally one needs to be able to have nested loops, or execute multiple kernels in sequence.
Convincing PyOP2 to perform these sorts of operations has been done (e.g.~\cite{gibsonSlateExtendingFiredrake2020,farrellPCPATCHSoftwareTopological2021}), but only in a sui-generis manner, resulting in code that is harder to maintain and not composable with other features.

Of the packages that support distributed-memory parallelism, OP2 and PyOP2 differ from Liszt in that their abstractions have no native mesh type; instead dealing in more abstract \textit{sets} and \textit{mappings} between sets.
Being able to do this is convenient as building a mature unstructured mesh package is difficult, making the development and maintenance of the stencil package harder.

\subsection{The missing abstraction}
\label{sec:intro_missing_abstraction}

\begin{figure}
  \centering
  \includegraphics{scott_vogelius_element_dof_layout_topological.pdf}
  \caption{
    The closest possible data layout for \cref{fig:scott_vogelius_element_dof_layout_packed} for a library that associates unknowns with topological entities.
    Data for each topological entity are stored in separate arrays.
  }
  \label{fig:scott_vogelius_element_dof_layout_topological}
\end{figure}

\begin{figure}
  \centering
  \includegraphics{scott_vogelius_element_dof_layout_pyop2.pdf}
  \caption{
    The data layout matching \cref{fig:scott_vogelius_element_dof_layout_packed} as it would be stored by \pyop2.
    Data for each function space ($V_h$ and $Q_h$) are stored in separate arrays.
  }
  \label{fig:scott_vogelius_element_dof_layout_pyop2}
\end{figure}

All of the packages listed above follow one of two possible approaches to storing mesh data.
Liszt, Ebb, and MeshTaichi all store mesh data by associating fields with mesh entities (e.g. vertices).
This means that high-order function spaces that store DoFs on multiple entities (e.g. \cref{fig:scott_vogelius_element}) are difficult to express and would have to be stored in a manner similar to that shown in \cref{fig:scott_vogelius_element_dof_layout_topological}.
Consequently, mesh renumbering data locality optimisations are impossible to express.

As an alternative approach, Simit, OP2, and PyOP2 are able to represent function space DoFs more generically as \textit{sets of nodes}.
Being able to treat all mesh entities equivalently means that mesh renumbering is possible but comes at the expense of topological information.
This is demonstrated in \cref{fig:scott_vogelius_element_dof_layout_pyop2}, which shows the data layout for a Scott-Vogelius function space as it would be stored in \pyop2.
Observe that the topological information - the \pycode{"mesh"} layer - is no longer present, emphasising that the relation between nodes and their owning mesh entities is lost.
This is the principle reason why developing expressive DSLs for mesh stencil calculations is difficult: \textbf{topological information is essential for a composable abstraction}.

\todo[inline]{Fundamentally, node sets do not \textit{look like} function spaces. Hence the abstraction is weak. We can point to things (e.g. pcpatch) where things break down, but fundamentally getting the abstraction right is a good thing in and of itself and is worth doing.}

\todo[inline]{
  Need section stating "the contribution of this thesis is...".
}
% *abstraction* for mesh stencil computations with the following characteristics
% an (incomplete) implementation of said abstractions, pyop3
% "pyop3 is an execution model and data model"
% "gives you things that can't even write in PyOP2..."

\section{Thesis outline}

In this thesis we develop abstractions and associated infrastructure to bridge this gap.
The core of this novel work is a new abstraction for describing hierarchical data layouts, termed \textit{axis trees}.
Axis trees allow the user to describe complex data layouts of the sort shown in \cref{fig:scott_vogelius_element_dof_layout_packed} fully, without needing to discard any of the topological information.
As a convenient side benefit, expressing data layout transformations (\cref{sec:data_layout_transformations}) becomes natural to do.

This is implemented in the new mesh stencil package \pyop3, a near-total rewrite of PyOP2 that aims to substantially improve its expressivity power and composability.

The remainder of this thesis is structured as follows:

\todo[inline]{This isn't very nice as a bulleted list, probably put into a paragraph.}

\begin{itemize}
  \item
    \cref{chapter:foundations} introduces the pieces of scientific software that contributed to \pyop3's design, including its precursor PyOP2 and the unstructured mesh framework PETSc DMPlex.
  \item
    \cref{chapter:axis_trees} and \cref{chapter:indexing} present axis trees for the first time, as well as techniques for their symbolic manipulation (termed \textit{indexing}).
  \item
    \cref{chapter:pyop3} 
  \item
    \cref{chapter:parallel} discusses extensions to the base design to enable distributed-memory parallelism.
  \item
    \cref{chapter:firedrake} integrates \pyop3 into the finite element package Firedrake, replacing PyOP2, and details of this are discussed.
  \item
    \cref{chapter:demonstrator_applications} does some simple performance tests and \cref{chapter:summary} is for closing remarks.
\end{itemize}

\todo[inline]{
  Not all of the algorithms that I present in the rest of the thesis have been merged into \pyop3 main because I ran out of time.
  In particular the layout tabulation and parallel indexing bits haven't been added.
  Do I need some sort of disclaimer on this?
  This could potentially be a good place for that.
}

% NOTE: I don't like including this but I think I have to be honest about this.
% put in a box
% Disclaimer:
% Whilst all of the functionality of \pyop3 in the following thesis has been tested and shown to work, the time constraints of the PhD mean that \pyop3 is unfinished.
% Thus not all of the provided functionality has been merged into the \texttt{main} branch yet.
% This is considered very high priority future work for the project.

% ^^^
% I can get around this by making it clear that the principle contribution is the abstraction.
% pyop3 is an incomplete implementation!

% "Though now quite capable, \pyop3 is only an \emph{incomplete} implementation of the abstractions presented in this thesis.
% Further development work is required before it is suitable for production use."


\end{document}
