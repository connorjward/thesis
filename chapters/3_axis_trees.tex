\documentclass[thesis]{subfiles}

\begin{document}

\chapter{Axis trees}
\label{chapter:axis_trees}

As we have seen thus far, existing software abstractions for mesh-like data layouts are limited by their ability to flexibly describe complex layouts without discarding important topological information.
We address this by introducing a new abstraction for describing data layouts: \emph{axis trees}.
The fundamental characteristics of an axis tree are as follows:

\begin{itemize}
  \item
    An axis tree is a tree of \emph{axis} objects.

  \item
    Each axis object has one or more \emph{axis components}.

  \item
    The axis at the top of the tree is called the \emph{root} axis.

  \item
    \emph{Non-root axes attach to components of their parent axis.}
    That is, each axis has a \emph{parent} consisting of the 2-tuple $(\textnormal{\it parent axis}, \textnormal{\it parent component})$.

  \item
    Both axes and components have a \emph{label}.

  \item
    No ancestors or children of an axis may share the same label.
    Axes in a tree may have the same labels, but only if they do not share a common \emph{path} from root to leaf.

  \item
    Axis component labels must be unique within an axis.

  \item
    Axis components store a size attribute.
\end{itemize}

\todo[inline]{Need to be explicit about what an axis is.}
% and put as definitions...

Axis trees are analogous to the metadata of a \numpy{} array, or a \pyop2 \pycode{DataSet}: they exist alongside a block of memory and provide the information necessary to interpret it.
Axis trees are especially similar to \numpy{} arrays in that they both describe multi-dimensional data and accept multi-indices that are transformed into offsets.
Indeed, single-component axes are equivalent to the axes of a \numpy{} array, hence the choice of name.
The core difference to these existing abstractions that axis trees bring is the ability to describe more heterogeneous data layouts since axes can \emph{branch}.

\subsubsection{Example 1: Linear axis trees}

\begin{figure}
  \centering
  \begin{subfigure}{.9\textwidth}
    \begin{pyalg2}
      axis_a = Axis({"x": 2}, "a")
      axis_b = Axis({"x": 3}, "b")
      axis_c = Axis({"x": 2}, "c")
      axes = AxisTree.from_iterable([axis_a, axis_b, axis_c])
    \end{pyalg2}
    \caption{\pyop3 code to build the axis tree.}
  \end{subfigure}

  \vspace{1em}

  \begin{subfigure}[t]{.3\textwidth}
    \centering
    \includegraphics{linear_axis_tree_complete.pdf}
    \caption{The axis tree.}
  \end{subfigure}
  \begin{subfigure}[t]{.3\textwidth}
    \centering
    \includegraphics{linear_data_tree_complete.pdf}
    \caption{The data layout.}
  \end{subfigure}
  \caption{
    A linear axis tree.
    The axes are labelled $a$, $b$, and $c$ and all have a single component, labelled $x$.
  }
  \label{fig:linear_axis_tree}
\end{figure}

As a first example to demonstrate axis trees we consider the simplest case: \emph{linear} axis trees.
Linear axis trees are defined as being axis trees where all of the axes have a single component.
They are equivalent to multi-dimensional arrays in \numpy{}.

An example of a linear axis tree is shown in \cref{fig:linear_axis_tree}.
Consisting of 3 axes labelled $a$, $b$, and $c$, it is equivalent to a \numpy{} array with shape \pycode{(2, 3, 2)} with axis $a$ of the axis tree matching axis 0 of the \numpy{} array and so on.
Each axis has a single component, all labelled $x$, with sizes 2, 3, and 2.

Some noteworthy features of this axis tree include:
\begin{itemize}
  \item
    The axes have unique labels ($a$, $b$, and $c$).
    Labelling two of the axes $a$, say, would not be allowed.

  \item
    The axis components all have the same label ($x$).
    This is permitted because component labels need only be unique \emph{per axis}.

  \item
    The parent-child relationships of the different axes are as follows:
    \begin{itemize}
      \item Axis $a$ is the root of the tree and so has no parent.
      \item Axis $b$ is the child of component $x$ of axis $a$, hence its parent is the 2-tuple $(a, x)$, sometimes written $a^x$.
      \item Axis $c$ is the child of $(b, x)$.
    \end{itemize}

  \item
    As the axis tree is linear, there is only one path from root to leaf: $\{ a^x, b^x, c^x \}$.
\end{itemize}

Axis components only need labels to be distinguishable within a multi-component axis.
For single-component axes this ambiguity is not present and so in future component labels will be omitted in this case for brevity.

\subsubsection{Example 2: Multi-component axis trees}

\begin{figure}
  \begin{subfigure}{.9\textwidth}
    \begin{pyalg2}
      axis_a = Axis({"x": 2, "y": 2}, "a")
      axis_b = Axis(3, "b")
      axis_c = Axis(2, "c")
      axes = AxisTree.from_nest({axis_a: [axis_b, axis_c]})
    \end{pyalg2}
    \caption{\pyop3 code to build the axis tree.}
  \end{subfigure}

  \vspace{1em}

  \begin{subfigure}{.4\textwidth}
    \centering
    \includegraphics{multi_component_axis_tree.pdf}
    \caption{The axis tree.}
  \end{subfigure}
  \begin{subfigure}{.4\textwidth}
    \centering
    \includegraphics{multi_component_data_tree.pdf}
    \caption{
      The data layout.
      The distinct components of axis $a$ are coloured red and blue for emphasis.
    }
    \label{fig:multi_component_data_tree}
  \end{subfigure}
  \caption{
    A simple multi-component axis tree.
  }
  \label{fig:multi_component_axis_tree_intro}
\end{figure}

The natural extension to linear axis trees is to consider the case where axes are permitted to have more than one component.
An example of such a multi-component axis tree is shown in \cref{fig:multi_component_axis_tree_intro}.
Compared with the linear case, this axis tree has a few key differences:
\begin{itemize}
  \item
    Axes $b$ and $c$ attach to \emph{different components} of the shared axis $a$.
    The parent of $b$ is $(a, x)$ whilst the parent of $c$ is $(a, y)$.
  \item
    Since axes $b$ and $c$ have a single component, the component label has been omitted.
  \item
    There are now two possible paths from root to leaf: $\{ a^x, b \}$ and $\{ a^y, c \}$.
\end{itemize}

When materialised as a data layout (\cref{fig:multi_component_data_tree}), the multiple components of axis $a$ are stored apart from each other, with all entries of component $x$ preceding those of component $y$.

\section{Data structures}

Axis trees are \emph{data layout descriptors} providing a means to interpret blocks of memory.
In the mesh stencil library \pyop3, where axis trees have been implemented, actually storing data requires new objects coupling axis trees to data.
Reusing terminology from \pyop2, \pyop3 has \pycode{Globals} (scalars), \pycode{Dats} (vectors), and \pycode{Mats} (matrices).
\pycode{Globals}, being scalar-valued, need no axis trees, \pycode{Dats} require one, and \pycode{Mats} require two (one each for the rows and columns).

Being the most widely used, we will mostly focus of \pycode{Dats} during this thesis.
\pycode{Mats} are more complicated and a discussion of their implementation is deferred to \cref{sec:impl_matrices}.

\section{Alternative data layouts}
\label{sec:axis_tree_alternative_layouts}

\begin{figure}
  \centering
  \begin{subfigure}{.45\textwidth}
    \centering
    \begin{minipage}{.35\textwidth}
      \begin{center}
        \includegraphics{linear_axis_tree.pdf}
      \end{center}
    \end{minipage}
    \begin{minipage}{.6\textwidth}
      \begin{center}
        \includegraphics{linear_data_tree.pdf}
      \end{center}
    \end{minipage}
  \end{subfigure}
  %
  % \hspace{1em}
  %
  \begin{subfigure}{.45\textwidth}
    \centering
    \begin{minipage}{.35\textwidth}
      \begin{center}
        \includegraphics{linear_axis_tree_flip.pdf}
      \end{center}
    \end{minipage}
    \begin{minipage}{.6\textwidth}
      \begin{center}
        \includegraphics{linear_data_tree_flip.pdf}
      \end{center}
    \end{minipage}
  \end{subfigure}

  \caption{
    Two equivalent data layouts storing the same information.
    With the left axis tree, axis $c$ is `innermost', with unit stride, whilst for the right tree this is axis $a$.
  }
  \label{fig:linear_axis_tree_flip}
\end{figure}

By relying on labels to identify axes in an axis tree, it is straightforward to express alternative layouts for the same data - a desirable characteristic for mesh stencil models (\cref{sec:intro_data_layout_flex}) - simply by reordering the axes in the tree.
For example, \cref{fig:linear_axis_tree_flip} shows two equivalent axis trees with differently ordered axes.
By leaving the axes themselves unchanged, and only switching their order, the semantics of the layout are preserved even though the access patterns are changed.

In order to further facilitate this behaviour, \pyop3 is carefully designed to avoid making assumptions about the ordering of axes.
For instance, paths, an essential component to many algorithms, are represented as unordered \emph{sets}, rather than lists.
For \cref{fig:linear_axis_tree_flip}, this means that the paths from root to leaf - $\{ a, b, c \}$ and $\{ b, c, a \}$ - are exactly the same.
By avoiding assumptions about axis ordering, a user can create a data layout using any ordering they choose, and the rest of the code will continue to work without modification.

\section{Ragged axis trees}
\label{sec:ragged_axis_trees}

\begin{figure}
  \centering
  \begin{subfigure}{.9\textwidth}
    \begin{pyalg2}
      # make the size array
      axis_a = Axis(2, "a")
      axis_b = Axis(2, "b")
      size_axes = AxisTree.from_iterable([axis_a, axis_b])
      size_dat = Dat(size_axes, data=[1, 0, 2, 1])

      # make the full axis tree
      axis_c = Axis(size_dat, "c")?\label{code:ragged_size_dat}?
      axes = AxisTree.from_iterable([axis_a, axis_b, axis_c])
    \end{pyalg2}
    \caption{\pyop3 code to build the axis tree.}
  \end{subfigure}

  \vspace{1em}

  \begin{minipage}{.35\textwidth}
    \begin{center}
      \includegraphics{ragged_axis_tree.pdf}
    \end{center}
  \end{minipage}
  %
  \begin{minipage}{.4\textwidth}
    \begin{center}
      \includegraphics{ragged_data_tree.pdf}
    \end{center}
  \end{minipage}
  %
  \begin{minipage}{.2\textwidth}
    \begin{center}
      \includegraphics{ragged_axis_tree_size.pdf}
    \end{center}
  \end{minipage}
  %
  \caption{
    A ragged axis tree (left) and the resulting data layout (middle).
    The cross in the data layout indicates that no values exist for the multi-index $\{a_0, b_1\}$.
    The smaller axis tree (right) is the axis tree for the size array \pycode{[[1,0],[2,1]][?$i_a$?,?$i_b$?]}.
  }
  \label{fig:ragged_axis_tree}
\end{figure}

Thus far we have only considered data layouts where the axis components have constant sizes.
However, there are circumstances where the size is variable; for instance when representing a map from vertices to their incident edges ($\support(v)$, \cref{sec:dmplex_queries}), as the number of incident edges is not constant for all vertices.
Axis trees describing this sort of structure are described as \emph{ragged}.

Ragged axis trees are created by passing a \pycode{Dat} as the size of the axis component, instead of an integer.
This is shown in \cref{fig:ragged_axis_tree}.
Here, the axis tree has 3 axes labelled $a$, $b$, and $c$.
Axes $a$ and $b$ both have a constant size of 2 but $c$, the innermost axis, has a variable size given by the linear \pycode{Dat} with shape \pycode{(2, 2)} and axis labels $a$ and $b$ (line~\ref{code:ragged_size_dat}).

The values of this \pycode{Dat} can be written as \pycode{[[1,0],[2,1]][?$i_a$?,?$i_b$?]}, meaning that the size of $c$ is determined by the table:
\begin{center}
  \begin{tblr}{|[1pt] X[1.2cm,c] |[1pt] X[1.2cm,c] |[1pt] c |[1pt]}
    \hline[1pt]
    \boldmath${i_a}$ & \boldmath$i_b$ & \boldmath$\textnormal{\bf size}(i_a, i_b)$ \\
    \hline[1pt]
    0 & 0 & 1 \\
    \hline[1pt]
    0 & 1 & 0 \\
    \hline[1pt]
    1 & 0 & 2 \\
    \hline[1pt]
    1 & 1 & 1 \\
    \hline[1pt]
  \end{tblr}
\end{center}

\section{Computing offsets}
\label{sec:axis_tree_layouts}

In order for axis trees to provide a complete specification of a data layout, they need to provide a mechanism that transforms multi-indices into offsets.
To that end, axis trees carry additional \emph{layout functions}, that take a multi-indices and return offsets.

Layout functions are computed automatically from the specification of the axis tree.
They are computed in two stages:
\begin{enumerate}
  \item
    \emph{Partial} layout functions are determined for each axis component.
  \item
    Then, the full layout function is determined by summing the partial layout functions.
\end{enumerate}

To explain how the layout functions are actually computed we present a series of algorithms, starting from the simplest case and iteratively adding features until the complete\footnote{There are additional considerations in parallel that are not considered here, see \cref{sec:parallel_data_layouts}.} algorithm is reached.

\subsection{Intermediate algorithm 1: just linear axis trees}

\begin{figure}
  \centering

  \begin{subfigure}[t]{.3\textwidth}
    \centering
    \includegraphics{linear_axis_tree.pdf}
    \caption{The axis tree.}
  \end{subfigure}
  \begin{subfigure}[t]{.4\textwidth}
    \centering
    \includegraphics{linear_data_tree.pdf}
    \caption{The data layout.}
  \end{subfigure}

  \vspace{1em}

  \begin{subfigure}{\textwidth}
    \centering
    \begin{tblr}{|[1pt]c|[1pt]l|[1pt]l|[1pt]}
      \hline[1pt]
      \textbf{Path} & \SetCell{c}\textbf{Partial layout} & \SetCell{c}\textbf{Full layout} \\
      \hline[1pt]
      $\{a\}$ & $\textnormal{offset}(i_a) = 6 i_a$ & $\textnormal{offset}(i_a) = 6 i_a$ \\
      \hline[1pt]
      $\{a, b\}$ & $\textnormal{offset}(i_b) = 2 i_b$ & $\textnormal{offset}(i_a, i_b) = 6 i_a + 2 i_b$ \\
      \hline[1pt]
      $\{a, b, c\}$ & $\textnormal{offset}(i_c) = i_c$ & $\textnormal{offset}(i_a, i_b, i_c) = 6 i_a + 2 i_b + i_c$ \\
      \hline[1pt]
    \end{tblr}
    \caption{The layout functions.}
  \end{subfigure}
  \caption{
    The layout functions for a linear axis tree.
  }
  \label{fig:linear_axis_tree_layouts}
\end{figure}

As an introduction to computing layout functions we return to the simplest possible case: linear (and non-ragged) axis trees.
From the layout functions in \cref{fig:linear_axis_tree_layouts} we may make the following observations:
\begin{itemize}
  \item
    Partial layouts, expressed per axis, are always just the index multiplied by the step size.
    For example, the partial layout for axis $b$ is $2 i_b$ because the strides are over axis $c$, which has size 2.
  \item
    The innermost axis (here $c$) always has a trivial partial layout of just the index because it has a step size of 1.
  \item
    Full layout functions are the sum of the partial layout functions as the tree is descended.
    For instance, the full layout function at path $\{a,b\}$ is the sum of the partial layouts $6 i_a$ and $2 i_b$.

  \item
    Multiple full layout functions exist, one for each partial descent of the axis tree.
    If we only have indices $i_a$ and $i_b$, corresponding to the path $\{a,b\}$, for example, it is legitimate to compute the offset as $6i_a+2i_b$
    If instead the indices $i_b$ and $i_c$, corresponding to the path $\{b,c\}$, were provided then this would not be possible as it would be ambiguous whether $(a_0, i_b, i_c)$ or $(a_1, i_b, i_c)$ were meant.
\end{itemize}

\begin{algorithm}
  \begin{center}
    \begin{minipage}{.9\textwidth}
      \begin{pyalg2}
        def collect_layouts_linear(axis: Axis):
          layouts = {}?\label{code:layouts_dict_init}?

          # partial layout for the current axis
          if has_subaxis(axis):
            subaxis = get_subaxis(axis)
            step = get_axis_size(subaxis)
          else:
            step = 1
          layouts[axis] = AxisVar(axis.label) * step?\label{code:linear_layout}?

          # traverse subtree
          if has_subaxis(axis): 
            subaxis = get_subaxis(axis)
            sublayouts = collect_layouts_linear(subaxis)?\label{code:linear_sublayouts}?
            layouts.update(sublayouts)

          return layouts
      \end{pyalg2}
    \end{minipage}
  \end{center}

  \caption{
    Algorithm for computing the partial layout functions of a linear, non-ragged axis tree such as that shown in \cref{fig:linear_axis_tree_layouts}.
    The function is initially invoked by passing the root axis of the tree.
  }
  \label{alg:collect_layouts_linear}
\end{algorithm}

Pseudocode for determining the right partial layout functions for such an axis tree is shown in \cref{alg:collect_layouts_linear}.
The function \pycode{collect_layouts_linear} is invoked recursively on each axis of the tree, starting with the root.
For each axis the partial layout functions are collected into the \pycode{layouts} dictionary (line~\ref{code:layouts_dict_init}) using the current axis (\pycode{axis}) as the key.
The partial layouts are symbolic expressions of the form \pycode{AxisVar(axis.label) * step} (line \ref{code:linear_layout}), where \pycode{step} is an integer and \pycode{AxisVar()} is a symbolic index object representing indices such as $i_a$, $i_b$, and $i_c$ above.

\subsection{Intermediate algorithm 2: with multi-component axis trees}
\label{sec:layout_alg_multi_component}

\begin{figure}
  \centering
  \begin{subfigure}[t]{.35\textwidth}
    \centering
    \includegraphics{multi_component_axis_tree.pdf}
    \caption{The axis tree.}
  \end{subfigure}
  %
  \begin{subfigure}[t]{.45\textwidth}
    \centering
    \includegraphics{multi_component_data_tree.pdf}
    \caption{The data layout.}
  \end{subfigure}

  \vspace{1em}

  \begin{subfigure}{\textwidth}
    \centering
    \begin{tblr}{|[1pt]l|[1pt]l|[1pt]l|[1pt]}
      \hline[1pt]
      \textbf{Path} & \SetCell{c}\textbf{Partial layout} & \SetCell{c}\textbf{Full layout} \\
      \hline[1pt]
      $\{ a^x \}$ & $\textnormal{offset}(i_a) = 3 i_a$ & $\textnormal{offset}(i_a) = 3 i_a$ \\
      \hline[1pt]
      $\{ a^x, b \}$ & $\textnormal{offset}(i_b) = i_b$ & $\textnormal{offset}(i_a, i_b) = 3 i_a + i_b$ \\
      \hline[1pt]
      $\{ a^y \}$ & $\textnormal{offset}(i_a) = 2 i_a + 6$ & $\textnormal{offset}(i_a) = 2 i_a + 6$ \\
      \hline[1pt]
      $\{ a^y, c \}$ & $\textnormal{offset}(i_c) = i_c$ & $\textnormal{offset}(i_a, i_c) = 2 i_a + 6 + i_c$ \\
      \hline[1pt]
    \end{tblr}
    \caption{The layout functions.}
  \end{subfigure}

  \caption{The layout functions for a multi-component axis tree.}
  \label{fig:multi_component_axis_tree_layouts}
\end{figure}

When considering multi-component axis trees, a number of complications to the simple linear case are introduced.
First, there is no longer a single partial layout function per axis, instead there is a partial layout function \emph{per axis component}.
In addition, as well as weighted sums of indices, layout functions can now include \emph{integer offsets}.

These modifications are demonstrated in \cref{fig:multi_component_axis_tree_layouts}.
Unlike before, axis $a$ now has two components, $x$ and $y$, with each component having a separate child axis.
If we again inspect the computed layout functions we may make the following observations:
\begin{itemize}
  \item
    The paths are now partitioned by component: $a^x$ and $a^y$ are treated separately.
  \item
    The fact that all the $y$ component entries of axis $a$ come after $x$ is accounted for by the addition of an integer offset of 6, which is the size of the $x$ block of $a$.
  \item
    As before, the full layout functions are constructed by adding together the partial layout functions.
    However, in this case the fact that there are two distinct paths - $\{a^x, b\}$ and $\{a^y, c\}$ - through the axis tree means that not all partial functions are added together.
\end{itemize}

\begin{algorithm}
  \begin{flushright}
    \begin{minipage}{.96\textwidth}
      \begin{pyalg2*}{highlightlines={5,6,13,14,17}}
        def collect_layouts_multi_component(axis: Axis):
          layouts = {}

          # partial layouts for the current axis
          start = 0?\label{code:multi_component_start_var}?
          for component in axis.components:?\label{code:mc_loop1}?
            if has_subaxis(axis, component):
              subaxis = get_subaxis(axis, component)
              step = get_axis_size(subaxis)
            else:
              step = 1

            layouts[(axis, component)] = AxisVar(axis.label) * step + start?\label{code:mc_layout_store}?
            start += step

          # traverse subtree
          for component in axis.components:?\label{code:mc_loop2}?
            if has_subaxis(axis, component): 
              subaxis = get_subaxis(axis, component)
              sublayouts = collect_layouts_multi_component(subaxis)
              layouts.update(sublayouts)

          return layouts
      \end{pyalg2*}
    \end{minipage}
  \end{flushright}

  \caption{
    Algorithm for computing the partial layout functions of an axis tree where any of the axes may have multiple components.
    Some lines are highlighted in red to emphasise differences with \cref{alg:collect_layouts_linear}.
  }
  \label{alg:collect_layouts_multi_component}
\end{algorithm}

A modified version of \cref{alg:collect_layouts_linear} is shown in \cref{alg:collect_layouts_multi_component}.
The key differences are highlighted in red and include:
\begin{itemize}
  \item
    The axis tree traversal is now be over \emph{per component} subaxes.
    This means that additional loops over \pycode{axis.components} are required (lines~\ref{code:mc_loop1},~\ref{code:mc_loop2}).
  \item
    Since separate partial layouts are required for each axis component, the \pycode{layouts} dictionary now stores values per \pycode{(axis, component)} pair (line~\ref{code:mc_layout_store}).
  \item
    The integer offset between partial layouts is tracked by the \pycode{start} variable (line~\ref{code:multi_component_start_var}), and it is added to the partial layout function (line~\ref{code:mc_layout_store}).
\end{itemize}

\subsection{Complete algorithm: with ragged axis trees}
\label{sec:layout_alg_ragged}

For ragged axis trees a new approach is needed.
Since the strides between entries are no longer uniform it is no longer sufficient for layout functions to be expressed as a simple affine function, and instead the offsets must be tabulated into some lookup table in an analogous process to the approach taken by \ccode{PetscSections} (\cref{alg:petsc_section_tabulate}).

As an example, consider the axis tree shown in \cref{fig:ragged_axis_tree_layout}.
Though linear, the innermost axis ($c$) of the tree has a non-uniform size.
Compared with previous, non-ragged axis trees, this has a number of implications when computing the layouts:
\begin{itemize}
  \item
    Due to the non-uniform size of axis $c$, both axes $a$ and $b$ have non-uniform strides.
    The offsets must therefore be tabulated into \emph{offset arrays}.

  \item
    Offset arrays are implemented as \pycode{Dats} with linear axis trees.

  \item
    The offset arrays at a minimum must include the current axis in their own axis tree.
    Axis $a$, for example, has partial layout function $\pycode{[0,1][?$i_a$?]}$, meaning that it is a 1-dimensional array with an axis tree consisting of only axis $a$.
    Similar to the size array in \cref{sec:ragged_axis_trees}, this partial layout is (trivially) equivalent to the table:

    \begin{center}
      \begin{tblr}{|[1pt] X[1.2cm,c] |[1pt] c |[1pt]}
        \hline[1pt]
        \boldmath$i_a$ & \boldmath$\textnormal{\bf offset}(i_a)$ \\
        \hline[1pt]
        0 & 0 \\
        \hline[1pt]
        1 & 1 \\
        \hline[1pt]
      \end{tblr}
    \end{center}

  \item
    Since the stride of axis $b$ depends on the size of axis $c$, which in turn depends on the index $i_a$, the offset array for axis $b$ depends on \emph{both} $i_a$ and $i_b$.
    By depending on axis $a$ as well as itself, we say that axis $a$ is a \textit{dependent} axis for axis $b$.
\end{itemize}

\begin{figure}
  \centering
  \begin{subfigure}{.4\textwidth}
    \centering
    \includegraphics{ragged_axis_tree.pdf}
    \caption{The axis tree.}
  \end{subfigure}
  \begin{subfigure}{.45\textwidth}
    \centering
    \includegraphics{ragged_data_tree.pdf}
    \caption{The data layout.}
  \end{subfigure}

  \vspace{1em}

  \begin{subfigure}{\textwidth}
    \centering
    \begin{tblr}{|[1pt]c|[1pt]l|[1pt]l|[1pt]}
      \hline[1pt]
      \textbf{Path} & \SetCell{c}\textbf{Partial layout} & \SetCell{c}\textbf{Full layout} \\
      \hline[1pt]
      $\{a\}$ & $\textnormal{offset}(i_a) = \pycode{[0,1][?$i_a$?]}$ & $\textnormal{offset}(i_a) = \pycode{[0,1][?$i_a$?]}$ \\
      \hline[1pt]
      $\{a,b\}$
      &
      $\begin{aligned}[t]
        &\textnormal{offset}(i_a, i_b) = \\
        &\quad \pycode{[[0,1],[0,2]][?$i_a$?,?$i_b$?]}
      \end{aligned}$
      % & $\textnormal{offset}(i_a, i_b) = \pycode{[[0,1],[0,2]][?$i_a$?,?$i_b$?]}$
      & $\begin{aligned}[t]
        &\textnormal{offset}(i_a, i_b) = \\
        &\quad \pycode{[0,1][?$i_a$?]} + \pycode{[[0,1],[0,2]][?$i_a$?,?$i_b$?]} \\
      \end{aligned}$ \\
      \hline[1pt]
      $\{a,b,c\}$
      & $\textnormal{offset}(i_c) = i_c$
      & $\begin{aligned}[t]
        &\textnormal{offset}(i_a,i_b,i_c) = \\
        &\quad \pycode{[0,1][?$i_a$?]} + \pycode{[[0,1],[0,2]][?$i_a$?,?$i_b$?]} + i_c
      \end{aligned}$ \\
      \hline[1pt]
    \end{tblr}
    \caption{The layout functions.}
  \end{subfigure}

  \caption{The layout functions for a ragged axis tree.}
  \label{fig:ragged_axis_tree_layout}
\end{figure}


\begin{algorithm}
  \caption{
    Algorithm for computing the layout functions of an axis tree where any of the contained axes may be ragged.
    Key differences with \cref{alg:collect_layouts_multi_component} are highlighted in red.
  }

  \begin{center}
    \begin{minipage}{.9\textwidth}
      \begin{pyalg2*}{highlightlines={7,12}}
        def collect_layouts_ragged(axis: Axis):
          layouts = {}

          # partial layouts for the current axis
          start = 0
          for component in axis.components:
            if has_constant_step(axis, component):?\label{code:has_constant_step}?
              # affine case, see ?\cref{alg:collect_layouts_multi_component}?
              ...
            else:
              # axis component has a ragged subaxis
              layouts[(axis, component)] = tabulate_offsets(...)?\label{code:call_tabulate_offsets}?

          # traverse subtree
          for component in axis.components:
            if has_subaxis(axis, component): 
              subaxis = get_subaxis(axis, component)
              sublayouts = collect_layouts_ragged(subaxis)
              layouts.update(sublayouts)

          return layouts
      \end{pyalg2*}
    \end{minipage}
  \end{center}
  \label{alg:collect_layouts_ragged}
\end{algorithm}

To determine the partial layouts for ragged axis trees we must modify \cref{alg:collect_layouts_multi_component} to account for the case where axis component step size is non-uniform.
The new version is shown in \cref{alg:collect_layouts_ragged}.
In it, none of the code from the earlier algorithms is changed, but an additional branch is added to handle the new case (line~\ref{code:has_constant_step}).
If the axis component's steps are found to be non-uniform then, instead of making an affine layout function involving \pycode{AxisVars}, the function \pycode{tabulate_offsets()} is called (line~\ref{code:call_tabulate_offsets}), returning an appropriate offset array to be used as the partial layout instead.

\begin{algorithm}
  \caption{
    Code that tabulates the offsets for an axis with non-uniform strides.
  }

  \begin{flushright}
    \begin{minipage}{.96\textwidth}
      \begin{pyalg2}
        def tabulate_offsets(axis: Axis, component: AxisComponent):
          dependent_axes = collect_dependent_axes(axis, component)?\label{code:collect_dependent_axes}?

          # make the offset array
          offset_axes = AxisTree.from_iterable([dependent_axes, axis])?\label{code:offset_axes}?
          offset_dat = Dat(offset_axes?\label{code:offset_dat})?

          # loop over dependent axes
          for dep_indices in dependent_axes.iter():?\label{code:loop_dependent_axes}?
            # reset the offset counter
            offset = 0?\label{code:reset_offset}?
            # loop over the current axis
            for current_index in axis.iter():?\label{code:loop_current_axis}?
              # store the current offset in offset_dat
              offset_dat[*dep_indices, current_index] = offset
              offset += step_size(axis, component)

          return offset_dat
      \end{pyalg2}
    \end{minipage}
  \end{flushright}
  \label{alg:tabulate_offsets}
\end{algorithm}

The behaviour of \pycode{tabulate_offsets()} is shown in \cref{alg:tabulate_offsets}.
First, an empty \pycode{Dat} (\pycode{offset_dat}) is constructed to hold the tabulated offsets (line~\ref{code:offset_dat}).
Its axis tree (line~\ref{code:offset_axes}) is linear and built from the current axis as well as any dependent axes (line~\ref{code:collect_dependent_axes}).
Then, in a process very similar to what a \ccode{PetscSection} does during \ccode{PetscSectionSetUp()} (\cref{alg:petsc_section_tabulate}), both the dependent axes and current axis are looped over, incrementing and storing the offset in \pycode{offset_dat}.

The instructions \pycode{dependent_axes.iter()} (line~\ref{code:loop_dependent_axes}) and \pycode{axis.iter()} (line~\ref{code:loop_current_axis}) return iterators\footnotemark that, when looped over, yield a sequence of multi-indices that are used to index into \pycode{offset_dat}.

\footnotetext{
  At present in \pyop3 these iterators are implemented in Python, making them very slow to execute for large axis trees.
  Accelerating this algorithm through either rewriting it in a faster language like C or using code generation is a high priority.
}

Note that at present it is not possible to have multi-component axes if one of the components is ragged.
Resolving this requires that the \pycode{start} variable in \cref{alg:collect_layouts_ragged} support adding offset arrays as well as integers.

\section{Outlook}

% axis trees are a novel abstraction for representing data layouts

% they preserve the hierarchical structure of mesh data

% as we shall see in \cref{chapter:firedrake}, the sorts of complex data layouts that occur in finite element calculations are straightforward to describe using axis trees

% keeping extra topological information allows for an expressive interface


\end{document}
