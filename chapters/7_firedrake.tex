\documentclass[thesis]{subfiles}

\begin{document}

\chapter{Firedrake integration}

We will demonstrate the functionality of \pyop3 by integrating it with the finite element framework Firedrake.
The intent is for \pyop3 to in the near future serve as a replacement for \pyop2 inside Firedrake.

In order to demonstrate the steps necessary for this integration to work, we will work through solving the Stokes problem described in \cref{sec:stokes_equations}.

\section{The Firedrake framework}

\begin{figure}
  \includegraphics{firedrake_structure.pdf}
  \caption{TODO}
  \label{fig:firedrake_structure}
\end{figure}

% introduce UFL

\section{Solving the Stokes equations}

\begin{listing}
  \centering
  \begin{minipage}{.9\textwidth}
    \inputminted{python}{./scripts/stokes_demo.py}
  \end{minipage}
  \caption{
    Firedrake code for setting up and solving the Stokes problem set up in \cref{sec:stokes_equations}.
  }
  \label{listing:stokes_demo}
\end{listing}

%1. mesh (arbitrary dimension and cell type - sort of)
% 2. function spaces
% 3. functions and matrices? sparsities?

% what is Firedrake! should go here not in background.

% mesh.points, V.axes
% mesh numbering is handled here via an indexing operation - pyop3 does not know about it.
% DatView for viewing function components
% mixed dats are now contiguous

\section{Packing}

% these are bad to include because it's bad design.
\subsection{Tensor product cells}

\subsection{Hexahedral elements}

%\subsection{Fieldsplit} ???

%\subsection{Performance} ???
% all about minimising memory bandwidth, aim for roofline peak at all points
% I need to implement the map compression algorithm (generically)

\end{document}
