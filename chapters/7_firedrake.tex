\documentclass[thesis]{subfiles}

\begin{document}

\chapter{Firedrake integration}
\label{chapter:firedrake}

One of the main advantages to a framework like \pyop3 is that, by automating a process typically done by hand, it allows for higher level abstractions to be built on top of it.
In this chapter we will demonstrate this by integrating \pyop3 into the finite element framework Firedrake~\cite{FiredrakeUserManual}.

\begin{figure}
  \includegraphics{firedrake_structure_old.pdf}
  \caption{
    The existing structure of the Firedrake finite element framework.
    Packages shown in red are managed by Firedrake whereas the grey packages are key external dependencies.
    Users typically express computations as Python scripts using both UFL and Firedrake; this is represented by the blue `User input' box.
    The direction of the arrows indicate a caller-callee relationship.
    For instance Firedrake calls \pyop2 but not vice versa.
    The bidirectional arrow between Firedrake and PETSc indicates that they each call the other, which for PETSc is via callback functions.
  }
  \label{fig:firedrake_structure_old}
\end{figure}

\section{Firedrake}

Firedrake is a Python framework for automating the solving of PDEs using the finite element method.
% productive, high level, near mathematical
% parallel by default
% uses PyOP2

The framework consists of a number of interacting software components.
Shown in \cref{fig:firedrake_structure_old} the core components, in addition to Firedrake and \pyop2, include: UFL (Unified Form Language)~\cite{alnaesUnifiedFormLanguage2014a}, TSFC (Two-Stage Form Compiler)~\cite{homolyaTSFCStructurePreservingForm2018}, loopy, and PETSc.
A number of packages not relevant to \pyop2/\pyop3 have been omitted for simplicity.

\begin{listing}
  \centering
  \begin{minipage}{.9\textwidth}
    \begin{pyalg2}
      from firedrake import *

      # initialise the mesh and function spaces
      mesh = UnitSquareMesh(10, 10)?\label{code:stokes_init_mesh}?
      V = VectorFunctionSpace(mesh, "P", 3)?\label{code:stokes_spaces_begin}?
      Q = FunctionSpace(mesh, "DP", 2)
      W = MixedFunctionSpace([V, Q])?\label{code:stokes_spaces_end}?

      u, p = TrialFunctions(W)
      v, q = TestFunctions(W)

      # viscosity
      nu = Constant(666)

      # define lhs and rhs
      a = nu * inner(grad(u), grad(v)) * dx - p * div(v) * dx + q * div(u) * dx?\label{code:stokes_bilinear_form}?
      L = Cofunction(W.dual())  # zero rhs?\label{code:stokes_linear_form}?

      # construct the boundary condition
      g = Constant([666, 666])
      bc = DirichletBC(W.sub(0), g, "on_boundary")

      # assemble and solve the problem
      solution = Function(W)
      solve(a == L, solution, bcs=[bc])?\label{code:stokes_solve}?
    \end{pyalg2}
  \end{minipage}
  \caption{
    Firedrake code for setting up and solving the Stokes problem from \cref{sec:stokes_equations}.
  }
  \label{listing:stokes_demo}
\end{listing}

The way in which these packages interact is best shown by way of an example.
\Cref{listing:stokes_demo} shows a `typical' Firedrake script for solving the Stokes equations with the formulation from \cref{sec:stokes_equations}.
It consists of the following steps:
\begin{enumerate}
  \item
    The mesh (l.~\ref{code:stokes_init_mesh}) and function spaces (l.~\ref{code:stokes_spaces_begin}-\ref{code:stokes_spaces_end}) are created.
    Internally the mesh is stored as a PETSc DMPlex (\cref{sec:foundations_dmplex}), which is distributed across processes with appropriate point star forests.
    The function spaces then associate DoFs with topological elements of the mesh according to the provided finite element definition.
    In this case the finite elements are specified by the arguments `\pycode{"P"}' and `\pycode{3}', and `\pycode{"DP"}' and `\pycode{2}' to `\pycode{VectorFunctionSpace}' and `\pycode{FunctionSpace}' respectively.
    These correspond to the $[P_3]^2$ and $P_2^{\textnormal{disc}}$ elements that together form a Scott-Vogelius element.

  \item
    The linear (l.~\ref{code:stokes_linear_form}) and bilinear (l.~\ref{code:stokes_bilinear_form}) forms are declared symbolically with UFL.
    Notice how the definition of the bilinear form in particular strongly resembles the mathematics of the original weak form (\cref{eq:weak_stokes}).
    The linear form `\pycode{L}' is known to be trivially zero and so a pre-assembled cofunction is used.

  \item
    Lastly, the PDE is solved (l.~\ref{code:stokes_solve}) with appropriate boundary conditions.
    To solve it, the bilinear (\pycode{a}) and linear (\pycode{L}) forms are first assembled into a matrix and vector respectively.
    Since the linear form for this PDE is known zero its assembly is trivial.
    Then, having assembled the linear system, it is then solved by PETSc with its wealth of different solver types, putting the solution for both the velocity and pressure together into the mixed function \pycode{solution}.

    To accomodate for the boundary condition, values in the assembled matrix and vector corresponding to the constrained DoFs are dropped.
\end{enumerate}

\subsection{\pyop2's contribution}

\pyop2 is a crucial component to this process.
To start with, \emph{all of Firedrake's global data structures wrap \pyop2 ones}.
For instance \pycode{Functions} and \pycode{Cofunctions} wrap \pyop2 \pycode{Dats}, and assembled bilinear forms, represented by Firedrake \pycode{Matrix} objects, wrap \pyop2 \pycode{Mats}.
The \pycode{DataSets} that act as data layout descriptors for the \pyop2 objects are drawn from the \emph{nodes} of the function spaces.

Further, \emph{\pyop2 is responsible for finite element assembly}.
The UFL forms are converted by TSFC into local kernels, equivalently stencil payloads, expressed as a loopy \pycode{LoopKernel} object.
\pyop2 then handles the outer loop over mesh entities (here the cells), evaluating the local kernel for each.
A parallel loop would be constructed with the following structure:
\begin{pyinline}
  par_loop(local_kernel,                         # stencil payload
           mesh.cell_set,                        # iteration set
           dat0(READ, cell2nodes)                # arguments
           mat0(INC, (cell2nodes, cell2nodes)))  #   "
\end{pyinline}
Notice that since \pycode{Dats} and \pycode{Mats} store data relative to nodes of the function spaces, as opposed to topological entities of the mesh, the indirection maps map from the iteration set to nodes of the function space.

In order to integrate \pyop3 with Firedrake, replacing \pyop2, we must clearly replace these two elements of \pyop2 with their \pyop3 equivalents: nodal \pycode{DataSets} must be replaced with \pycode{AxisTrees} and \pyop2 parallel loops must be replaced by \pyop3 loop expressions.

\begin{figure}
  \centering
  \includegraphics{scott_vogelius_element_dof_layout.pdf}
  \caption{
    Appropriate global data layout for a function space with a Scott-Vogelius element.
  }
  \label{fig:firedrake_data_tree}
\end{figure}

\section{Representing global data with axis trees}

\begin{figure}
  \centering
  \begin{subfigure}{\textwidth}
    \centering
    \includegraphics{scott_vogelius_element_dof_layout_plain.pdf}
    \caption{
      The data layout.
    }
    \label{fig:firedrake_data_tree_plain}
  \end{subfigure}

  \vspace{1em}

  \begin{subfigure}{\textwidth}
    \centering
    \includegraphics{scott_vogelius_space_axis_tree_plain.pdf}
    \caption{
      The axis tree.
      The variables \pycode{nnodes_V} and \pycode{nnodes_Q} are \pyop3 \pycode{Dats}, indicating that the data layout is ragged.
    }
    \label{fig:firedrake_axis_tree_plain}
  \end{subfigure}
  \caption{
    The initial ragged data layout for a function space.
  }
  \label{fig:firedrake_plain}
\end{figure}

\begin{figure}
  \centering
  \includegraphics[scale=.8]{scott_vogelius_space_axis_tree.pdf}
  \caption{
    The (indexed) axis tree generated after indexing the ragged axis tree in \cref{fig:firedrake_axis_tree_plain}.
    The mesh entities are distinguishable whilst remaining interleaved.
  }
  \label{fig:firedrake_axis_tree}
\end{figure}

\begin{figure}
  \centering
  \includegraphics[scale=.7]{mesh_renumbering_transform.pdf}
  \caption{
    The indexing transformation taking a ragged axis tree for a function space to one where the different topological entities are distinguishable.
    The arrays \pycode{[0,3,...]}, \pycode{[2,...]}, and \pycode{[1,4,...]} represent the indices of the cells, edges, and vertices respectively.
  }
  \label{fig:mesh_renumbering_transform}
\end{figure}

In many respects the fundamental behaviour of global data structures is unchanged between \pyop2 and \pyop3.
Both use \numpy arrays and PETSc matrices to store vectors and matrices as \pycode{Dats} and \pycode{Mats}, and \pycode{Dats} still use star forests to exchange ghost data.
The core difference between the two libraries is in the addition of axis trees, which replace \pyop2 \pycode{DataSets}.

In order to store function space data in \pyop3 \pycode{Dats} and \pycode{Mats}, appropriate axis trees must first be constructed to prescribe the data layouts.
If we consider the Scott-Vogelius example that we have been using, we would expect, having applied a mesh renumbering, to have a data layout like that shown in \cref{fig:firedrake_data_tree}.
The layout naturally decomposes into multiple separate axes: the function spaces (\pycode{"field"}), the points of the mesh (\pycode{"mesh"}), the per-entity nodes (\pycode{"node"}), and any vector nature of the nodes (\pycode{"component"}).

Having this hierarchical structure is convenient.
Views of particular function spaces or vector components are straightforward to obtain via indexing and the axes may be freely permuted to give data layouts with the same semantics but different data locality properties.

\subsection{Renumbering the mesh}

Given this decomposition it would appear simple to construct an appropriate axis tree to describe the layout.
However, there is a complication: axis components are stored contiguously.
Constructing a \pycode{"mesh"} axis naively as
\begin{pyinline}
  mesh_axis = Axis({"cell": ncells, "vertex": nvertices, "edge": nedges},
                   "mesh")
\end{pyinline}
leads to a non-interleaved layout with all cell DoFs preceding all vertex DoFs preceding all edge DoFs.
This is clearly incorrect.

To fix this we take a two step approach.
First, instead of constructing a multi-component \pycode{"mesh"} axis from the start, we instead construct a single component axis where the component represents all points in the mesh:
\begin{pyinline}
  mesh_axis = Axis({"point": npoints}, "mesh")
\end{pyinline}
We can then construct an axis tree with the correct DoF ordering, albeit without being able to distinguish the different mesh entities.
This is demonstrated in \cref{fig:firedrake_plain}.
Observe that, since mesh entities are indistinguishable, the number of nodes per-entity is no longer fixed and so we have a ragged axis tree.

Given this axis tree, the correct entity information may be recovered via \emph{indexing}.
To do so we index the \pycode{"mesh"} axis with the slice:
\begin{pyinline}
  Slice("mesh", [Subset("point", icell, label="cell"),
                 Subset("point", iedge, label="edge"),
                 Subset("point", ivert, label="vertex")])
\end{pyinline}
Where \pycode{icell}, \pycode{iedge} and \pycode{ivert} are integer arrays containing the cell, edge, and vertex indices respectively.
Indexing the \pycode{"mesh"} axis with this slice yields a new multi-component axis where the different entity types may be distinguished and separately addressed.
This transformation is shown in \cref{fig:mesh_renumbering_transform}.
The result of applying this index to the single-component, ragged Scott-Vogelius axis tree (\cref{fig:firedrake_axis_tree_plain}) is shown in \cref{fig:firedrake_axis_tree}.

By indexing the \pycode{"mesh"} axis in this way we make it such that the resulting axis tree has `dual' representations - as flat points or as a collection of distinct mesh entities - which is exactly the same approach as DMPlex (see discussion in \cref{sec:foundations_dmplex}).
\pyop3, with its indexable axis trees, is therefore in some sense \emph{a generalisation of the abstractions provided by DMPlex}.

\subsubsection{Computing the renumbering}

\begin{algorithm}
  \caption{
    Algorithm that computes the point renumbering to improve data locality in finite element assembly loops.
    Owned points are always stored before ghost points.
  }

  \begin{center}
    \begin{minipage}{.9\textwidth}
      \begin{pyalg2}
        def plex_renumbering(plex: PETSc.DMPlex, cell_order: np.ndarray[int]):
          # bookkeeping to track result
          renumbering = np.empty(num_points(plex))
          owned_ptr = 0
          ghost_ptr = num_owned_points(plex)
          seen_points = set()

          # loop over cells
          c_start, c_end = plex.getHeightStratum(0)
          for cell in range(c_start, c_end):
            # renumber the cell
            cell_renum = cell_order[cell]

            # pack points in the cell closure together, skipping already
            # seen points and respecting the owned/ghost partition
            closure = get_closure(plex, cell_renum)
            for pt in closure:
              if pt in seen_points:
                continue
              else:
                if is_ghost(pt):
                  renumbering[ghost_ptr] = pt
                  ghost_ptr += 1
                else:
                  renumbering[owned_ptr] = pt
                  owned_ptr += 1
                seen_points.add(pt)
        return renumbering
      \end{pyalg2}
    \end{minipage}
  \end{center}
  \label{alg:plex_renumbering}
\end{algorithm}

\Cref{alg:plex_renumbering} is used to compute the point renumbering used to interleave DoFs for the function spaces.
One loops over the cells of the mesh and packs entities in their closure together.
This means that DoFs associated with edges and vertices will be `close' in memory to their incident cell, improving performance for finite element assembly loops where one accesses all DoFs in a cell's closure together.
In addition to this, to further improve locality the outer loop over cells loop is ordered with an reverse Cuthill-McKee numbering~\cite{cuthillReducingBandwidthSparse1969} (\pycode{cell_order} in \cref{alg:plex_renumbering}).
This means that nearby cell closures are kept close in memory, improving data reuse for shared entities between adjacent cells.

The algorithm is very similar to that already used by Firedrake and \pyop2 (alg. 3 in~\cite{langeEfficientMeshManagement2016}) except, due to reasons explained in \cref{sec:communication_optimisations}, the \ownediter entity class is dropped.

\section{Finite element assembly}

\begin{figure}
  \centering
  \includegraphics{scott_vogelius_index_tree.pdf}
  \caption{
    An index tree appropriate for transforming an axis tree for a Scott-Vogelius function space (\cref{fig:firedrake_axis_tree}) into a cell-local view of the cell and its closure.
  }
  \label{fig:firedrake_index_tree}
\end{figure}

Now that we have the required global data structures we may now consider the finite element assembly process.
For the Stokes problem above, the assembly loop created for the LHS matrix is equivalent to the following \pyop3 code:
\begin{pyinline}
  loop(cell := mesh.cells.owned.index(),
       kernel(viscosity_dat[closure(cell)],
              output_mat[closure(cell), closure(cell)]))
\end{pyinline}
Which expresses the operation that one should loop over the locally-owned cells in the mesh (\pycode{mesh.cells.owned}) and execute the local kernel (\pycode{kernel}) for each one.
\pycode{kernel} takes two arguments, with access descriptors \pycode{READ} and \pycode{INC}, representing the viscosity (a \pycode{Dat}) and output matrix (a \pycode{Mat}), each restricted, via indexing, to a view of the DoFs supported on the cell.
\pycode{viscosity_dat}, having only a single associated axis tree, takes only a single index, whilst \pycode{output_mat} requires two.

Note that, since the axis trees for \pycode{viscosity_dat} and \pycode{output_mat} are non-flat, simply indexing with `\pycode{closure(cell)}' is syntactic sugar.
Internally it is translated into an index tree appropriate for indexing all of the axes.
This is shown in \cref{fig:firedrake_index_tree}.

With such a loop expression, \pyop3 is able to generate and execute code in parallel using the processes established in \cref{chapter:pyop3}.

\subsection{More integral types}

\begin{figure}
  \centering
  \includegraphics{interior_facet_support.pdf}
  \caption{
    Stencil for an interior facet integral.
    The facet currently being iterated over is marked with the star.
  }
  \label{fig:interior_facet_support}
\end{figure}

As well as cells, Firedrake also supports integrating over the (interior and exterior) facets of a mesh.
In these cases the data that is packed and passed to the local kernel consists of the `macro' cell formed from the cells in the support of the facet (e.g. \cref{fig:interior_facet_support}).

With \pyop3 it is straightforward to express assembly loops for facet integrals using map composition (\cref{sec:indexing_map_composition}):
\begin{pyinline}
  facet = mesh.interior_facets.owned.index()
  packed = dat[closure(support(facet))]
\end{pyinline}
This works because the support of an interior facet ($\support(f)$) consists of the two cells it touches, and the closure of those gives the correct macro cell.

\subsection{Handling orientations}

One of the big advantages of \pyop3 over \pyop2 is in its ability to express transformations that cannot be memoized.
An example of this is the way in which element orientations are handled.

% the problem
% -----------
% adjacent cells may disagree on the orientation of a shared facet or edge
% the local kernel is then packed incorrectly, so the computation is incorrect
% this is only a problem for some element types, the error may be avoided for simplices of any dimension and a non-mobius strip solution exists for quads by choosing a global numbering that dictates a particular entity orientation
% in particular this trick cannot be used for hexahedral elements

\begin{figure}
  \centering
  \begin{subfigure}{0.45\textwidth}
    \centering
    \includegraphics{lagrange_element_3_default.pdf}
  \end{subfigure}
  %
  \begin{subfigure}{0.45\textwidth}
    \centering
    \includegraphics{lagrange_element_3_flip.pdf}
  \end{subfigure}
  \caption{Reference $P_3$ (Lagrange, degree 3) element (left), and with an edge flipped (right).}
  \label{fig:element_orientation_permute}
\end{figure}

\begin{figure}
  \centering
  \begin{subfigure}{0.45\textwidth}
    \centering
    \includegraphics{raviart_thomas_default.pdf}
  \end{subfigure}
  %
  \begin{subfigure}{0.45\textwidth}
    \centering
    \includegraphics{raviart_thomas_flip.pdf}
  \end{subfigure}
  \caption{Reference Raviart-Thomas element (left), and with an edge flipped (right).}
  \label{fig:element_orientation_flip}
\end{figure}

The general solution to this problem is to make the packing code aware of the orientation of the different entities.
This is relatively straightforward for function spaces with point evaluation DoFs (\cref{fig:element_orientation_permute}) because the necessary transformation is simply a permutation and so can embedded in the cell-node maps.
This is what Firedrake does currently.

However, this approach fails to work for function spaces with vector-valued DoFs, such as \hdiv and \hcurl, since the transformation can include a scaling.
See, for example, the case shown in \cref{fig:element_orientation_flip}.
In order for the edge DoF to be packed correctly for both cells, one of them needs to scale its value by $-1$.
In general this transformation can be a matrix.
The software package Basix~\cite{scroggsBasixRuntimeFinite2022,scroggsConstructionArbitraryOrder2021} is capable of generating these transformation matrices.

% how things are done in pyop3
% ----------------------------
With \pyop3 we can express the permutation transformation as a symbolic indexing operation.
% this means we can generate code for it (do at runtime), step towards more complex runtime transformations
% compared to pyop2 where it is memoized in advance

% consider simplest possible, P3 and disregard non-edge DoFs
To explain how this is done we consider the edge flip shown in \cref{fig:element_orientation_permute}.
For simplicity consider a one-cell mesh and we omit cell and vertex DoFs.

As well as the axis tree for the function space we now have two extra pieces of information:
\begin{itemize}
  \item
    The orientation of the edges, stored as an array of integers per cell closure.
    For this single-cell-mesh case the orientation array of the flipped triangle is $\{ c_0 \to \pycode{[0, 1, 0]} \}$, or, equivalently, \pycode{orientations = [[0, 1, 0]]}.
    An orientation of 0 is considered `default', and 1 means that it is flipped.

  \item
    The DoF permutation for a given orientation.
    Since there are only two ways that an edge can be oriented this is therefore stored as $\{ 0 \to \pycode{[0, 1]},\ 1 \to \pycode{[1, 0]} \}$, or in code as \pycode{perms = [[0, 1], [1, 0]]}.
\end{itemize}

Now, in order to correctly pack the DoFs for the kernel the following permutation of the DoFs axis is required:
\begin{pyinline}
  cell = 0
  packed = dat[closure(cell)][:, [[0, 1], [1, 0], [0, 1]]]
\end{pyinline}
which is equivalent to following transformation:
\begin{center}
  \includegraphics{element_orientation_permute_data.pdf}
\end{center}

The array \pycode{[[0, 1], [1, 0], [0, 1]]} clearly comes from the permutation array above, so we may transform the code to:
\begin{pyinline}
  cell = 0
  packed = dat[closure(cell)][:, perms[[0, 1, 0]]]
\end{pyinline}
The final array \pycode{[0, 1, 0]} is simply the orientations array, giving the final transformation as:
\begin{pyinline}
  cell = 0
  packed = dat[closure(cell)][:, perms[orientations[cell]]]
\end{pyinline}

With this we now have a fully symbolic, indexing approach to permuting the DoFs for any function space.
Having a symbolic representation of this transformation means that one can generate code for the transformation, instead of memoizing the modified maps in advance.
Whilst not strictly necessary for the permutation case described here it is essential for more complex transformations that this be expressible.
Extending Firedrake and \pyop3 to support general transformations is future work.

\end{document}
