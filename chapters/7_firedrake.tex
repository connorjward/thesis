\documentclass[thesis]{subfiles}

\begin{document}

\chapter{Firedrake integration}
\label{chapter:firedrake}

One of the main advantages to a framework like \pyop3 is that, by automating a process typically done by hand, it allows for higher level abstractions to be built on top of it.
In this chapter we will demonstrate this by integrating \pyop3 into the finite element framework Firedrake~\cite{FiredrakeUserManual}.

\section{The Firedrake framework}

% firedrake motivation
Firedrake is a framework for automating the solutions to the finite element method.
% productive, high level, near mathematical

\begin{figure}
  \includegraphics{firedrake_structure_old.pdf}
  \caption{
    The existing structure of the Firedrake finite element framework.
  }
  \label{fig:firedrake_structure_old}
\end{figure}

Firedrake is composed of a number of interacting software components (\cref{fig:firedrake_structure_old}):
At the top level, users write Python scripts expressing variational forms in Unified Form Language (UFL)~\cite{alnaesUnifiedFormLanguage2014a} and solving them via the Firedrake API.

\paragraph{Given a variational form expressed in UFL, Firedrake can then assemble the required matrices and vectors.}

\paragraph{Lastly, given the right assembled objects, Firedrake coordinates the (non-) linear solve and possible preconditioner applications.}
In order to do this Firedrake calls PETSc~\cite{petsc-user-ref,petsc-web-page,petsc-efficient}, a library with a wealth of different solvers and preconditioners.
Like Firedrake, PETSc places an emphasis on composability and rapid prototyping; it is straightforward to make fine-grained changes to the choice of solvers and preconditioners simply by passing a dictionary of options to Firedrake's \pycode{solve()} function.

\vspace{\baselineskip}

\pyop2 is clearly central to Firedrake's functionality and it is this component that \pyop3 is intended to be a drop-in replacement for.

\subsection{Solving a finite element problem with Firedrake}

\begin{listing}
  \centering
  \begin{minipage}{.9\textwidth}
    \begin{pyalg2}
      from firedrake import *

      # initialise the mesh and function spaces
      mesh = UnitSquareMesh(10, 10)
      V = VectorFunctionSpace(mesh, "P", 3)
      Q = FunctionSpace(mesh, "DP", 2)
      W = MixedFunctionSpace([V, Q])

      u, p = TrialFunctions(W)
      v, q = TestFunctions(W)

      # viscosity
      nu = Constant(666)

      # define lhs and rhs
      a = nu * inner(grad(u), grad(v)) * dx - p * div(v) * dx + q * div(u) * dx
      L = Cofunction(W.dual())  # zero rhs

      # construct the boundary condition
      g = Constant([666, 666])
      bc = DirichletBC(W.sub(0), g, "on_boundary")

      # assemble and solve the problem
      solution = Function(W)
      solve(a == L, solution, bcs=[bc])
    \end{pyalg2}
  \end{minipage}
  \caption{
    Firedrake code for setting up and solving the Stokes problem set up in \cref{sec:stokes_equations}.
  }
  \label{listing:stokes_demo}
\end{listing}

An example Firedrake script for solving the Stokes equations (\cref{sec:stokes_equations}) is shown in \cref{listing:stokes_demo}.

% ...

% NOTE: do not talk about pyop3, that comes after - just explain the script
\paragraph{Create the mesh}
% arbitrary dimension and cell type (sort of)

\paragraph{Build the function spaces}

a function space associates a finite element with a mesh

here we create a mixed FE space with SV element pair

\paragraph{Construct the forms}

These are exactly the same as the equations in \cref{???}

\paragraph{Solve the variational problem}

Call \pycode{solve()}. This:

The bilinear form \pycode{a} is assembled...
assembles the LHS matrix and RHS vector

solves the linear system using PETSc - it is not shown here but can pass solver parameters... - solution for both u and p fields stored in \pycode{solution}

\section{Representing global data}

% TODO: Perhaps include sv_data_tree.pdf here

\begin{figure}
  \centering
  \includegraphics[scale=.8]{scott_vogelius_space_axis_tree.pdf}
  \caption{
    TODO.
  }
  \label{fig:firedrake_axis_tree}
\end{figure}

As demonstrated in \cref{chapter:axis_trees}, data layouts for function spaces are well represented as axis trees.
For the Stokes problem above, the axis tree created for the mixed function space \pycode{W} is shown in \cref{fig:firedrake_axis_tree}.
The \pycode{"mesh"} axis (appearing twice) comes from the mesh and the other axes (\pycode{"space"}, \pycode{"node"}, and \pycode{"component"}) all come from the function space definition.
As described in \cref{sec:foundations_dmplex}, the mesh carries a point star forest and this is used to build the correct DoF star forest for exchanging ghost data (\cref{sec:parallel_star_forests}).

At present the ordering of axes for function spaces is fixed, but it would be straightforward to extend Firedrake to support alternative data layouts (see \cref{sec:axis_tree_alternative_layouts}, \cref{sec:indexing_data_layout_transformations}).

\subsection{Mesh renumbering}
\label{sec:firedrake_mesh_renumbering}

An observant reader may have noticed that in the axis tree just introduced (\cref{fig:firedrake_axis_tree}) no interleaving of DoFs appears to be happening, which would be problematic for reasons of data locality (\cref{sec:intro_mesh_numbering}).
This is in fact not the case: we have skipped a step in our explanation.

\begin{figure}
  \centering
  \includegraphics[scale=.7]{mesh_renumbering_transform.pdf}
  \caption{TODO}
  \label{fig:mesh_renumbering_transform}
\end{figure}

When the axis tree for the function space is created it is actually linear and ragged.
The \pycode{"mesh"} axis only has one axis component that represents all of the mesh entities and the different entity types are separated via a slicing operation.
By doing this we make explicit the `dual' nature of a DMPlex mesh (\cref{sec:foundations_dmplex}) - through indexing one can view mesh data \emph{either} as a flat array of points, or as one partitioned by entity type.

This is demonstrated in \cref{fig:mesh_renumbering_transform}.
The original mesh axis (left, labelled with indices $p_i$) is indeed interleaved as desired and the cell/edge/vertex axis tree partitioning only exists in the indexed axis tree.

\begin{algorithm}
  \begin{center}
    \begin{minipage}{.9\textwidth}
      \begin{pyalg2}
        def plex_renumbering(plex: PETSc.DMPlex, cell_order: np.ndarray[int]):
          # bookkeeping to track result
          renumbering = np.empty(num_points(plex))
          owned_ptr = 0
          ghost_ptr = num_owned_points(plex)
          seen_points = set()

          # loop over cells
          c_start, c_end = plex.getHeightStratum(0)
          for cell in range(c_start, c_end):
            # renumber the cell
            cell_renum = cell_order[cell]

            # pack points in the cell closure together, skipping already
            # seen points and respecting the owned/ghost partition
            closure = get_closure(plex, cell_renum)
            for pt in closure:
              if pt in seen:
                continue
              else:
                if is_ghost(pt):
                  renumbering[ghost_ptr] = pt
                  ghost_ptr += 1
                else:
                  renumbering[owned_ptr] = pt
                  owned_ptr += 1
                seen.add(pt)
        return renumbering
      \end{pyalg2}
    \end{minipage}
  \end{center}

  \caption{
    TODO.
  }
  \label{alg:plex_renumbering}
\end{algorithm}

To determine the point numbering a cell-local numbering is used.
Summarised in \cref{alg:plex_renumbering}, one loops over the cells of the mesh and packs entities in their closure together.
To further improve locality the cell loop is ordered with an RCM numbering so nearby cell closures are kept close in memory.
\Cref{alg:plex_renumbering} is very similar to alg. 3 of~\cite{langeEfficientMeshManagement2016} except, due to reasons explained in \cref{sec:communication_optimisations}, the \textit{non-core} (also referred to as \textit{owned}) entity class is dropped.

\section{Finite element assembly}

Now that we have the required global data structures we may now consider the finite element assembly process.
For the Stokes problem above, the assembly loop created for the LHS matrix is equivalent to the following \pyop3 code:

\begin{pyinline}
  loop(cell := mesh.cells.owned.index(),
       kernel(viscosity_dat[closure(cell)],
              output_mat[closure(cell), closure(cell)]))
\end{pyinline}

\begin{figure}
  \centering
  \includegraphics{scott_vogelius_index_tree.pdf}
  \caption{
    An complete index tree appropriate for transforming a Scott-Vogelius axis tree (\cref{fig:firedrake_axis_tree}) into a cell-local view.
  }
  \label{fig:firedrake_index_tree}
\end{figure}

\begin{itemize}
  \item
    \pycode{mesh.cells.owned} is a \pyop3 axis representing the owned cells of the mesh.

  \item
    \pycode{kernel} is the local kernel generated by TSFC using the form \pycode{a}.
    It has intents of \pycode{READ} for the first argument and \pycode{INC} for the second.

  \item
    The kernel arguments \pycode{viscosity_dat} and \pycode{output_mat} are a \pycode{Dat} and \pycode{Mat} respectively.
    Each is restricted to a view of the DoFs supported on the cell by indexing with \pycode{closure(cell)}.
    \pycode{viscosity_dat}, having only a single associated axis tree, takes only a single index, whilst \pycode{output_mat} requires two. \\
    As the axis trees for the data structures are non-flat simply indexing with \pycode{closure(cell)} is some syntactic sugar.
    Internally the index is translated into the appropriate index tree (\cref{fig:firedrake_index_tree}).
\end{itemize}

Notice that this closely resembles the pseudo-code shown in \cref{alg:fem_assembly}.

At this point we now have a loop expression correctly encapsulating the assembly process.
\pyop3 can now generate and execute code for it in parallel using the processes established in \cref{chapter:execution_model}.

\subsection{More integral types}

\begin{figure}
  \centering
  \includegraphics{interior_facet_support.pdf}
  \caption{
    Stencil for an interior facet integral.
    The facet currently being iterated over is marked with the star.
  }
  \label{fig:interior_facet_support}
\end{figure}

In addition to cells, Firedrake also supports integrating over the (interior and exterior) facets of a mesh.
In these cases the data that is packed and passed to the local kernel consists of the `macro' cell formed from the cells in the support of the facet (e.g. \cref{fig:interior_facet_support}).

With \pyop3 it is straightforward to express assembly loops for facet integrals using \textit{map composition} (\cref{sec:indexing_map_composition}):

\begin{pyinline}
  facet = mesh.interior_facets.owned.index()
  packed = dat[closure(support(facet))]
\end{pyinline}

\subsection{Handling orientations}

One of the big advantages of \pyop3 over \pyop2 is in its ability to express transformations that cannot be memoized.
An example of this is the way in which element orientations are handled.

% the problem
% -----------
% adjacent cells may disagree on the orientation of a shared facet or edge
% the local kernel is then packed incorrectly, so the computation is incorrect
% this is only a problem for some element types, the error may be avoided for simplices of any dimension and a non-mobius strip solution exists for quads by choosing a global numbering that dictates a particular entity orientation
% in particular this trick cannot be used for hexahedral elements

\begin{figure}
  \centering
  \begin{subfigure}{0.45\textwidth}
    \centering
    \includegraphics{lagrange_element_3_default.pdf}
  \end{subfigure}
  %
  \begin{subfigure}{0.45\textwidth}
    \centering
    \includegraphics{lagrange_element_3_flip.pdf}
  \end{subfigure}
  \caption{Reference $P_3$ (Lagrange, degree 3) element (left), and with an edge flipped (right).}
  \label{fig:element_orientation_permute}
\end{figure}

\begin{figure}
  \centering
  \begin{subfigure}{0.45\textwidth}
    \centering
    \includegraphics{raviart_thomas_default.pdf}
  \end{subfigure}
  %
  \begin{subfigure}{0.45\textwidth}
    \centering
    \includegraphics{raviart_thomas_flip.pdf}
  \end{subfigure}
  \caption{Reference Raviart-Thomas element (left), and with an edge flipped (right).}
  \label{fig:element_orientation_flip}
\end{figure}

The general solution to this problem is to make the packing code aware of the orientation of the different entities.
This is relatively straightforward for function spaces with point evaluation DoFs (\cref{fig:element_orientation_permute}) because the necessary transformation is simply a permutation and so can embedded in the cell-node maps.
This is what Firedrake does currently.

However, this approach fails to work for function spaces with vector-valued DoFs, such as \hdiv and \hcurl, since the transformation can include a scaling.
See, for example, the case shown in \cref{fig:element_orientation_flip}.
In order for the edge DoF to be packed correctly for both cells, one of them needs to scale its value by $-1$.
In general this transformation can be a matrix.
The software package Basix~\cite{scroggsBasixRuntimeFinite2022,scroggsConstructionArbitraryOrder2021} is capable of generating these transformation matrices.

% how things are done in pyop3
% ----------------------------
With \pyop3 we can express the permutation transformation as a symbolic indexing operation.
% this means we can generate code for it (do at runtime), step towards more complex runtime transformations
% compared to pyop2 where it is memoized in advance

% consider simplest possible, P3 and disregard non-edge DoFs
To explain how this is done we consider the edge flip shown in \cref{fig:element_orientation_permute}.
For simplicity consider a one-cell mesh and we omit cell and vertex DoFs.

As well as the axis tree for the function space we now have two extra pieces of information:
\begin{itemize}
  \item
    The orientation of the edges, stored as an array of integers per cell closure.
    For this single-cell-mesh case the orientation array of the flipped triangle is $\{ c_0 \to \pycode{[0, 1, 0]} \}$, or, equivalently, \pycode{orientations = [[0, 1, 0]]}.
    An orientation of 0 is considered `default', and 1 means that it is flipped.

  \item
    The DoF permutation for a given orientation.
    Since there are only two ways that an edge can be oriented this is therefore stored as $\{ 0 \to \pycode{[0, 1]},\ 1 \to \pycode{[1, 0]} \}$, or in code as \pycode{perms = [[0, 1], [1, 0]]}.
\end{itemize}

Now, in order to correctly pack the DoFs for the kernel the following permutation of the DoFs axis is required:
\begin{pyinline}
  cell = 0
  packed = dat[closure(cell)][:, [[0, 1], [1, 0], [0, 1]]]
\end{pyinline}
which is equivalent to following transformation:
\begin{center}
  \includegraphics{element_orientation_permute_data.pdf}
\end{center}

The array \pycode{[[0, 1], [1, 0], [0, 1]]} clearly comes from the permutation array above, so we may transform the code to:
\begin{pyinline}
  cell = 0
  packed = dat[closure(cell)][:, perms[[0, 1, 0]]]
\end{pyinline}
The final array \pycode{[0, 1, 0]} is simply the orientations array, giving the final transformation as:
\begin{pyinline}
  cell = 0
  packed = dat[closure(cell)][:, perms[orientations[cell]]]
\end{pyinline}

With this we now have a fully symbolic, indexing approach to permuting the DoFs for any function space.
Having a symbolic representation of this transformation means that one can generate code for the transformation, instead of memoizing the modified maps in advance.
Whilst not strictly necessary for the permutation case described here it is essential for more complex transformations that this be expressible.
Extending Firedrake and \pyop3 to support general transformations is future work.

\end{document}
