\documentclass[thesis]{subfiles}

\begin{document}

\chapter{Firedrake integration}
\label{chapter:firedrake}

One of the main advantages to a framework like \pyop3 is that, by automating a process typically done by hand, it allows for higher level abstractions to be built on top of it.
In this chapter we will demonstrate this by integrating \pyop3 into the finite element framework Firedrake~\cite{FiredrakeUserManual}.

\begin{figure}
  \includegraphics{firedrake_structure_old.pdf}
  \caption{
    The existing structure of the Firedrake finite element framework.
    Packages shown in red are managed by Firedrake whereas the grey packages are key external dependencies.
    Users typically express computations as Python scripts using both UFL and Firedrake; this is represented by the blue `User input' box.
    The direction of the arrows indicate a caller-callee relationship.
    For instance Firedrake calls \pyop2 but not vice versa.
    The bidirectional arrow between Firedrake and PETSc indicates that they each call the other, which for PETSc is via callback functions.
  }
  \label{fig:firedrake_structure_old}
\end{figure}

\section{Firedrake}

Firedrake is a Python framework for automating the solving of PDEs using the finite element method.
It uses code generation to convert near-mathematical expressions provided by the user into fast C code, resulting in a product that is both productive to use and fast to execute.
Firedrake is parallel by default; a script that works on someone's laptop can run without modification on a cluster\footnotemark.
It makes heavy use of \pyop2.

\footnotetext{
  For performance reasons the user may wish to modify their choice of solver and/or discretisation.
  See, for example,~\cite{betteridgeCodeGenerationProductive2021}.
}

The framework consists of a number of interacting software components.
Shown in \cref{fig:firedrake_structure_old} the core components, in addition to Firedrake and \pyop2, include: UFL (Unified Form Language)~\cite{alnaesUnifiedFormLanguage2014a}, TSFC (Two-Stage Form Compiler)~\cite{homolyaTSFCStructurePreservingForm2018}, loopy, and PETSc.
The contributions of these packages to the overall framework will be made clear below.
A number of packages not relevant to \pyop2/\pyop3 have been omitted for simplicity.

\begin{listing}
  \centering
  \begin{minipage}{.9\textwidth}
    \begin{pyalg2}
      from firedrake import *

      # initialise the mesh and function spaces
      mesh = UnitSquareMesh(10, 10)?\label{code:stokes_init_mesh}?
      V = VectorFunctionSpace(mesh, "P", 3)?\label{code:stokes_spaces_begin}?
      Q = FunctionSpace(mesh, "DP", 2)
      W = MixedFunctionSpace([V, Q])?\label{code:stokes_spaces_end}?

      u, p = TrialFunctions(W)
      v, q = TestFunctions(W)

      # viscosity
      nu = Constant(666)

      # define lhs and rhs
      a = nu * inner(grad(u), grad(v)) * dx - p * div(v) * dx + q * div(u) * dx?\label{code:stokes_bilinear_form}?
      L = Cofunction(W.dual())  # zero rhs?\label{code:stokes_linear_form}?

      # construct the boundary condition
      g = Constant([666, 666])
      bc = DirichletBC(W.sub(0), g, "on_boundary")

      # assemble and solve the problem
      solution = Function(W)
      solve(a == L, solution, bcs=[bc])?\label{code:stokes_solve}?
    \end{pyalg2}
  \end{minipage}
  \caption{
    Firedrake code for setting up and solving the Stokes problem from \cref{sec:stokes_equations}.
  }
  \label{listing:stokes_demo}
\end{listing}

The way in which these packages interact is best shown by way of an example.
\Cref{listing:stokes_demo} shows a `typical' Firedrake script for solving the Stokes equations with the formulation from \cref{sec:stokes_equations}.
It consists of the following steps:
\begin{enumerate}
  \item
    The mesh (l.~\ref{code:stokes_init_mesh}) and function spaces (l.~\ref{code:stokes_spaces_begin}-\ref{code:stokes_spaces_end}) are created.
    Internally the mesh is stored as a PETSc DMPlex (\cref{sec:foundations_dmplex}), which is distributed across processes with appropriate point star forests.
    The function spaces then associate DoFs with topological elements of the mesh according to the provided finite element definition.
    In this case the finite elements are specified by the arguments `\pycode{"P"}' and `\pycode{3}', and `\pycode{"DP"}' and `\pycode{2}' to `\pycode{VectorFunctionSpace}' and `\pycode{FunctionSpace}' respectively.
    These correspond to the $[P_3]^2$ and $P_2^{\textnormal{disc}}$ elements that together form a Scott-Vogelius element.

  \item
    The linear (l.~\ref{code:stokes_linear_form}) and bilinear (l.~\ref{code:stokes_bilinear_form}) forms are declared symbolically with UFL.
    Notice how the definition of the bilinear form in particular strongly resembles the mathematics of the original weak form (\cref{eq:weak_stokes}).
    The linear form `\pycode{L}' is known to be trivially zero and so a pre-assembled cofunction is used.

  \item
    Lastly, the PDE is solved (l.~\ref{code:stokes_solve}) with appropriate boundary conditions.
    To solve it, the bilinear (\pycode{a}) and linear (\pycode{L}) forms are first assembled into a matrix and vector respectively.
    Since the linear form for this PDE is known zero its assembly is trivial.
    Then, having assembled the linear system, it is then solved by PETSc with its wealth of different solver types, putting the solution for both the velocity and pressure together into the mixed function \pycode{solution}.

    To accomodate for the boundary condition, values in the assembled matrix and vector corresponding to the constrained DoFs are dropped.
\end{enumerate}

\subsection{\pyop2's contribution}

\pyop2 is a crucial component to this process.
To start with, \emph{all of Firedrake's global data structures wrap \pyop2 ones}.
For instance \pycode{Functions} and \pycode{Cofunctions} wrap \pyop2 \pycode{Dats}, and assembled bilinear forms, represented by Firedrake \pycode{Matrix} objects, wrap \pyop2 \pycode{Mats}.
The \pycode{DataSets} that act as data layout descriptors for the \pyop2 objects are drawn from the \emph{nodes} of the function spaces.

Further, \emph{\pyop2 is responsible for finite element assembly}.
The UFL forms are converted by TSFC into local kernels (i.e. stencil payloads) expressed as a loopy \pycode{LoopKernel} object.
\pyop2 then handles the outer loop over mesh entities (here the cells), evaluating the local kernel for each.
A parallel loop would be constructed with the following structure:
\begin{pyinline}
  par_loop(local_kernel,                         # stencil payload
           mesh.cell_set,                        # iteration set
           dat0(READ, cell2nodes)                # arguments
           mat0(INC, (cell2nodes, cell2nodes)))  #   "
\end{pyinline}
Notice that since \pycode{Dats} and \pycode{Mats} store data relative to nodes of the function spaces, as opposed to topological entities of the mesh, the indirection maps map from the iteration set to nodes of the function space.

In order to integrate \pyop3 with Firedrake, replacing \pyop2, we must clearly replace these two elements of \pyop2 with their \pyop3 equivalents: nodal \pycode{DataSets} must be replaced with \pycode{AxisTrees} and \pyop2 parallel loops must be replaced by \pyop3 loop expressions.

\begin{figure}
  \centering
  \includegraphics{scott_vogelius_element_dof_layout.pdf}
  \caption{
    Appropriate global data layout for a function space with a Scott-Vogelius element.
  }
  \label{fig:firedrake_data_tree}
\end{figure}

\section{Representing global data with axis trees}

\todo[inline]{I think that a better way to tell this story would be to present the `naive' approach first and then say `oh we missed a step'. It is more intuitive.}

\begin{figure}
  \centering
  \begin{subfigure}{\textwidth}
    \centering
    \includegraphics{scott_vogelius_element_dof_layout_plain.pdf}
    \caption{
      The data layout.
    }
    \label{fig:firedrake_data_tree_plain}
  \end{subfigure}

  \vspace{1em}

  \begin{subfigure}{\textwidth}
    \centering
    \includegraphics{scott_vogelius_space_axis_tree_plain.pdf}
    \caption{
      The axis tree.
      The variables \pycode{nnodes_V} and \pycode{nnodes_Q} are \pyop3 \pycode{Dats}, indicating that the data layout is ragged.
    }
    \label{fig:firedrake_axis_tree_plain}
  \end{subfigure}
  \caption{
    The initial ragged data layout for a function space.
  }
  \label{fig:firedrake_plain}
\end{figure}

\begin{figure}
  \centering
  \includegraphics[scale=.8]{scott_vogelius_space_axis_tree.pdf}
  \caption{
    The (indexed) axis tree generated after indexing the ragged axis tree in \cref{fig:firedrake_axis_tree_plain}.
    The mesh entities are distinguishable whilst remaining interleaved.
  }
  \label{fig:firedrake_axis_tree}
\end{figure}

\begin{figure}
  \centering
  \includegraphics[scale=.7]{mesh_renumbering_transform.pdf}
  \caption{
    The indexing transformation taking a ragged axis tree for a function space to one where the different topological entities are distinguishable.
    The arrays \pycode{[0,3,...]}, \pycode{[2,...]}, and \pycode{[1,4,...]} represent the indices of the cells, edges, and vertices respectively.
  }
  \label{fig:mesh_renumbering_transform}
\end{figure}

In many respects the fundamental behaviour of global data structures is unchanged between \pyop2 and \pyop3.
Both use \numpy arrays and PETSc matrices to store vectors and matrices as \pycode{Dats} and \pycode{Mats}, and \pycode{Dats} still use star forests to exchange ghost data.
The core difference between the two libraries is in the addition of axis trees, which replace \pyop2 \pycode{DataSets}.

In order to store function space data in \pyop3 \pycode{Dats} and \pycode{Mats}, appropriate axis trees must first be constructed to prescribe the data layouts.
If we consider the Scott-Vogelius example that we have been using, we would expect, having applied a mesh renumbering, to have a data layout like that shown in \cref{fig:firedrake_data_tree}.
The layout naturally decomposes into multiple separate axes: the function spaces (\pycode{"field"}), the points of the mesh (\pycode{"mesh"}), the per-entity nodes (\pycode{"node"}), and any vector nature of the nodes (\pycode{"component"}).

Having this hierarchical structure is convenient.
Views of particular function spaces or vector components are straightforward to obtain via indexing and the axes may be freely permuted to give data layouts with the same semantics but different data locality properties.

\subsection{Renumbering the mesh}
\label{sec:firedrake_renumbering_mesh}

Given this decomposition it would appear simple to construct an appropriate axis tree to describe the layout.
However, there is a complication: axis components are stored contiguously.
Constructing a \pycode{"mesh"} axis naively as
\begin{pyinline}
  mesh_axis = Axis({"cell": ncells, "vertex": nvertices, "edge": nedges},
                   "mesh")
\end{pyinline}
leads to a non-interleaved layout with all cell DoFs preceding all vertex DoFs preceding all edge DoFs.
This is clearly incorrect.

To fix this we take a two step approach.
First, instead of constructing a multi-component \pycode{"mesh"} axis from the start, we instead construct a single component axis where the component represents all points in the mesh:
\begin{pyinline}
  mesh_axis = Axis({"point": npoints}, "mesh")
\end{pyinline}
We can then construct an axis tree with the correct DoF ordering, albeit without being able to distinguish the different mesh entities.
This is demonstrated in \cref{fig:firedrake_plain}.
Observe that, since mesh entities are indistinguishable, the number of nodes per-entity is no longer fixed and so we have a ragged axis tree.

Given this axis tree, the correct entity information may be recovered via \emph{indexing}.
To do so we index the \pycode{"mesh"} axis with the slice:
\begin{pyinline}
  Slice("mesh", [Subset("point", icell, label="cell"),
                 Subset("point", iedge, label="edge"),
                 Subset("point", ivert, label="vertex")])
\end{pyinline}
Where \pycode{icell}, \pycode{iedge} and \pycode{ivert} are integer arrays containing the cell, edge, and vertex indices respectively.
Indexing the \pycode{"mesh"} axis with this slice yields a new multi-component axis where the different entity types may be distinguished and separately addressed.
This transformation is shown in \cref{fig:mesh_renumbering_transform}.
The result of applying this index to the single-component, ragged Scott-Vogelius axis tree (\cref{fig:firedrake_axis_tree_plain}) is shown in \cref{fig:firedrake_axis_tree}.

By indexing the \pycode{"mesh"} axis in this way we make it such that the resulting axis tree has `dual' representations - as flat points or as a collection of distinct mesh entities - which is exactly the same approach as DMPlex (see discussion in \cref{sec:foundations_dmplex}).
\pyop3, with its indexable axis trees, is therefore in some sense \emph{a generalisation of the abstractions provided by DMPlex}.

\subsubsection{Computing the renumbering}

\begin{algorithm}
  \caption{
    Algorithm that computes the point renumbering to improve data locality in finite element assembly loops.
    Owned points are always stored before ghost points.
  }

  \begin{center}
    \begin{minipage}{.9\textwidth}
      \begin{pyalg2}
        def plex_renumbering(plex: PETSc.DMPlex, cell_order: np.ndarray[int]):
          # bookkeeping to track result
          renumbering = np.empty(num_points(plex))
          owned_ptr = 0
          ghost_ptr = num_owned_points(plex)
          seen_points = set()

          # loop over cells
          c_start, c_end = plex.getHeightStratum(0)
          for cell in range(c_start, c_end):
            # renumber the cell
            cell_renum = cell_order[cell]

            # pack points in the cell closure together, skipping already
            # seen points and respecting the owned/ghost partition
            closure = get_closure(plex, cell_renum)
            for pt in closure:
              if pt in seen_points:
                continue
              else:
                if is_ghost(pt):
                  renumbering[ghost_ptr] = pt
                  ghost_ptr += 1
                else:
                  renumbering[owned_ptr] = pt
                  owned_ptr += 1
                seen_points.add(pt)
        return renumbering
      \end{pyalg2}
    \end{minipage}
  \end{center}
  \label{alg:plex_renumbering}
\end{algorithm}

\Cref{alg:plex_renumbering} is used to compute the point renumbering used to interleave DoFs for the function spaces.
One loops over the cells of the mesh and packs entities in their closure together.
This means that DoFs associated with edges and vertices will be `close' in memory to their incident cell, improving performance for finite element assembly loops where one accesses all DoFs in a cell's closure together.
In addition to this, to further improve locality the outer loop over cells loop is ordered with an reverse Cuthill-McKee numbering~\cite{cuthillReducingBandwidthSparse1969} (\pycode{cell_order} in \cref{alg:plex_renumbering}).
This means that nearby cell closures are kept close in memory, improving data reuse for shared entities between adjacent cells.

The algorithm is very similar to that already used by Firedrake and \pyop2 (alg. 3 in~\cite{langeEfficientMeshManagement2016}) except, due to reasons explained in \cref{sec:communication_optimisations}, the \ownediter entity class is dropped.

\section{Finite element assembly}

\begin{figure}
  \centering
  \includegraphics{scott_vogelius_index_tree.pdf}
  \caption{
    An index tree appropriate for transforming an axis tree for a Scott-Vogelius function space (\cref{fig:firedrake_axis_tree}) into a cell-local view of the cell and its closure.
  }
  \label{fig:firedrake_index_tree}
\end{figure}

Now that we have the required global data structures we may now consider the finite element assembly process.
For the Stokes problem above, the assembly loop created for the LHS matrix is equivalent to the following \pyop3 code:
\begin{pyinline}
  loop(cell := mesh.cells.owned.index(),
       kernel(viscosity_dat[closure(cell)],
              output_mat[closure(cell), closure(cell)]))
\end{pyinline}
Which expresses the operation that one should loop over the locally-owned cells in the mesh (\pycode{mesh.cells.owned}) and execute the local kernel (\pycode{kernel}) for each one.
\pycode{kernel} takes two arguments, with access descriptors \pycode{READ} and \pycode{INC}, representing the viscosity (a \pycode{Dat}) and output matrix (a \pycode{Mat}), each restricted, via indexing, to a view of the DoFs supported on the cell.
\pycode{viscosity_dat}, having only a single associated axis tree, takes only a single index, whilst \pycode{output_mat} requires two.

Note that, since the axis trees for \pycode{viscosity_dat} and \pycode{output_mat} are non-flat, simply indexing with `\pycode{closure(cell)}' is syntactic sugar.
Internally it is translated into an index tree appropriate for indexing all of the axes.
This is shown in \cref{fig:firedrake_index_tree}.

With such a loop expression, \pyop3 is able to generate and execute code in parallel using the processes established in \cref{chapter:pyop3}.

\subsection{More integral types}
\label{sec:firedrake_facet_integration}

\begin{figure}
  \centering
  \includegraphics{interior_facet_support.pdf}
  \caption{
    Stencil for an interior facet integral.
    The facet currently being iterated over is marked with the star.
  }
  \label{fig:interior_facet_support}
\end{figure}

As well as cells, Firedrake also supports integrating over the (interior and exterior) facets of a mesh.
In these cases the data that is packed and passed to the local kernel consists of the `macro' cell formed from the cells in the support of the facet (e.g. \cref{fig:interior_facet_support}).

With \pyop3 it is straightforward to express assembly loops for facet integrals using map composition (\cref{sec:indexing_map_composition}):
\begin{pyinline}
  facet = mesh.interior_facets.owned.index()
  packed = dat[closure(support(facet))]
\end{pyinline}
This works because the support of an interior facet ($\support(f)$) consists of the two cells it touches, and the closure of those gives the correct macro cell.

\subsection{Handling orientations}
\label{sec:firedrake_orientations}

\begin{figure}
  \centering
  \begin{subfigure}{0.45\textwidth}
    \centering
    \includegraphics{lagrange_element_3_default.pdf}
  \end{subfigure}
  %
  \begin{subfigure}{0.45\textwidth}
    \centering
    \includegraphics{lagrange_element_3_flip.pdf}
  \end{subfigure}
  \caption{Reference $P_3$ element (left), and with an edge flipped (right).}
  \label{fig:element_orientation_permute}
\end{figure}

\begin{figure}
  \centering
  \begin{subfigure}{0.45\textwidth}
    \centering
    \includegraphics{raviart_thomas_default.pdf}
  \end{subfigure}
  %
  \begin{subfigure}{0.45\textwidth}
    \centering
    \includegraphics{raviart_thomas_flip.pdf}
  \end{subfigure}
  \caption{Reference Raviart-Thomas element (left), and with an edge flipped (right).}
  \label{fig:element_orientation_flip}
\end{figure}

For finite element assembly, it is important that the data being packed into the local temporary is provided in `canonical' form.
In order to do this, adjacent cells must agree on the orientation of shared entities (i.e. facets, edges, and vertices).
If these orientations do not agree then the local assembly kernel is passed mangled data, leading to an erroneous calculation.

Two simple examples of where orientations have an impact are shown in \cref{fig:element_orientation_permute,fig:element_orientation_flip}.
Both show the impact of packing data for a finite element calculation for a triangle where one of the edges has been flipped.
In both cases the packed data would be mangled.
In \cref{fig:element_orientation_permute} the DoFs \textbf{3} and \textbf{4} are packed in the wrong order and the vector DoF in \cref{fig:element_orientation_flip} is pointing the wrong direction.

A number of approaches exist to rectify this issue.
For meshes with simplex cells one can globally number the vertices of the mesh such that adjacent cells always agree on the orientation of the shared entities~\cite{rognesEfficientAssemblyDiv2010}.
This numbering approach is also possible, with some limitations, for meshes with quadrilateral~\cite{homolyaParallelEdgeOrientation2016} and hexahedral~\cite{agelekOrientingEdgesUnstructured2017} cells.
However, the approach does not work in general across the full gamut of cell types - including mixed cell meshes - that can occur in finite element simulations.
Instead, one needs to encode the orientation of the shared entities relative to the cell and \emph{transform the DoFs} appropriately.

For DoF transformations that are expressible as permutations (e.g. \cref{fig:element_orientation_permute}) this transformation may be handled in advance of packing by modifying the cell-node map used to pack the temporary.
This is the approach taken by Firedrake that allows it to support Lagrange elements on hexahedral meshes.
However, some transformations require more general linear transformations that cannot be computed in advance~\cite{scroggsBasixRuntimeFinite2022,scroggsConstructionArbitraryOrder2021}.
For example, to un-mangle the DoFs for \cref{fig:element_orientation_flip} the offending DoF must be flipped by multiplying by $-1$.
This operation cannot be done in advance and must be done as part of the packing transformation.
\pyop2 has no support for this type of transformation and so \hdiv and \hcurl elements are not supported on hexahedral meshes.

\subsubsection{Expressing transformations with \pyop3}

% "part of the generated code" and substituted layout... (at runtime)

Unlike \pyop2, expressing DoF transformations of this sort is relatively straightforward to do with \pyop3.
Since DoF packing is expressed simply as an indexing operation taking, say, a \pycode{Dat} and returning another \pycode{Dat}, further transformations can be applied to the packed object.
For instance, one could do:
\begin{pyinline}
  packed0 = dat[closure(cell)]
  packed1 = packed0[transform1]
  packed2 = packed1[transform2]
\end{pyinline}
This sort of operation is not expressible in \pyop2 because packed temporaries are represented by a different type.

To give an example of this working in practice we consider packing the edge DoFs for the case shown in \cref{fig:element_orientation_permute}, omitting cell and vertex DoFs for simplicity.
We want to permute the DoFs for a single edge, so the transformation that we want to express can be visualised as:
\begin{center}
  \includegraphics{element_orientation_permute_data.pdf}
\end{center}
Which corresponds the following \pyop3 code:
\begin{pyinline}
  cell = 0  # there is only one cell
  packed = dat[closure(cell)][:, [[0,1],[1,0],[0,1]]]
\end{pyinline}
The slice (\pycode{:}) included in the transformation indicates that the edge axis should be unchanged.

In order to extend this transformation to the more general case with multiple cells two additional pieces of information are required:
\begin{enumerate}
  \item
    \emph{The orientation of the edges, stored as an array of integers per cell closure.}
    For this simple case the orientation array of the flipped triangle is $\{ c_0 \to \pycode{[0,1,0]} \}$, or, equivalently, \pycode{orientations = [[0,1,0]]}.
    An orientation of 0 is considered `default', and 1 means that it is flipped.

  \item
    \emph{The DoF permutation for a given orientation.}
    Since there are only two ways that an edge can be oriented this is therefore stored as $\{ \pycode{0} \to \pycode{[0,1]},\ \pycode{1} \to \pycode{[1,0]} \}$, or in code as \pycode{perms = [[0,1],[1,0]]}.
\end{enumerate}

With these we can begin to transform the packing expression to something more general.
If we first substitute the permutations `\pycode{[[0,1],[1,0],[0,1]]}' with the appropriate entries from \pycode{perms} then we get:
\begin{pyinline}
  cell = 0
  packed = dat[closure(cell)][:, perms[[0,1,0]]]
\end{pyinline}
Then, we can observe that the entries `\pycode{[0,1,0]}' are precisely the orientations of the edges, allowing us to make the following substitution:
\begin{pyinline}
  cell = 0
  packed = dat[closure(cell)][:, perms[orientations[cell]]]
\end{pyinline}
Giving us a general expression that applies to meshes with any number of cells.
Having applied this additional transformation, the resulting indexed \pycode{Dat} can now be safely passed through to the local kernel because its DoFs have been transformed into `canonical' form.

At present only permutation transformations have been implemented, to allow the \pyop3 version of Firedrake to assemble Lagrange elements on hexahedra, but extending the approach to support more general transformations should be straightforward.

\end{document}
