% Data tree for a ragged array
\documentclass[tikz]{standalone}

\usepackage{tkz-euclide}
\usetikzlibrary{arrows,calc,graphs,graphdrawing,positioning,tikzmark,shapes.geometric,patterns.meta,decorations.pathreplacing}

\begin{document}
\begin{tikzpicture}

\definecolor{celldofcolor}{rgb}{0.0,0.0,0.6}
\definecolor{edgedofcolor}{rgb}{0.8,0.0,0.0}
\definecolor{vertdofcolor}{RGB}{154,205,50}
\definecolor{nodedofcolor}{rgb}{0.5,0.5,0.5}

\colorlet{bgcelldofcolor}{celldofcolor!60}
\colorlet{bgedgedofcolor}{edgedofcolor!60}
\colorlet{bgvertdofcolor}{vertdofcolor!60}
\colorlet{bgnodedofcolor}{nodedofcolor!60}


\tikzstyle {background} = [lightgray];
\tikzstyle{ptlabel} = [anchor=center,color=black,opacity=1]
\tikzstyle{connector} = [densely dashed]

% TODO: Remove
\newcommand{\drawBox}[5]{
  \filldraw[draw=black,fill=#5,line width=.7pt] #1 rectangle #2;
  \node [at={#3},fill=#5] {#4};
}

\newcommand{\drawBoxNew}[3]{
  \filldraw[draw=black,fill=#3,line width=.7pt] #1 rectangle #2;
}

\newcommand{\labelBox}[3]{
  \node [at={#1},fill=#3,inner sep=2pt,minimum width=0pt] {#2};
}

\newcommand{\drawDof}[3]{%
  \filldraw[draw=black,fill=#3,line width=.7pt] #1 rectangle ++ (1,1);
  \node [at={($#1+(.5,.5)$)}] {#2};
}

\newcommand{\drawCellDof}[2]{%
  \drawDof{#1}{#2}{bgcelldofcolor};
}

\newcommand{\drawEdgeDof}[2]{%
  \drawDof{#1}{#2}{bgedgedofcolor};
}

\newcommand{\drawVertDof}[2]{%
  \drawDof{#1}{#2}{bgvertdofcolor};
}

\newcommand{\drawNodeDof}[2]{%
  \drawDof{#1}{#2}{bgnodedofcolor};
}

\newcommand{\drawWhiteDof}[2]{%
  \drawDof{#1}{#2}{white};
}

\newcommand{\drawWhiteDofSmall}[2]{%
  \drawDof{#1}{\footnotesize{#2}}{white};
}

\newcommand{\drawDofSmall}[3]{%
  \drawDof{#1}{\footnotesize{#2}}{#3};
}

% https://tex.stackexchange.com/questions/123760/draw-crosses-in-tikz
\tikzset{cross/.style={cross out,draw=black,minimum size=8pt,inner sep=0pt,outer sep=0pt,line width=1.5pt}};

\newcommand{\drawCross}[1]{
  \draw #1 node[cross] {};
}

\newcommand{\drawComma}[1]{
  \node [font=\huge] at #1 {,};
}

\newcommand{\drawConnector}[2]{
  \draw [connector] #1 -- #2;
}

\newcommand{\labelLayoutAxis}[1]{
  \node [at={(-1.8,\yshift+.5)},fill=none] {#1};
}


% top
\coordinate (origin) at (0,0);
\drawWhiteDof{(origin)}{$a_0$};
\drawWhiteDof{($(origin)+(1,0)$)}{$a_1$};

\coordinate (aleft) at (origin);
\coordinate (amid) at ($(origin)+(1,0)$);
\coordinate (aright) at ($(origin)+(2,0)$);

% middle left
\coordinate (origin) at ($(amid)+(-2.5,-2)$);
\drawWhiteDof{(origin)}{$b_0$};
\drawWhiteDof{($(origin)+(1,0)$)}{$b_1$};

\draw [connector] (aleft) -- ($(origin)+(0,1)$);
\draw [connector] (amid) -- ($(origin)+(2,1)$);

\coordinate (b00) at (origin);
\coordinate (b01) at ($(origin)+(1,0)$);
\coordinate (b02) at ($(origin)+(2,0)$);

% middle right
\coordinate (origin) at ($(amid)+(.5,-2)$);
\drawWhiteDof{(origin)}{$b_0$};
\drawWhiteDof{($(origin)+(1,0)$)}{$b_1$};

\draw [connector] (amid) -- ($(origin)+(0,1)$);
\draw [connector] (aright) -- ($(origin)+(2,1)$);

\coordinate (b10) at (origin);
\coordinate (b11) at ($(origin)+(1,0)$);
\coordinate (b12) at ($(origin)+(2,0)$);

% bottom p0 (1)
\coordinate (origin) at ($(b00)+(-.25,-2)$);
\drawWhiteDof{(origin)}{$c_0$};
\draw [connector] (b00) -- ($(origin)+(0,1)$);
\draw [connector] (b01) -- ($(origin)+(1,1)$);

% bottom p1 (0)
\coordinate (origin) at ($(b01)+(-.25,-2)$);
\draw ($(origin)+(.5,.5)$) node[cross] {};
\draw [connector] ($(b01)+(.5,0)$) -- ($(origin)+(.5,.5)$);

% bottom p2 (2)
\coordinate (origin) at ($(b10)+(-1.25,-2)$);
\drawWhiteDof{(origin)}{$c_0$};
\drawWhiteDof{($(origin)+(1,0)$)}{$c_1$};
\draw [connector] (b10) -- ($(origin)+(0,1)$);
\draw [connector] (b11) -- ($(origin)+(2,1)$);

% bottom p3 (1)
\coordinate (origin) at ($(b11)+(.25,-2)$);
\drawWhiteDof{(origin)}{$c_0$};
\draw [connector] (b11) -- ($(origin)+(0,1)$);
\draw [connector] (b12) -- ($(origin)+(1,1)$);

\end{tikzpicture}
\end{document}
