\documentclass[tikz]{standalone}

\usepackage{tkz-euclide}
\usetikzlibrary{arrows,calc,graphs,graphdrawing,positioning,tikzmark,shapes.geometric,patterns.meta,decorations.pathreplacing}

\usepackage{minted}
\usepackage{amsmath}
\usepackage{mymacros}

\begin{document}
\begin{tikzpicture}[remember picture]

\newcommand{\drawCompositionCircle}[1]{
  \node [circle,draw,minimum size=2mm,inner sep=0] at #1 {};
}

\newcommand{\bigArrowNew}[2]{
  \draw [line width=.8pt,-{Stealth}] #1 -- ++ #2;
}

\newcommand{\bigArrow}[1]{
  \bigArrowNew{#1}{(1,0)}
}


\tikzstyle{treenodecommon} = [minimum width=1.5cm];
\tikzstyle{treenode} = [treenodecommon,minimum height=.7cm,yshift=.15cm,text centered,draw=black,line width=.7pt,anchor=south west];
\tikzstyle{treenodebg} = [treenodecommon,minimum height=1cm,draw=none,anchor=south west];

\tikzstyle{treenodearrow} = [-{Stealth[sep=2pt]},draw=red,line width=.7pt];
\tikzstyle{underline} = [draw=red,line width=.7pt];

\newcommand{\drawTreeNode}[3]{
  \node [treenodebg] (xxx) at #2 {};
  \node [treenode] #1 at #2 {#3};
}

\newcommand{\underlineComponentWithWidth}[3]{
  % This is basically magic. For nice alignment I want the x values of the
  % component (\subnode), but the y values from the full node.
  % https://tikz.dev/library-math
  \tikzmath{
    coordinate \p;
    \p1 = (#2.west);
    \p2 = (#2.east);
    \p3 = (#1.south);
  }
  \draw [underline] ($(\px1,\py3)+(#3,.15)$) -- ($(\px2,\py3)+(-#3,.15)$);
}

\newcommand{\underlineComponent}[2]{
  \underlineComponentWithWidth{#1}{#2}{.15};
}

\newcommand{\underlineComponentNarrow}[2]{
  \underlineComponentWithWidth{#1}{#2}{.05};
}

\newcommand{\drawTreeConnector}[3]{
  \tikzmath{
    coordinate \p;
    \p1 = (#2.center);
    \p2 = (#1.south);
  }
  \draw [treenodearrow] ($(\px1,\py2)+(0,.15)$) -- #3;
}

\newcommand{\component}{\subnode[minimum width=0pt]};

\definecolor{celldofcolor}{rgb}{0.0,0.0,0.6}
\definecolor{edgedofcolor}{rgb}{0.8,0.0,0.0}
\definecolor{vertdofcolor}{RGB}{154,205,50}
\definecolor{nodedofcolor}{rgb}{0.5,0.5,0.5}

\colorlet{bgcelldofcolor}{celldofcolor!60}
\colorlet{bgedgedofcolor}{edgedofcolor!60}
\colorlet{bgvertdofcolor}{vertdofcolor!60}
\colorlet{bgnodedofcolor}{nodedofcolor!60}


\tikzstyle {background} = [lightgray];
\tikzstyle{ptlabel} = [anchor=center,color=black,opacity=1]
\tikzstyle{connector} = [densely dashed]

% TODO: Remove
\newcommand{\drawBox}[5]{
  \filldraw[draw=black,fill=#5,line width=.7pt] #1 rectangle #2;
  \node [at={#3},fill=#5] {#4};
}

\newcommand{\drawBoxNew}[3]{
  \filldraw[draw=black,fill=#3,line width=.7pt] #1 rectangle #2;
}

\newcommand{\labelBox}[3]{
  \node [at={#1},fill=#3,inner sep=2pt,minimum width=0pt] {#2};
}

\newcommand{\drawDof}[3]{%
  \filldraw[draw=black,fill=#3,line width=.7pt] #1 rectangle ++ (1,1);
  \node [at={($#1+(.5,.5)$)}] {#2};
}

\newcommand{\drawCellDof}[2]{%
  \drawDof{#1}{#2}{bgcelldofcolor};
}

\newcommand{\drawEdgeDof}[2]{%
  \drawDof{#1}{#2}{bgedgedofcolor};
}

\newcommand{\drawVertDof}[2]{%
  \drawDof{#1}{#2}{bgvertdofcolor};
}

\newcommand{\drawNodeDof}[2]{%
  \drawDof{#1}{#2}{bgnodedofcolor};
}

\newcommand{\drawWhiteDof}[2]{%
  \drawDof{#1}{#2}{white};
}

\newcommand{\drawWhiteDofSmall}[2]{%
  \drawDof{#1}{\footnotesize{#2}}{white};
}

\newcommand{\drawDofSmall}[3]{%
  \drawDof{#1}{\footnotesize{#2}}{#3};
}

% https://tex.stackexchange.com/questions/123760/draw-crosses-in-tikz
\tikzset{cross/.style={cross out,draw=black,minimum size=8pt,inner sep=0pt,outer sep=0pt,line width=1.5pt}};

\newcommand{\drawCross}[1]{
  \draw #1 node[cross] {};
}

\newcommand{\drawComma}[1]{
  \node [font=\huge] at #1 {,};
}

\newcommand{\drawConnector}[2]{
  \draw [connector] #1 -- #2;
}

\newcommand{\labelLayoutAxis}[1]{
  \node [at={(-1.8,\yshift+.5)},fill=none] {#1};
}


\tikzmath {
  \yshift = 0;
}

\drawTreeNode{(space)}{(0,\yshift)}{$\textnormal{slice}(\pycode{"space"},\ \{ \component{space0}{\pycode{"Vh"}: \pycode{[::]}},\ \component{space1}{\pycode{"Qh"}: \pycode{[::]}} \})$};

\tikzmath { \yshift = \yshift - 2; }

\drawTreeNode{(mesh0)}{(-5,\yshift)}{$\textnormal{map}(f(c),\ \{ \component{mesh00}{\pycode{"vertex"}},\ \component{mesh01}{\pycode{"edge"}},\ \component{mesh02}{\pycode{"cell"}} \})$};
\underlineComponent{space}{space0};
\drawTreeConnector{space}{space0}{(mesh0)};

\drawTreeNode{(mesh1)}{(5,\yshift)}{$\textnormal{map}(f(c),\ \{ \component{mesh10}{\pycode{"vertex"}},\ \component{mesh11}{\pycode{"edge"}},\ \component{mesh12}{\pycode{"cell"}} \})$};
\underlineComponent{space}{space1};
\drawTreeConnector{space}{space1}{(mesh1)};

\tikzmath {
  \yshift = \yshift - 2;
  \xstart = -12;
  \xshift = 5;
}

\newcommand{\drawInnermost}[2]{
  \drawInnermostScalar{#1}{#2}
  \drawTreeNode{(dof#1#2)}{(\xstart+\xshift*#2,\yshift-2)}{$\textnormal{slice}(\pycode{"component"},\ \{ \pycode{[::]} \})$}
  \underlineComponent{node#1#2}{node#1#20}
  \drawTreeConnector{node#1#2}{node#1#20}{(dof#1#2)}
}

\newcommand{\drawInnermostScalar}[2]{
  \drawTreeNode{(node#1#2)}{(\xstart+\xshift*#2,\yshift)}{$\textnormal{slice}(\pycode{"node"},\ \{ \component{node#1#20}{\pycode{[::]}} \})$}
  \underlineComponent{mesh#1}{mesh#1#2}
  \drawTreeConnector{mesh#1}{mesh#1#2}{(node#1#2)}
}

\drawInnermost{0}{0}
\drawInnermost{0}{1}
\drawInnermost{0}{2}

\tikzmath {
  \xstart = 3;
  \xshift = 4;
}

\drawInnermostScalar{1}{0}
\drawInnermostScalar{1}{1}
\drawInnermostScalar{1}{2}

\end{tikzpicture}
\end{document}
