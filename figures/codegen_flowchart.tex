\documentclass[tikz]{standalone}
\usepackage{tkz-euclide}
\usetikzlibrary{arrows,calc,graphs,graphdrawing,positioning,tikzmark,shapes.geometric,patterns.meta,decorations.pathreplacing}


\begin{document}
\begin{tikzpicture}[yscale=-1]

\tikzstyle{flowchartnode} = [
  draw=black,
  align=center,
  anchor=west,
  font=\footnotesize,
  text centered,
  minimum height=.8cm,
];
\tikzstyle{process} = [
  flowchartnode,
  rectangle,
  rounded corners,
  fill=red!70,
];
\tikzstyle{io} = [
  flowchartnode,
  trapezium,
  trapezium left angle=70,
  trapezium right angle=110,
  fill=blue!30,
  trapezium stretches=true,
];
\tikzstyle{arrow} = [-{stealth}];


\coordinate (start) at (0,0);
\tikzmath{
  \xshift = .8;
  \yshift = 1.5;
}

\node (pyop3) [io] at (start) {Loop expression};
\node (transform_pyop3) [process] at ($(pyop3.east)+(\xshift,0)$) {Transform loop\\expression};
\node (lower_pyop3) [process] at ($(transform_pyop3.east)+(\xshift,0)$) {Lower loop expression\\to loopy kernel};

\node (transform_loopy) [process] at ($(lower_pyop3.east)+(\xshift*.6,\yshift*.5)$) {Transform loopy kernel};

\node (compiled_function) [io] at ($(start)+(0,\yshift)$) {Compiled function};
\node (compile_code) [process] at ($(compiled_function.east)+(\xshift,0)$) {Compile C string};
\node (lower_loopy) [process] at ($(compile_code.east)+(\xshift,0)$) {Lower loopy kernel\\to C string};

\draw [arrow] (pyop3) -- (transform_pyop3);
\draw [arrow] (transform_pyop3) -- (lower_pyop3);
% \draw [arrow] (lower_pyop3) -- (transform_loopy);
% \draw [arrow] (transform_loopy) -- (lower_loopy);
\draw [arrow] (lower_loopy) -- (compile_code);
\draw [arrow] (compile_code) -- (compiled_function);

\draw [arrow,rounded corners=5pt] (lower_pyop3.east) -| (transform_loopy.north);
\draw [arrow,rounded corners=5pt] (transform_loopy.south) |- (lower_loopy.east);

\draw [arrow,densely dashed] (pyop3.west) to [bend left=60] node[midway,left,align=center,font=\footnotesize] {Call with data from\\loop expression} (compiled_function.west);

\end{tikzpicture}
\end{document}
