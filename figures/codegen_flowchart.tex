\documentclass[tikz]{standalone}
\usepackage{tkz-euclide}
\usepackage{ifthen}
\usetikzlibrary{arrows,calc,graphs,graphdrawing,positioning,tikzmark,shapes.geometric,patterns.meta,decorations.pathreplacing,math,fit,patterns}

\usepackage{mymacros}


\begin{document}
\begin{tikzpicture}

\tikzstyle{flowchartnode} = [
  draw=black,
  align=center,
  anchor=center,
  font=\footnotesize,
  text centered,
  minimum height=.8cm,
];
\tikzstyle{process} = [
  flowchartnode,
  rectangle,
  rounded corners,
  fill=red!70,
];
\tikzstyle{io} = [
  flowchartnode,
  trapezium,
  trapezium left angle=70,
  trapezium right angle=110,
  fill=blue!30,
  trapezium stretches=true,
];
\tikzstyle{arrow} = [-{stealth}];

\coordinate (start) at (0,0);
\tikzmath{
  \shift = -1.2;
}

\node (pyop3) [io] at (start) {Loop expression};
\node (transform_pyop3) [process] at ($(start)+(0,\shift*1)$) {Transform loop expression};
\node (lower_pyop3) [process] at ($(start)+(0,\shift*2)$) {Lower loop expression\\to loopy kernel};
\node (transform_loopy) [process] at ($(start)+(0,\shift*3)$) {Transform loopy kernel};
\node (lower_loopy) [process] at ($(start)+(0,\shift*4)$) {Lower loopy kernel to C};
\node (compile_code) [process] at ($(start)+(0,\shift*5)$) {Compile C code};
\node (compiled_function) [io] at ($(start)+(0,\shift*6)$) {Compiled function};

\draw [arrow] (pyop3) -- (transform_pyop3);
\draw [arrow] (transform_pyop3) -- (lower_pyop3);
\draw [arrow] (lower_pyop3) -- (transform_loopy);
\draw [arrow] (transform_loopy) -- (lower_loopy);
\draw [arrow] (lower_loopy) -- (compile_code);
\draw [arrow] (compile_code) -- (compiled_function);

\draw [arrow,densely dashed] (pyop3.east) to [bend left=40] node[midway,right,align=center,font=\footnotesize] {Call with data from\\loop expression} (compiled_function.east);

\end{tikzpicture}
\end{document}
