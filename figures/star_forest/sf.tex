\documentclass[tikz]{standalone}

\usepackage{tkz-euclide}
\usetikzlibrary{arrows,calc,graphs,graphdrawing,positioning,tikzmark,shapes.geometric,patterns.meta,decorations.pathreplacing}

\begin{document}

\begin{tikzpicture}

\tikzstyle {edge} = [draw=black,line width=1pt];
\tikzstyle {eghost} = [edge,dashed];
\tikzstyle {divider} = [dashed];

\tikzstyle {arrow} = [line width=.8pt,-{Stealth[sep=4pt]}];

% 2-cell mesh (unsplit)
\newcommand{\drawmesh}{%
  \tkzDefPoint(0,2){v0}
  \tkzDefPoint(3,4){v1}
  \tkzDefPoint(3,0){v2}
  \tkzDefPoint(6,2){v3}
  \tkzDefBarycentricPoint(v0=1,v1=1)\tkzGetPoint{e0}
  \tkzDefBarycentricPoint(v0=1,v2=1)\tkzGetPoint{e1}
  \tkzDefBarycentricPoint(v1=1,v2=1)\tkzGetPoint{e2}
  \tkzDefBarycentricPoint(v1=1,v3=1)\tkzGetPoint{e3}
  \tkzDefBarycentricPoint(v2=1,v3=1)\tkzGetPoint{e4}
  \tkzDefBarycentricPoint(v0=1,v1=1,v2=1)\tkzGetPoint{c0}
  \tkzDefBarycentricPoint(v1=1,v2=1,v3=1)\tkzGetPoint{c1}

  % edge DoFs
  \tkzDefBarycentricPoint(v0=2,v1=1)\tkzGetPoint{e0d0}
  \tkzDefBarycentricPoint(v0=1,v1=2)\tkzGetPoint{e0d1}
  \tkzDefBarycentricPoint(v0=2,v2=1)\tkzGetPoint{e1d0}
  \tkzDefBarycentricPoint(v0=1,v2=2)\tkzGetPoint{e1d1}
  \tkzDefBarycentricPoint(v1=2,v2=1)\tkzGetPoint{e2d0}
  \tkzDefBarycentricPoint(v1=1,v2=2)\tkzGetPoint{e2d1}
  \tkzDefBarycentricPoint(v1=2,v3=1)\tkzGetPoint{e3d0}
  \tkzDefBarycentricPoint(v1=1,v3=2)\tkzGetPoint{e3d1}
  \tkzDefBarycentricPoint(v2=2,v3=1)\tkzGetPoint{e4d0}
  \tkzDefBarycentricPoint(v2=1,v3=2)\tkzGetPoint{e4d1}

  \tkzDrawSegments[edge](v0,v1 v0,v2 v1,v2 v1,v3 v2,v3)

  % dividing lines
  \tkzDefShiftPoint[v2](.3,.6){mid}
  \tkzDefShiftPoint[mid](90:4){top}
  \tkzDefShiftPoint[mid](210:2.5){left}
  \tkzDefShiftPoint[mid](330:1.8){right}
  \tkzDrawSegments[divider](mid,top mid,left mid,right)

  % label processes
  \tikzstyle {node} = [font=\bfseries\large];
  \node [node,at={(1.6,.4)}] {0};
  \node [node,at={(4.7,.4)}] {1};
  \node [node,at={(3.7,-.4)}] {2};
}

% 2-cell mesh (split)
\newcommand{\drawpmesh}{%
  % process 0
  \tkzDefPoint(0,2){p0v0}
  \tkzDefPoint(3,4){p0v1}
  \tkzDefPoint(3,0){p0v2}
  \tkzDefBarycentricPoint(p0v0=1,p0v1=1)\tkzGetPoint{p0e0}
  \tkzDefBarycentricPoint(p0v0=1,p0v2=1)\tkzGetPoint{p0e1}
  \tkzDefBarycentricPoint(p0v1=1,p0v2=1)\tkzGetPoint{p0e2}
  \tkzDefBarycentricPoint(p0v0=1,p0v1=1,p0v2=1)\tkzGetPoint{p0c0}
  \tkzDrawSegments[edge](p0v0,p0v1 p0v0,p0v2 p0v1,p0v2)

  % edge DoFs
  \tkzDefBarycentricPoint(p0v0=2,p0v1=1)\tkzGetPoint{p0e0d0}
  \tkzDefBarycentricPoint(p0v0=1,p0v1=2)\tkzGetPoint{p0e0d1}
  \tkzDefBarycentricPoint(p0v0=2,p0v2=1)\tkzGetPoint{p0e1d0}
  \tkzDefBarycentricPoint(p0v0=1,p0v2=2)\tkzGetPoint{p0e1d1}
  \tkzDefBarycentricPoint(p0v1=2,p0v2=1)\tkzGetPoint{p0e2d0}
  \tkzDefBarycentricPoint(p0v1=1,p0v2=2)\tkzGetPoint{p0e2d1}

  % process 1
  \tkzDefShiftPoint[p0v1](1.5,0){p1v0}
  \tkzDefShiftPoint[p0v2](1.5,0){p1v1}
  \tkzDefShiftPoint[p1v1](3,2){p1v2}
  \tkzDefBarycentricPoint(p1v0=1,p1v1=1)\tkzGetPoint{p1e0}
  \tkzDefBarycentricPoint(p1v0=1,p1v2=1)\tkzGetPoint{p1e1}
  \tkzDefBarycentricPoint(p1v1=1,p1v2=1)\tkzGetPoint{p1e2}
  \tkzDefBarycentricPoint(p1v0=1,p1v1=1,p1v2=1)\tkzGetPoint{p1c0}
  \tkzDrawSegments[edge](p1v0,p1v2 p1v1,p1v2)
  \tkzDrawSegments[eghost](p1v0,p1v1)

  % edge DoFs
  \tkzDefBarycentricPoint(p1v0=2,p1v1=1)\tkzGetPoint{p1e0d0}
  \tkzDefBarycentricPoint(p1v0=1,p1v1=2)\tkzGetPoint{p1e0d1}
  \tkzDefBarycentricPoint(p1v0=2,p1v2=1)\tkzGetPoint{p1e1d0}
  \tkzDefBarycentricPoint(p1v0=1,p1v2=2)\tkzGetPoint{p1e1d1}
  \tkzDefBarycentricPoint(p1v1=2,p1v2=1)\tkzGetPoint{p1e2d0}
  \tkzDefBarycentricPoint(p1v1=1,p1v2=2)\tkzGetPoint{p1e2d1}

  % process 2
  % do some Pythagoras (sqrt(3) * .75)
  \tkzDefPoint(3.75,-1.299){p2v0}

  % dividing lines
  \tkzDefBarycentricPoint(p0v2=1,p1v1=1,p2v0=1)\tkzGetPoint{mid}
  \tkzDefShiftPoint[mid](90:5){top}
  \tkzDefShiftPoint[mid](210:2){left}
  \tkzDefShiftPoint[mid](330:2){right}
  \tkzDrawSegments[divider](mid,top mid,left mid,right)

  % label processes
  \tikzstyle {node} = [font=\bfseries\large];
  \node [node,at={(2.5, -.4)}] {0};
  \node [node,at={(5,-.4)}] {1};
  \node [node,at={(4.5,-1.5)}] {2};
}

% TODO split into two (or four?) diagrams, one for point SF and the other for DoF SF

% specific to point SF
\tikzstyle {point} = [minimum size=7pt,draw=black,line width=1pt];
\tikzstyle {powned} = [point,fill=black];
\tikzstyle {pghost} = [point,fill=white];

% top left (point SF)
\begin{scope}[yshift=7cm]
  \drawmesh
  \tkzDrawPoints[powned](v0,v1,v2,v3,e0,e1,e2,e3,e4,c0,c1)
\end{scope}

% top right (split point SF)
\begin{scope}[xshift=7cm,yshift=7cm]
  \drawpmesh

  % arrows
  \draw [arrow] (p0v1) -- (p1v0);
  \draw [arrow] (p0v2) -- (p2v0);
  \draw [arrow] (p1v1) -- (p2v0);
  \draw [arrow] (p1e0) -- (p0e2);

  \tkzDrawPoints[powned](p0v0,p0e0,p0e1,p0e2,p0c0)
  \tkzDrawPoints[pghost](p0v1,p0v2)

  \tkzDrawPoints[powned](p1v0,p1v2,p1e1,p1e2,p1c0)
  \tkzDrawPoints[pghost](p1v1,p1e0)

  \tkzDrawPoints[powned](p2v0)
\end{scope}

\tikzstyle {dof} = [minimum size=7pt,line width=1pt];

\tikzstyle {celldof} = [dof,draw=blue!60,fill=blue!60];
\tikzstyle {edgedof} = [dof,draw=red!70,fill=red!70];
\tikzstyle {vertdof} = [dof,draw=green!80,fill=green!80];

\tikzstyle {eghostdof} = [edgedof,fill=white];
\tikzstyle {vghostdof} = [vertdof,fill=white];

% bottom left (DoF SF)
\begin{scope}
  \drawmesh
  \tkzDrawPoints[celldof](c0,c1)
  \tkzDrawPoints[edgedof](e0d0,e0d1,e1d0,e1d1,e2d0,e2d1,e3d0,e3d1,e4d0,e4d1)
  \tkzDrawPoints[vertdof](v0,v1,v2,v3)
\end{scope}

% bottom right (split DoF SF)
\begin{scope}[xshift=7cm]
  \drawpmesh

  % arrows
  \draw [arrow] (p0v1) -- (p1v0);
  \draw [arrow] (p0v2) -- (p2v0);
  \draw [arrow] (p1v1) -- (p2v0);
  \draw [arrow] (p1e0d0) -- (p0e2d0);
  \draw [arrow] (p1e0d1) -- (p0e2d1);

  \tkzDrawPoints[celldof](p0c0)
  \tkzDrawPoints[edgedof](p0e0d0,p0e0d1,p0e1d0,p0e1d1,p0e2d0,p0e2d1)
  \tkzDrawPoints[vertdof](p0v0)
  \tkzDrawPoints[vghostdof](p0v1,p0v2)

  \tkzDrawPoints[celldof](p1c0)
  \tkzDrawPoints[edgedof](p1e1d0,p1e1d1,p1e2d0,p1e2d1)
  \tkzDrawPoints[eghostdof](p1e0d0,p1e0d1)
  \tkzDrawPoints[vertdof](p1v0,p1v2)
  \tkzDrawPoints[vghostdof](p1v1)

  \tkzDrawPoints[vertdof](p2v0)
\end{scope}

\end{tikzpicture}

\end{document}
