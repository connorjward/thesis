\documentclass[a4,10pt]{article}

\usepackage{mymacros}

\title{\pyop3: A Domain-Specific Language for Expressing Iterations over Mesh-like Data Structures}
\author{Connor J. Ward and David A. Ham}
\date{\today}

\begin{document}
  \maketitle

  \begin{abstract}

  \end{abstract}

  \section{Introduction}

  % Discuss DSLs

  \section{\pyop3}
  % mention sparsity?
  % mention ragged (good for star(vertex))

  \subsection{Representing mesh-like data layouts}

  % start with P3?

  \subsection{A language for loops}

  % give an example

  % typed multi-indices and substitution
  % context-sensitive

  \subsection{Compilation}

  \section{Performance}

  % all about minimising memory bandwidth, aim for roofline peak at all points
  % I need to implement the map compression algorithm (generically)

  % demonstrate that function calls (e.g. VecSetValues) are super slow? Not
  % really worth implementing.

  % key principle: optimisations are easy, synthesis is hard. Level of abstraction
  % facilitates memory optimisations.

  \section{Related work}

  % What other array languages/frameworks exist in this space?
  % * OP2
  % * PyOP2
  % * TACO (tree structure but not SoA-like)
  % * Halide?
  % * Liszt
  % * Simit
  % * Taichi
  % * AwkwardArray (ragged)
  % * DistArray (distributed)

  % ---

  % Does related work go here or in a separate section at the end?
  % Almost easier to put at the end as the concepts/distinctions are hard
  % to explain without having read the rest of the paper.


  \section{Conclusions}
\end{document}
